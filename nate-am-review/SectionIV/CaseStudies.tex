\subsubsection{In Situ Monitoring}
A study by Gobert et al. investigated using X-ray CT data to correlate with images obtained by a DSLR camera during the printing process. Gobert et al. printed a single part, containing several different features, and recorded the print results before and after every layer using a DSLR and eight different lighting conditions. The part was then characterized using X-ray CT to find the location and size of pores and inclusions. A regression model was trained between the location of the pores in the final part and the low resolution image data obtained during the build. 

One of the underlying assumptions of the study was that, even though the DSLR images did not record during the melting and solidification process, features would appear in each image that could be correlated back to the higher-resolution data in the CT images. The use of a regression model also meant that future images could be analyzed to predict whether or not pores had formed in various locations. The study by Gobert et al. was able to accurately predict whether or not an image contained a defect with 85\% accuracy after all the DSLR lighting conditions were combined into an ensemble classifier. 

With this type of in situ image recognition, defect formation can be detected during the printing process, precluding the need for certain post-characterization techniques. If defects form and are detected, then the printing process can be stopped, saving powder/feedstock, operating costs, and operating time. However, an even more desirable result is to first detect defect formation, then correct for it in situ. To do so, information will have to be assembled that relates the detection of defects, the manner in which the defects formed, and the prescribed fix to undo or stop the defects from growing. It is likely that combining all these information sources will require a significant leap in scientists' ability to apply machine learning algorithms to widely varying data sets and types. 
