\subsection{The Assumptions Behind Machine Learning}
Two assumptions are necessary when using machine learning:
\begin{enumerate}
\item \textit{The Similarity Hypothesis}: Parts manufactured at similar points in the design space will have similar properties.
\item \textit{The Relational Hypothesis}: A correlative relationship exists between the data input to the model and the response of the system.
\end{enumerate}

The similarity hypothesis is used to compare data and datasets, as well as to search and optimize through regression and classification algorithms; it is dependent upon mathematical tools for comparing similarity in a sensible way. Certain data types and data relationships can make similarity interpretation difficult. This is an active area of research within the data sciences

The second hypothesis is required for finding regression and classification functions that are physically accurate.

There are two types of machine learning covered in this review: unsupervised and supervised. Unsupervised learning will find trends in a dataset that are indicative of the underlying behavior. Supervised learning will learn a function $f(\mathbf{x}) = y$ that encodes part of the PSPP relationship.



