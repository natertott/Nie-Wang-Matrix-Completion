\subsection{Supervised Machine Learning}
In a \textit{supervised machine learning algorithm} the goal is to find a functional relationship that best approximates the underlying physical relationship $f(\mathbf{x}) = \mathbf{y}$. That is, supervised machine learning algorithms relate model inputs to labeled data. Functional relationships can take many forms, depending on the specific supervised ML algorithm being used. One method is to model the relationships as a vector product 
\eqn
\mathbf{X}T = \mathbf{Y}.
\label{map}
\equ
where $T$ is a vector of coefficients that weigh the machine inputs to approximate an entry in $\mathbf{Y}$. 

A researcher usually seeks this relationship through the measurements they have observed; in this case, the measurements are stored in the matrices of Eqn. \ref{initialmeasure}.
A common method to find a vector representation of $T$, and a critical element in most machine learning algorithms, is through least squares regression. Least squares regression finds $T$ through a minimization problem, given by
\eqn
\min || \mathbf{X}T - \mathbf{Y} ||_{2}^{2}.
\label{leastsquares}
\equ
Equation \ref{leastsquares} can be interpreted analogously to similarity measurements for unsupervised algorithms: the closer that $\mathbf{X}T - \mathbf{Y}$ is to zero, the more similar $T$ is to $f(\mathbf{x})$.

The methods of solving equation \ref{leastsquares} are many and varied; indeed, much of this review will focus on finding solutions to Eqn. \ref{leastsquares} for various problems throughout additive manufacturing.
The result is an approximation to the functional relationship $f(\mathbf{x}) = \mathbf{y}$.
A new point of interest in the design space $\mathbf{x'}$ can be chosen and its associated material property $\mathbf{y'}$ can be predicted by computing

\eqn
\mathbf{x'}T = \mathbf{y'}.
\equ

This simple example demonstrates how functional relationships can elucidate more information about design spaces.
Commonly used machine learning algorithms in materials science and engineering are given in Table \ref{ML}.
Note that the field of Machine Learning is evolving as fast as AM itself; hence Table \ref{ML} is by no means a comprehensive review of all machine learning algorithms, but rather is intended to provide a comparison between the form and function of some of the most widely adopted algorithms in materials science and engineering.

