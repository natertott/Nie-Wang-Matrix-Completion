\section{Conclusions}
Materials informatics has demonstrated great success in design and optimization of many material systems. Additive manufacturing is primed to benefit from the same algorithms and statistical models. Many of the major obstacles that lie ahead in additive manufacturing -- full scale modeling, integrated design, feedback and control -- can be tackled through the adoption of machine learning. However, machine learning itself is not the end-all-be-all of manufacturing. There are many obstacles in the application of machine learning itself that will need to be addressed along the way.

\subsection{Key Application Areas for Machine Learning in Additive Manufacturing}
\begin{itemize}
	\item \textbf{Coupled Physics-Statistics Models:} The original goal of materials informatics, dating back to high throughput thin-film studies in the 1990s, was to model material process-structure-property relationships that were highly complicated and lacked a single governing physical theory. Additive manufacturing is the embodiment of a complicated physical system, where governing equations across optics, fluid mechanics, solid mechanics, thermodynamics, and kinetics have to be incorporated into one model. Machine learning can build computationally accessible surrogate models of more complicated physical systems that are useful for engineering and design. 
	
	\item \textbf{Materials Design:} Materials design through machine learning has already been applied in a wide range of fields cited here, including thermoelectrics, photovoltaics, semiconductors, Heusler compounds, and many, many more. Design in these fields typically focuses around combinatorial studies of compositions, crystal structures, and a material response. Materials are manufactured through a wide variety of techniques but optimization is rarely applied to the manufacturing method itself, just the materials used in manufacturing. In additive, not only does the material system need to be tailored but the conditions of manufacturing also need optimization. Materials properties to consider range from composition and atomic properties to phase kinetics. Manufacturing optimization includes the energy density used, deposition rate, feedstock supply mechanism, and more. Machine learning can integrate optimization across these separate design considerations. Process optimization is likely to include in situ control.
	
	\item \textbf{Automated Process Control:} There are many variables to monitor and keep track of in the additive processes. There are equally many sensors and measurement techniques to monitor the process. Advancements in signal processing and computer vision must be taken advantage of to build incorporating process control models. Intelligent feedback and control for additive can simultaneously integrate and understand multiple signal types \textbf{and} optimize on multiple objective functions simultaneously. Taking full advantage of the promises of AM -- topologically optimized geometries, functionally graded materials, minimized design-to-fly time -- will require tight control over the manufacturing process.
\end{itemize}

\subsection{Further Developments are Needed in Both Additive Manufacturing and Machine Learning}
\begin{itemize}
	\item \textbf{Data Sharing Infrastructure:} Programs like the Materials Project, AFLOW, and OQMD have accelerated the rate at which materials design can occur, as well as the rate at which scientific data is shared. The democratization of data has allowed many different research teams to search through the materials design space in search of new materials, to great success. The same type of democratization is possible in additive if infrastructure exists for sharing of AM data. However, standardization of AM data types should be addressed before data can be shared in a useful, meaningful way.

	\item \textbf{Curation of Data and AM Standards:} Success in applying data-driven approaches is tied tightly to the quality of data being used. Even data that has been collected with the highest care and precision can be detrimental to a model if it is labeled incorrectly or inconsistently. Work is proceeding in standards development for additive manufacturing \cite{Seifi2016}. However, additive manufacturing technology development has sometimes proceeded faster than standardization. Care needs to be taken in developing AM standards that are consistent across manufacturing devices and can also account for developments in the broader technology.
	
	\item \textbf{Experimental Measurement and Sensor Development:} While in situ measurement devices are widespread, the time and length scales of additive manufacturing can push the limits of current high-end sensors. Imaging methods that can resolve the fast, dynamic, microscale melt pools of additive would allow for a huge leap in process monitoring and control. Equally important is developing methods of determining temperature history throughout the duration of builds. Both of these technologies are crucial for fine control over the additive process. 

	\item \textbf{Physics Informed Data Driven Models:} Additive manufacturing has developed amazingly over the past few decades thanks to traditional scientific and engineering approaches in many different fields. Modeling AM using classical thermal, mechanical and kinetic models has allowed for most of the successes in AM. This review is suggesting that machine learning be used as a complementary tool to these traditional approaches. It would be unwise to completely ignore physical theories which have shown applicability in AM. Rather, machine learning algorithms should be built around currently existing models. There are equally rich mathematical frameworks in both materials science and machine learning which are currently being utilized separately. The physics of AM at all length scales -- solidification, phase kinetics, heat transfer, solid mechanics, etc. -- should be used as first principles for building physics-informed statistical models. Many in the materials science community have considered how to use domain knowledge to build better informatics models \cite{Deaven1995, Morris1996, Wagner2016, Raccuglia2016}. The same should be applied to additive manufacturing.
\end{itemize}