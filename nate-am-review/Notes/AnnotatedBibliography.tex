\documentclass[notitlepage]{revtex4-1}
\usepackage{amsmath}
\usepackage{amssymb}

\usepackage{xcolor}
\usepackage{bibentry}
\usepackage{braket}
\nobibliography*


\begin{document}
\title{Annotated Bibliography \\ Applied Physics Review Paper \\ Nathan S Johnson et al.}
\maketitle

%This document is an annotated bibliography for the paper ``Applications of Machine Learning in Additive Manufacturing" by N.S. Johnson et al., to appear in Applied Physics Review when hell freezes over. \\

To be organized alphabetically by citation key.

%~~~~~~~~~~~~~~~~~~~~~~~~~~~~~~~~~
% Begin the bibliography here
%~~~~~~~~~~~~~~~~~~~~~~~~~~~~~~~~~

\begin{itemize}

	\item \bibentry{Abdelrahman2017} \textbf{Details a computer vision approach to detecting flaws during builds of LPBF. \begin{itemize}
		\item Computer vision
		\item In situ monitoring
	\end{itemize}}
	
	\item \bibentry{Alon1997} \textbf{Details the conditions necessary for success in some classes of learners; specifically details conditions necessary on Glivenko-Cantelli classes.}
	\item \bibentry{Antonysamy2013} \textbf{Study of grain structure and texture in powder bed e-beam Ti-6Al-4V; observes prior beta grain grown aligned in $\braket{0001}$ normal to build direction; contour layer has different grain structure.\begin{itemize}
		\item Parametric analysis -- build geometry (angle, wall thickness, geometry, etc.) $\to$ grain structure
	\end{itemize}}
	
	\item \bibentry{Anijdan2006} \textbf{Used an ANN and GA to correlate chemical composition and cooling rate to porosity in Al-Si.\begin{itemize}
		\item Alloy Design
	\end{itemize}}
	
	\item \bibentry{Bartel2018} \textbf{Uses SISSO to generate a descriptor which can predict Gibbs free energy from entries in the ICSD}
	
	\item \bibentry{Baufeld2011} \textbf{Compares microstructure and mechanical properties of Ti-6Al-4V fabricated by shaped metal deposition versus additive layer manufacturing (wire-based AM)\begin{itemize}
		\item Parametric analysis -- manufacturing process (SMD vs laser DED) $\to$ microstructure
	\end{itemize}}
	
	\item \bibentry{Bay2008} \textbf{Citation of original paper for speeded-up robust feature (SURF) algorithm. \begin{itemize}
		\item Computer vision
	\end{itemize}}
	
	\item \bibentry{Behler2015} \textbf{Details theoretical development of atomic potentials using neural networks; method of model approximation using machine learning (instead of DFT).}
	
	\item \bibentry{Bertoli2017} \textbf{Study using high speed imaging to monitor melt pools in SLM; experimentally determines cooling rates. \begin{itemize}
		\item In sit monitoring
	\end{itemize}}
	
	\item \bibentry{Berumen2010} \textbf{Uses basic detection techniques to study and monitor melt pool in SLM; has good explanation of physics of melt pools. \begin{itemize}
		\item In situ monitoring
	\end{itemize}}
	
	\item \bibentry{Bessa2017} \textbf{Framework for design and modeling of new material systems -- framework is as follows: 1. design of experiments, optimizing on microstructure, phase properties, and `external conditions,' 2. computational analyses based on sample design, generate material response database, 3. machine learning to database of generated materials}
	\item \bibentry{Beuth2001} \textbf{Outlines the creation of process maps for predicting melt pool size, thermal gradients, maximum residual stresses, and more.}
	\item \bibentry{Bhadeshia2009} \textbf{Addresses the use of neural networks in materials science; presents a loose guideline for maximising the impact of neural network models created.}
	
	\item \bibentry{Bi2013} \textbf{Study on process monitoring in laser aided additive manufacturing; looks at powder density, beam focus, defect generation, oxidation, dimensional accuracy. 
	\begin{itemize}
		\item Alloy Design
	\end{itemize}}
	
	\item \bibentry{Boley2015} \textbf{Calculates absorption of laser energy on metal powder; investigates how distribution of powder impacts laser absorption.
	\begin{itemize}
		\item Featurization application 
		\item Particle size distribution
	\end{itemize}}
	
	\item \bibentry{Boley2016} \textbf{Research article from LLNL on absorptivity measurements of metal powders; has applications to simulation of LPBF process.
	\begin{itemize}
		\item Alloy design
	\end{itemize}}
	
	\item \bibentry{Bontha2006} \textbf{Develops thermal process maps relating cooling rate and thermal gradient to laser power and velocity; looking at a LENS process; uses 2D rosenthal solution for the melt pool depth; compares with 2D nonlinear thermal finite element model \begin{itemize}
		\item Parametric analysis -- laser power \& velocity $\to$ cooling rate temperature gradient
	\end{itemize}}
	
	\item \bibentry{Bontha2009} \textbf{Study of size effects on microstructure formation in electron beam on a generalized additive process; investigations heat transfer using a 3D rosentathal solution for a moving point source; develops thermal process maps to predict microstructure type (columnar versus equiaxed).\begin{itemize}
		\item Parametric analysis -- cooling rate and temperature gradient $\to$ grain morphology
	\end{itemize}}
	
	\item \bibentry{Breiman1996} \textbf{Citation of an original paper on random forests.}
	\item \bibentry{Breiman2001} \textbf{Another citation of a paper on random forests.} 
	\item \bibentry{Brice2018} \textbf{A design of experiments study whereby nine different experiments were conducted on an aluminum alloy to determine impact of composition, substrate temperature on magnesium loss; found that changes in Mg content can impact microstructure significantly.\begin{itemize}
		\item Design of experiments
		\item Parametric analysis -- composition, substrate heating $\to$ microstructure
	\end{itemize}}
	
	\item \bibentry{Bro2014} \textbf{Overview of PCA methods.\begin{itemize}
		\item Dimensionality reduction via matrix factorization review
	\end{itemize}}
	\item \bibentry{Butler2018} \textbf{A review article on applications of machine learning in molecular and materials sciences; outlines machine learning techniques that are suitable for certain research questions.}
	\item \bibentry{Carrete2014} \textbf{Example of the success of random forests in predicting material behavior.}
	\item \bibentry{Ceder1998} \textbf{An early example of how ab initio searches can be used for discovery of compounds with certain properties}
	\item \bibentry{Chakraborti2004} \textbf{Review of genetic algorithms in materials design; divided into three main sections: review of genetic algorithms, genetic algorithms in materials design, genetic algorithms in materials processing.}
	\item \bibentry{Chen2002} \textbf{Describes phase field modeling for modeling microstructure evolution at the mesoscale.\begin{itemize}
		\item Phase field model
		\item ICME 
	\end{itemize}}
	\item \bibentry{Chen2012} \textbf{New class of Li-ion battery cathode materials discovered through high throughput ab initio searches.}
	
	\item \bibentry{Chen2016} \textbf{Observation of texture, microstructural characteristics of SLM Inconel 718; studies orientation of dendritic grains as a function of base temperature; looks are cracking in heat affected zone.\begin{itemize}
		\item Parametric analysis -- substrate/base cooling temperature $\to$ hot cracking behavior
	\end{itemize}}
	
	\item \bibentry{Cherian2000} \textbf{Covers use of a neural network to predict mechanical properties from parts formed using powder metallurgy.}
	\item \bibentry{Cherry2015} \textbf{Study on SLM of 316L SS; investigated impact of process parameters on surface roughness, porosity, and hardness. \begin{itemize}
		\item Parametric analysis -- laser energy density $\to$ surface finish, microstructure, density, hardness
		\item More specifics: they varied laser energy density by varying exposure time and point distance (amount of overlap between hatches)
	\end{itemize}}
	
	\item \bibentry{Chowdhury2016} \textbf{Applied computer vision and machine learning to automate recognition of microstructural features; visual bag of words, texture, pre-trained convolutional neural network used for feature extraction; classified dendritic versus non dendritic using support vector machine, voting, nearest neighbor, and random forests. \begin{itemize}
		\item In situ monitoring
		\item Computer vision
	\end{itemize}}
	
	\item \bibentry{Ciobanu2005} \textbf{A citation for Ciobanu that Stebner asked for.}
	\item \bibentry{Collins2016} \textbf{Attempts to establish a framework that incorporates processing variables, alloy composition, and thee resulting microstructure.}
	\item \bibentry{Curtarolo2003} \textbf{Uses principal component analysis on ab initio energy calculations to study correlations in crystal structure and formation energy.\begin{itemize}
		\item Dimensionality reduction
		\item Matrix factorization
	\end{itemize}}
	\item \bibentry{Curtarolo2005} \textbf{A study on the accuracy of ab initio methods.}
	\item \bibentry{Curtarolo2012} \textbf{Introduces AFLOW.}
	\item \bibentry{Curtarolo2012a} \textbf{Introduces AFLOWLIB.}
	\item \bibentry{Curtarolo2013} \textbf{A review of high throughput ab initio methods in DFT and how they can be beneficial to materials design.}
	\item \bibentry{Dai2014} \textbf{Simulation of temperature profile and densification of SLM using a finite volume method; models transition from powder to solid as well as surface tension induced by temperature gradient; studies impact of power; shows asymmetric temperature distribution wrt laser scanning area.\begin{itemize}
		\item Finite element thermomechanical model
		\item ICME model
	\end{itemize}}
	\item \bibentry{Deaven1995} \textbf{Describes a method of optimizing the charge balance between atoms based on position using a genetic algorithm approach. Pretty cool paper if you ask me.}
	\item \bibentry{DeCost2015} \textbf{Uses bag of visual features to create synthetic microstructural images; classifies generated images using a support vector machine. \begin{itemize}
		\item Computer vision
	\end{itemize}}
	
	\item \bibentry{DeCost2017} \textbf{Characterizes powder feedstock materials; uses SIFT for feature extraction; classifies powder particles. \begin{itemize}
		\item Computer vision
	\end{itemize}}
	
	\item \bibentry{DeCost2017a} \textbf{Applies bag of visual words to classify images of powder particles based on particle size distribution, morphology, and surface texture. \begin{itemize}
		\item Computer vision
	\end{itemize}}
	
	\item \bibentry{DeCost2017b} \textbf{Uses t-SNE to graphically explore datasets of microstructures at a range of scales in Ultrahigh carbon steel; ultrahigh carbon steel is known for having complex, hierarchical structures. \begin{itemize}
		\item Dimensionality reduction
	\end{itemize}}
	
	\item \bibentry{Dehoff2015} \textbf{Title kind of says it all; uses careful control of thermal gradient (G) and solidification velocity (R) in order to control crystallographic growth direction if Inconel 718; uses neutron diffraction to characterize texture.}
	\item \bibentry{DeJong2015} \textbf{Introduces a database for discovery of piezoelectric materials.}
	
	\item \bibentry{Delgado2012} \textbf{Full-factorial design of experiments for investigating manufacturing parameters on macroscopic mechanical properties; an `early' paper investigating impact of processing parameters on quality \begin{itemize}
		\item Design of Experiments
		\item Parametric analysis -- scan speed, layer thickness, build orientation $\to$ distortion, surface roughness, mechanical properties
	\end{itemize}}
		
	\item \bibentry{Deng2018} \textbf{Picks certain descriptors to predict sintered density of Cu-Al alloy using multilayer perceptron, neural networks; descriptors were chosen from processing parameters, composition, property of raw materials.}
	
	\item \bibentry{Denlinger2015} \textbf{In situ measurements of part distortion are made for titanium and nickel alloys as a function of dwell time between deposition of material; laser based powder bed study; also studied residual stress build up as a function of dwell time. \begin{itemize}
		\item Parametric analysis -- laser dwell time $\to$ part distortion from residual stresses
		\item In situ
	\end{itemize}}
	
	\item \bibentry{DePablo2014} \textbf{This article is from an NSF workshop in 2014 on the MGI; it details areas of recent growth, areas of need, central challenges related to, and perspectives on materials informatics, under the umbrella of the `Materials Genome Initiative'; focuses mainly on modeling.}
	\item \bibentry{Ding2011} \textbf{Large scale ($>$500mm) finite element model of gas WAAM; focused on residual stress development and temperature history; generated temperature vs time and location for single-wall builds; also simulated distortion \begin{itemize}
		\item Large scale model
		\item Finite element residual stress
		\item ICME model
	\end{itemize}}
	
	\item \bibentry{Dudiy2006} \textbf{Study on using genetic algorithms to design a material with a specific band gap; inverse band gap design; good explanation of genetic algorithms using natural language; great paper for alloy design.
	\begin{itemize}
		\item Alloy Design
	\end{itemize}}
	
	\item \bibentry{Egmont-Petersen2002} \textbf{Literature review of neural network methods from the early 2000s. \begin{itemize}
		\item Theory
		\item Convolutional neural network review
	\end{itemize}}
	
	\item \bibentry{Efron2014} \textbf{A text on bootstrap methods of estimating standard errors with consideration of model selection.}
	\item \bibentry{Fischer2006} \textbf{They essentially develop their own data mining approach to predict the probability of a given composition forming a certain stable crystal structure.
	\begin{itemize}
		\item Alloy Design
	\end{itemize}}
	\item \bibentry{Flores-Livas2017} \textbf{I am not sure how I found this article; study on using a unique representation of crystal structures as part of a search algorithm; searches for stable crystal structures as a function of pressure; has a good lit review on machine learning methods for crystal structure prediction.}
	\item \bibentry{Foster2015} \textbf{Discusses HPC methods for the types of analysis that go on at Argonne National Laboratory.}
	
	\item \bibentry{Franceschetti1999} \textbf{They describe an algorithmic approach to predicting the crystal structure of various compositions grown using molecular beam epitaxy or metallographic chemical vapor deposition; high-throughput study.
	\begin{itemize}
		\item Alloy Design
	\end{itemize}}
	
	\item \bibentry{Francois2017} \textbf{This is a review of modeling methods and challenges in wire feed and powder bed AM; covers a range of scales from microscale to macroscale; compiled by the three major DOE labs. \begin{itemize}
		\item ICME Review 
	\end{itemize}}
	\item \bibentry{Frazier2014} \textbf{Big ass, low level review of additive manufacturing; covers all the processes you would ever need.}
	\item \bibentry{Gao2015} \textbf{Review of state of the art for AM and challenges for AM; reviews all 3D printing applications, not just metal; goes into the process of how a CAD file is used to create a 3D printed part; discusses societal and logistical impacts.}
	\item \bibentry{Gatys2015} \textbf{Generates a database of synthetic textures for use with image recognition in human brains; more focused on neuroscience.}
	
	\item \bibentry{Gaultois2016} \textbf{Case study of Citrine's web-based random forest platform on a database of thermoelectric materials. \begin{itemize}
		\item Model reduction
	\end{itemize}}
	
	\item \bibentry{Gaynor2016} \textbf{Topology optimization specifically to reduce the amount of sacrificial support structures that need to be used.
	\begin{itemize}
		\item Topology optimization
	\end{itemize}}
	
	\item \bibentry{Ghiringhelli2015} \textbf{A great paper; discusses finding improved descriptors for predicting material properties.}
	\item \bibentry{Gilmer1998} \textbf{Review article on thin film deposition methods; useful when discussing high throughput investigations.}
	
	\item \bibentry{Gobert2018} \textbf{Study of computer vision and machine learning methods for predicting pore/inclusion formation during LPBF process; compares images taken during different layer manufacture to slices in an XCT of the finished part; classifies layers as having a defect or not using a support vector machine. \begin{itemize}
		\item In situ
		\item Computer vision
	\end{itemize}}
	
	\item \bibentry{Gong2015} \textbf{Phase field model of EBAM process; specifically looking at microstructure evolution; solidification of Ti-6Al-4V; looked at effect of undercooling on dendrite growth. \begin{itemize}
		\item Phase field microstructure evolution, solidification
		\item ICME model
	\end{itemize}}
	
	\item \bibentry{Gopakumar2018} \textbf{Use of Maximin algorithm to predict multiple material properties from composition, specifically in the form of Pareto fronts, Ashby-chart like plots; dense in theory.}
	
	\item \bibentry{Gouge2018} \textbf{Review of AM processes and modeling efforts behind them; specific to finite element methods. \begin{itemize}
		\item FE Review for AM modeling
		\item ICME
	\end{itemize}}
	
	\item \bibentry{Grefenstette1986} \textbf{Use of a genetic algorithm to teach a machine to perform a generic optimization task.}
	\item \bibentry{Guo2013} \textbf{A review of AM processes and challenges.}
	\item \bibentry{Hart2005} \textbf{Uses a genetic algorithm to find model-accurate coarse grained Hamiltonians of many body systems; relevant to crystal structure prediction and material design.}
	
	\item \bibentry{Hautier2010} \textbf{Uses the cumulant distribution function of Fischer et al. to perform a search through a database of ab initio generated crystal structures.
	\begin{itemize}
		\item Alloy Design
	\end{itemize}}
	
	\item \bibentry{Hayes2017} \textbf{Develops a model to predict tensile strength of wire feed AM Ti-6Al-4V from process parameters; presents constitutive equations.}
	\item \bibentry{Hernandez-Nava2015} \textbf{Investigation of metallic foams (similar to lattice structures) produced via additive manufacturing.}
	
	\item \bibentry{Ho1998} \textbf{Uses a genetic algorithm to find the stable crystal structure for clusters of silicon.
	\begin{itemize}
		\item Alloy Design
	\end{itemize}}
	
	\item \bibentry{Hohenberg1964} \textbf{A seminal paper on Hohenberg-Kohn-Sham DFT.}
	
	\item \bibentry{ICME2013} \textbf{Conferences proceedings from the World Congress on ICME. \begin{itemize}
		\item ICME
	\end{itemize}}
	
	\item \bibentry{Ikeda1997} \textbf{Use a genetic algorithm to design an alloy composition with specific material properties; has a great motivation/intuition for setting up GAs to achieve a desired optimization condition.
	\begin{itemize}
		\item Alloy Design
	\end{itemize}}
	
	\item \bibentry{Jain2011} \textbf{Outlines the infrastructure for DFT calculations and machine learning on DFT results; infrastructure would later become Materials Project; has good discussion of how certain mathematical properties in crystal structure representation are amenable to machine learning algorithms.}
	\item \bibentry{Jain2013} \textbf{Introduces the Materials Project, outline, scope, goals, etc.}
	\item \bibentry{Jain2016} \textbf{Application of data mining algorithms to several different materials databases.}
	\item \bibentry{Jia2014} \textbf{Studies relationship of processing parameters to microstructure and some mechanical properties; looks at processing conditions impact on microstructure and properties; observed balling at lower laser energy density; found good manufacturing parameters for near-fully-dense parts; measured microhardness and wear properties. \begin{itemize}
		\item Parametric analysis -- laser parameters (power, scan speed, thus density) $\to$ density, microstructure, microhardness, oxidation resistance
	\end{itemize}}
	
	\item \bibentry{Jin2008} \textbf{Theory paper on multi objective optimization; mainly discusses re-representing NN's as multiobjective optimization problems for both supervised and unsupervised tasks; relevant to fingerprint modeling section \begin{itemize}
		\item Theory
		\item Machine learning review
		\item Multiobjective optimization
	\end{itemize}}
	
	\item \bibentry{Johannesson2002} \textbf{Uses a genetic algorithm to operate on iterations of DFT simulations; specifies GA such that specific final properties can be generated.}
	
	\item \bibentry{Jokhakar2016} \textbf{Applies a litany of machine learning approaches to predict when defects will occur during steel manufacturing; monitors various aspects of continuous steel manufacturing process and takes measurements; used random forests, neural networks, support vector machines, and ensemble modeling to predict defect formation. \begin{itemize}
		\item In situ monitoring
	\end{itemize}}
	
	\item \bibentry{Kalidindi2016} \textbf{Basic overview of data mining and materials informatics as it relates to the Materials Genome Initiative; identifies toolsets which may be useful in integrating aspects of materials science.}
	
	\item \bibentry{Kalinin2015} \textbf{A review/progress article on how large datasets of materials microstructures/images can be used for materials design or process design; focused specifically around discovery of new materials; identifies several different machine learning algorithms. \begin{itemize}
		\item Computer vision
	\end{itemize}}
	
	\item \bibentry{Kamath2016} \textbf{Uses a data-driven surrogate model to identify important variables for certain final properties in SLM; varies the laser parameters, the powder bed parameters (bed thickness, particle size, etc,), and material properties (alloy); approach is to find areas of design space which produce melt pools of appropriate geometry; performed regression using random forests to predict melt pool depths.}
	\item \bibentry{Khairallah2016} \textbf{Study on recoil pressure and Marangoni convection in LPBF of 316L SS; Looks at particle melting, partial melting, and pore formation; also investigates denudation zone near laser path; discusses remedies to these problems. \begin{itemize}
		\item Melt pool, powder model
		\item ICME
	\end{itemize}}
	
	\item \bibentry{Khorasani2018} \textbf{Introduces the Taguchi method of experimental design; used design of experiments to compare processing parameters (laser power, scan speed, hatch space, laser pattern angle coupling, post processing) to material properties (density, strength, elongation, average surface area). \begin{itemize}
		\item Parametric analysis -- laser parameters (above) $\to$ material properties (above)
	\end{itemize}}
	
	\item \bibentry{Kim2016} \textbf{This appears to be a paper which used random forests on a large dataset to predict dialetric breakdown.}
	\item \bibentry{Kim2017} \textbf{Presents a dataset of synthesis parameters for materials; for use in conjunction with machine learning of material structures from ab initio DFT.}
	\item \bibentry{King2014} \textbf{Ex situ analysis of single tracks builds of LPBF; varied the energy density of the laser in different deposition; characterized the transition between conduction-mode melting and keyhole-mode melting based on incident energy density.}
	\item \bibentry{King2015} \textbf{This is a massive review article on the physics, modeling, and materials science of LPBF; truly a huge resource. \begin{itemize}
		\item ICME Review for LPBF
	\end{itemize}}
	\item \bibentry{King2015a} \textbf{Title says it all; a review of modeling of AM at LLNL.}
	\item \bibentry{Kirklin2013} \textbf{A high throughput combinatorial investigation of stable structures for use as anode materials; screening/search algorithm called grand canonical linear programming was used.}
	\item \bibentry{Kirkpatrick1983} \textbf{This article describes optimization by simulated annealing for DFT.}
	\item \bibentry{Kohn1965} \textbf{Seminal paper on Hohenberg-Kohn-Sham DFT.}
	\item \bibentry{Koinuma2004} \textbf{This is a review of combinatorial (high throughput ab initio) approaches to designing materials; covers both analysis methods and manufacturing methods for screening material properties; focus on inorganic functional materials.}
	\item \bibentry{Kolmogorov2006} \textbf{Performs an ab initio screening of stable structures for a specific subset of materials -- layered metal borides.}
	\item \bibentry{Kulkarni2004} \textbf{A good study of how prior mat sci knowledge can be used to set up a machine learning study efficiently and in an interpretable way; uses previously developed theoretical and phenomenological models to design microstructures with specific material properties using genetic algorithms; materials by design.}
	
	\item \bibentry{Kundin2015} \textbf{Phase field modeling of rapidly solidifying Inconel 718; evaluates a phenomenological model of dendrite growth as a function of undercooling. \begin{itemize}
		\item Parametric analysis with a model -- constitutional undercooling $\to$ dendrite arm spacing
		\item Phase field
		\item ICME
	\end{itemize}}
	
	\item \bibentry{Kusne2015a} \textbf{Not quite sure if this will actually make it into the review; they apply an algorithm called mean shift theory in order to perform on-the-fly analysis of diffraction results from a combinatorial library of synethsized materials; they were looking for a rare-earth free permanent magnet.}
	\item \bibentry{Kwon2018} \textbf{Used a deep neural network to classify melt pool images; database of images was taken across 6 different laser power levels; the study is more concerned with training a successful NN than the AM or mat sci of the problem.}
	\item \bibentry{Lambrakos2009} \textbf{Presents a model for inverse modeling of heat transfer in layer-by-layer deposition processes; aimed toward prediction of temperature histories of final parts based on geometry.}
	
	\item \bibentry{Landrum2003} \textbf{Uses decision trees in order to predict whether or not ordered/disordered phases are ferromagnetic.
	\begin{itemize}
		\item Alloy Design
	\end{itemize}}
	
	\item \bibentry{Langelaar2016} \textbf{A topology optimization algorithm which takes constraints of AM specifically into mind; excludes geometries which are impossible to manufacture in AM (unsupported structures); compared with traditional topology optimization algorithms. \begin{itemize}
		\item Topology optimization
	\end{itemize}}
	
	\item \bibentry{Langelaar2017} \textbf{Improvement over the algorithm presented in \cite{Langelaar2016} from what I can tell.
	\begin{itemize} 
		\item Topology optimization
	\end{itemize}}
	
	\item \bibentry{Leong2003} \textbf{This paper prospects using solid freeform fabrication to produce biomedical devices.}
	
	\item \bibentry{Li2014} \textbf{Parametric analysis of processing conditions on thermal history of SLM; finite element model; specifically looking at effects of laser power and scan speed on cooling rate and solidification velocity; also looked at wettability of melt pools and the formation of micropores; compared with prints made at same processing conditions as model. \begin{itemize}
		\item Parametric analysis -- laser parameters (above) $\to$ thermal history
		\item FE Model
		\item ICME
	\end{itemize}}
	
	\item \bibentry{Ling2017} \textbf{Outlines the backbone of Citrines recommendation system for experiments; uses sequential learning approach based on random forests; found the optimal choice with three times fewer tests needed.}
	\item \bibentry{Ling2017a} \textbf{This paper details the use of convolutional neural networks for classify microstructures; generalized some aspects of classifying microstructure images.}
	\item \bibentry{Liu2006} \textbf{Pretty early paper on machine learning of database information, specifically DFT and CALPHAD based modeling; informatics approach used is linear regression.}
	\item \bibentry{Liu2014} \textbf{This is a machine learning approach to inverse design of microstructures, optimizing on magnetic properties; framework involves random data generation, feature selection, and classification; algorithms used include maximum likelihood, genetic algorithm, exhaustive search, better than random guided search, and linear programming.} {\color{red} A good paper to read through thoroughly.}
	
	\item \bibentry{Long2007} \textbf{They use PCA on a dataset of microdiffraction results to find characteristic XRD patterns for a given material system; they then rapidly identify new material systems based on a PCA analysis of their microdiffraction patterns. \begin{itemize}
		\item Matrix factorization
		\item Dimensionality reduction
	\end{itemize}}
	
	\item \bibentry{Long2009} \textbf{Uses non-negative matrix factorization to rapidly classify a dataset of microdiffraction patterns; compared with PCA results from \cite{Long2007}; provides a more interpretable dimensionality reduction of microdiffraction patterns. \begin{itemize}
		\item Matrix factorization
		\item Dimensionality reduction
	\end{itemize}}
	
	\item \bibentry{Lowe2004} \textbf{An original paper on SIFT. \begin{itemize}
		\item Computer vision
	\end{itemize}}
	
	\item \bibentry{Lu2019} \textbf{A thermomechanical models of the laser cladding process of Ti-6Al-4V; \begin{itemize}
		\item Process modeling
		\item Laser cladding Ti-6Al-4V
	\end{itemize}}
	
	\item \bibentry{Lubbers2016} \textbf{Study on using pre-trained convolutional neural networks to characterize synthetic microstructure images; uses manifold learning to embed characterizations into low-dimensional space; low-dimensional embedding reveals information about parameters which generated the images; could be applied to real microstructures maybe? \begin{itemize}
		\item Computer vision
	\end{itemize}}
	
	\item \bibentry{Mani2017} \textbf{A review on needs for in situ feedback and control; focused on feed-forward and closed-loop monitoring of AM systems. \begin{itemize}
		\item In situ monitoring
	\end{itemize}}
	
	\item \bibentry{Mannodi-Kanakkithodi2016} \textbf{Uses fingerprint methods (including a genetic algorithm) for design of polymer chains to be used as dialetric material; Nature Scientific Reports article \begin{itemize}
		\item Alloy design
		\item Materials design
	\end{itemize}}
	
	\item \bibentry{Manvatkar2014} \textbf{Investigation of spatial variation of melt pool geometry, cooling rate, peak temperature in various layers throughout a LENS build; used cooling rates and solidification parameters to predict hardness; compared with experimental results.}
	
	\item \bibentry{Martin2017} \textbf{This paper details the design of an aluminum alloy specifically for additive manufacturing; they used a random search to find lattice-matched nucleants for aluminum alloys.
	\begin{itemize}
		\item Alloy Design
	\end{itemize}}
	
	\item \bibentry{Martukanitz2014} \textbf{Presents an ICME approach to modeling AM; model incorporates thermal, mechanical, and material responses during manufacturing; discusses process maps, impacts of energy density on solidification, heat transfer in models, mass transfer in models, solid state phase transformations which occur during manufacturing. \begin{itemize}
		\item Full ICME approach
	\end{itemize}}
	
	\item \bibentry{McKeown2016} \textbf{In situ investigation of solidification velocities for rapidly solidifying Al-Cu alloys; performed dynamic TEM to observe solidification front, microstructure evolution, and instability at liquid-solid interface. \begin{itemize}
		\item In situ monitoring
	\end{itemize}}
	
	\item \bibentry{Mehta2018} \textbf{An arXiv preprint with reviews of many, many different machine learning algorithms.}
	\item \bibentry{Mellor2014} \textbf{Perspective for implementing AM at a wide scale from a socio-technical perspective; identifies personnel infrastructure which needs to be in place for AM to succeed.}
	\item \bibentry{Meredig2014} \textbf{Application of ensembles of decision trees to predict formation energies of various materials; compares the results with DFT simulations.}
	\item \bibentry{MGI} \textbf{A white paper on the materials genome initiative.}
	
	\item \bibentry{Michaleris2014} \textbf{Finite element model developed for studying heat transfer in a generalized layer-by-layer AM system; uses inactive and active elements for model; looks at heat transfer as wall is built up. \begin{itemize}
		\item FE thermomechanical model
		\item ICME
	\end{itemize}}
	
	\item \bibentry{Moelans2008} \textbf{A review on phase field modeling approaches to microstructure evolution.}
	\item \bibentry{Morgan2005} \textbf{Discusses some of the challenges with setting up high through ab initio investigations in DFT. Discusses data mining techniques to find formation energies and stable crystal structures.}
	\item \bibentry{Morris1996} \textbf{Cleverly applies a genetic algorithm approach to the problem of finding the minimum energy stable configuration of N many electronic charges; extends up to $N = 200$; good paper for demonstrating how carefully considering you problem in relation to the algorithm architecture can aid in success of applying machine learning.}
	\item \bibentry{NIMS} \textbf{Citation for Japan's National Institute of Materials Science.}
	
	\item \bibentry{Nie2014} \textbf{Finite element and stochastic analysis models used to study microstructure evolution of a nickel-based superalloy (Nb stabilized, so its probably 615 or 718); studies nucleation and growth of dendrites, Nb segregation, laves phase formation; evaluated relationship between temperature gradient (G) and solidification velocity (R) and the dendrite arm spacing, distribution of laves phase, tendency to hot crack. \begin{itemize}
		\item Parametric analysis -- temperature history (gradient, rates) $\to$ dendrite arm spacing
		\item FE model
		\item ICME
	\end{itemize}}
	
	\item \bibentry{Oganov2006} \textbf{An early application of genetic algorithms/evolutionary algorithms on ab initio DFT searches; starts with a database of ab initio total energy calculations, uses evolutionary algorithm to find stable crystal structure.
	\begin{itemize}
		\item Alloy Design
	\end{itemize}}
	
	\item \bibentry{Olivares-Amaya2011} \textbf{Application of multiple linear regression on a database of material properties for prediction of photovoltaic materials.}
	\item \bibentry{Oliynyk2016} \textbf{Paper that uses ensemble modeling for prediction of Heusler compounds.
	\begin{itemize}
		\item Alloy design
	\end{itemize}}
	\item \bibentry{Ouyang2017} \textbf{This paper presents sure independence screening and sparsifying operators (SISSO) which is a method of descriptor analysis; algorithm takes a set of starting descriptors and mathematical operations and creates new descriptors out of combinations of the original set; uniqueness of the approach is that it limits the descriptors which can be formed based on dimensional analysis i.e. the physical dimension of the new descriptor must make sense.}
	
	\item \bibentry{Pal2014} \textbf{They present a thermomechanical finite element model that is faster and more robust (they claim) than other approaches; their target is toward verifying in situ close loop process control and for predicting residual stress or distortion build up.\begin{itemize}
		\item FE thermomechanical model
		\item ICME
	\end{itemize}}
	
	\item \bibentry{Pilania2013} \textbf{Paper is focused on material property prediction based on stable crystal structure; uses fingerprint representation of chemo-structural characteristics and electronic charge density to screen for materials; discovers `selection rules' for given material properties based on characteristics of fingerprints.}
	\item \bibentry{Plotkowski2017} \textbf{Proposes an approach to predict transient heat conduction during e-beam PBF; looks at melt pool geometry and solid-liquid interface velocity; experimentally validated on IN718.}
	\item \bibentry{Poggio2004} \textbf{Very technical paper on the necessary conditions for certain classes of learners to be successful.}
	\item \bibentry{Poole2011} \textbf{Nate's favorite book from undergrad.}
	\item \bibentry{Quinlan1986} \textbf{A text on decision trees.}
	
	\item \bibentry{Raghavan2016} \textbf{Looking at transient thermal behavior as AM parts are heated and re-heated during layerwise manufacturing; looked at spatial and temporal evolution of temperature gradient (G) and solidification velocity (R); characterized thermal history based on electron beam parameters; analyzed conditions necessary for a columnar-to-equiaxed transition; validated using experimental results. \begin{itemize}
		\item FE thermomechanical model
		\item ICME
	\end{itemize}}
	
	\item \bibentry{Ramprasad2017} \textbf{Review article on machine learning in materials science; covers successful studies undertaken in the past decade; looks specifically at fingerprints or descriptors of material systems; describes learning approaches based on types of material descriptors and the level of the descriptor (material level, molecular level, atomic level).}
	
	\item \bibentry{Raplee2017} \textbf{Focused around process monitoring of electron beam powder bed fusion; calibrates temperature profiles from thermographic data; observes temperatures during transition between solid and liquid during manufacture. \begin{itemize}
		\item In situ monitoring
	\end{itemize}}
	
	\item \bibentry{Roweis2000} \textbf{A primer on a non linear dimensionality reduction technique. \begin{itemize}
		\item Dimensionality reduction
	\end{itemize}}
	
	\item \bibentry{Roy2012} \textbf{A paper on ab initio methods for finding half heusler compounds.
	\begin{itemize}
		\item Alloy design
	\end{itemize}}
	
	\item \bibentry{Sabbaghi2016} \textbf{Develops a Bayesian inference approach for predicting deformation that occurs during 3D printing.}
	\item \bibentry{Sahoo2016} \textbf{Phase field study of Ti-6Al-4V microstructure growth in electron beam powder AM. Looks at growth of dendrite arms as a function of temperature gradient; ignores crystallographic information, does not distinguish between nucleating phases during solidification; reports temperature gradients, which could be useful for in situ Ti-6Al-4V diffraction study; outlines some equations behind phase field modeling pretty well. \begin{itemize}
		\item Phase field dendrite growth
		\item ICME
	\end{itemize}}
	\item \bibentry{Sames2016} \textbf{Review of techniques for processing and metallurgy of AM; covers processing defects, heat transfer, solidification, solid-state precipitation, mechanical properties and post-processing metallurgy.{\color{red} A good review to read.}}
	\item \bibentry{Schutt2014} {\color{blue} This is a great article that should be highlighted up front in the final draft.} \textbf{This article applies KRR to predicting the density of electronic states above the Fermi energy using a novel crystal structure representation; they use radial basis functions defined in terms of the neighborhood distance of different atoms in the unit cell; they define it for a unit cell with two atoms, could easily be extended to $\mathbb{R}^3$ and higher; high prediction accuracy.}
	
	\item \bibentry{Scime2018} \textbf{Study using a CNN to autonomously detect and classify anomalies in a LPBF process; it looks like it was fairly effective; optical light camera mounted overhead of powder bed took images during printing; CNN was able to identify anomalies such as recoater hopping, recoater streaking, debric, super-elevation, part damage, and incomplete spreading at accuracies over 70\% for all using a Multi-scale CNN (MsCNN). \begin{itemize}
		\item In situ monitoring
		\item Computer vision
	\end{itemize}}
	
	\item \bibentry{Serra2007} \textbf{Applies support vector machines (SVM) to predict zeolite synthesis based on starting gel compositions and weight ratios; compared SVM to NN and classification trees.
	\begin{itemize}
		\item Alloy design
	\end{itemize}}
	
	\item \bibentry{Setyawan2010} \textbf{This presents a study on bottlenecks in high throughput band structure calculation from DFT; looks like its a case study in using AFLOW.}
	\item \bibentry{Setyawan2011} \textbf{Demonstration of AFLOW on predicting the band structure of 7439 materials to search for a good scintillator for $\gamma$ ray radiation detection.}
	\item \bibentry{Shan2010} \textbf{This is a survey of papers focused on: strategies for tackling the high dimensionality of problems, model approximation techniques, and direct optimization strategies for computationally-expensive black-box functions and promising ideas behind non-gradient optimization algorithms.} {\color{red} Could be a good resource to point people toward when discussing more theoretical considerations.}
	\item \bibentry{Shannon1948} \textbf{Seminal paper in information theory.}
	\item \bibentry{Snyder2012} \textbf{Application of machine learning to predict electron density functions a la Kohn-Sham DFT; uses PCA on a dataset of functionals to predict other functionals.}
	\item \bibentry{Strano2013} \textbf{Surface roughness analysis, modeling, and prediction for 316L SS produced via SLM; develops a model for predicting surface roughness which takes into account unmelted particles on the surface; model is based on geometric setup of the build. \begin{itemize}
		\item Parametric analysis -- geometry/orientation $\to$ surface roughness
	\end{itemize}}
	
	\item \bibentry{Stucke2003} \textbf{Applies a genetic search algorithm to optimize packing density of crystal nanoparticles.}
	\item \bibentry{Sumpter1996} \textbf{A review of convolutional neural networks and their applications in materials science through 1996. Probably the earliest work on machine learning in materials science I've found yet.}
	\item \bibentry{Szost2016} \textbf{Comparative investigation of microstructure and residual stress in Ti-6Al-4V manufactured with CLAD and WAAM.}
	\item\ \bibentry{Tan2011} \textbf{Model integrating cellular automata and phase field methods to analyze dendrite growth of multiphase alloys; a pretty in-depth model; has a section at the end applying the model to welding. \begin{itemize}
		\item Combined Phase Field and Cellular Automata model
		\item ICME
	\end{itemize}}
	
	\item \bibentry{Tapia2014} \textbf{A review on process monitoring and process control in AM through 2014. \begin{itemize}
		\item In situ monitoring
		\item Review
	\end{itemize}}

	\item \bibentry{Tapia2017} \textbf{Application of a surrogate model to a dataset of both computational and experimental data; uses kriging/Gaussian Process Modeling to create a response surface of melt pool depth versus laser parameters. \begin{itemize}
		\item Surrogate model
	\end{itemize}}
	
	\item \bibentry{Tenenbaum2000} \textbf{Proposes a novel technique for dimensionality reduction. \begin{itemize}
		\item Dimensionality reduction
	\end{itemize}}
	
	\item \bibentry{Toyserkani2004} \textbf{Finite element model of the laser cladding process (LENS-like); model can predict clad geometry as a function of time and processing parameters (laser pulse shaping, travel velocity, laser pulse energy, powder jet geometry, material properties); has applications to the shape of deposited material in SLM or LENS. \begin{itemize}
		\item FE model
		\item ICME 
	\end{itemize}}
	
	\item \bibentry{Trapp2017} \textbf{Measurements of the absorptivity of powders during an SLM process.
	\begin{itemize}
		\item Alloy design
	\end{itemize}}
	
	\item \bibentry{Turner2014} \textbf{Its a review of fused deposition modeling and wire-feed/extrusion based additive manufacturing processes; focuses on how the processes operate; not specific to metal.}
	\item \bibentry{VandeWalle2008} \textbf{Model development of tensorial cluster expansion for encoding structure-property relationships; very technical.}
	\item \bibentry{Vellido2012} \textbf{A review-ish paper discussing and highlighting the need for interpretability in machine learning algorithms.}
	
	\item \bibentry{Villars1998} \textbf{A citation for the Linus Pauling files.
	\begin{itemize}
		\item Alloy design
		\item Database mining
	\end{itemize}}
	
	\item \bibentry{Wagner2016} \textbf{Discussion of how theory should guide choice of machine learning algorithms; case studies of common pitfalls in applying machine learning that can occur in materials science; discusses problems such as overfitting data, descriptor selection, preserving physical interpretability.}{\color{red} Great article to highlight in theory section.}
	\item \bibentry{Wang2017} \textbf{Describes a control feedback system for shape deviation during fused deposition modeling (plastic 3D printing). \begin{itemize}
		\item In situ monitoring
	\end{itemize}}
	
	\item \bibentry{Wang2017a} \textbf{Use of a high-throughput study to find processing conditions for AM stainless steel that produces a desirable microstructure; obtains the highest ever strength-ductility tradeoff published for 316L SS. \begin{itemize}
		\item Parametric analysis -- processing parameters (all) $\to$ microstructure \& mechanical properties
	\end{itemize}}

	
	\item \bibentry{White2000} \textbf{A review of high throughput combinatorial methods in pharmacology.}
	\item \bibentry{Wilkinson2016} \textbf{An article about the FAIR guiding principles for scientific data.}
	\item \bibentry{WohlersReport2017} \textbf{An annual report about the industry of additive manfuacturing.}
	
	\item \bibentry{Wolf2000} \textbf{An early paper using evolutionary algorithms for design of materials.
	\begin{itemize}
		\item Alloy design
	\end{itemize}}
	
	\item \bibentry{Wong2012} \textbf{A mid-level review of processes in additive manufacturing.}
	\item \bibentry{Woodley1999} \textbf{Early paper using genetic algorithms for crystal structure prediction using only composition.}
	\item \bibentry{Wu2014} \textbf{Parametric study of process parameters' (laser scanning pattern, laser power, laser speed, and build direction) impact on residual stress in LPBF; uses neutron diffraction to characterize residual stress. \begin{itemize}
		\item Parametric analysis -- processing parameters (above) $\to$ residual stress 
	\end{itemize}}
	
	\item \bibentry{Wu2018} \textbf{Applies combinatorial approaches to finding optimal composition of an alloy for desired properties.
	\begin{itemize}
		\item Alloy design
	\end{itemize}}
	
	\item \bibentry{Xiang1995} \textbf{This is the OG article on high throughput investigations into materials design.}
	
	\item \bibentry{Xu2008} \textbf{Applies a genetic algorithm to the design of alloys.
	\begin{itemize}
		\item Alloy design
	\end{itemize}}
	
	\item \bibentry{Xu2015} \textbf{Carefully controls the laser parameters to induce martensite decomposition in SLM Ti-6Al4-V. \begin{itemize}
		\item Parametric optimization -- laser parameters ({\color{red} look up}) $\to$ microstructure 
	\end{itemize}}
	
	\item \bibentry{Yan2015} \textbf{Finds descriptors for accurately predicting thermoelectric coeffecient zT from first principles.}
	\item \bibentry{Yan2018} \textbf{Long paper discussing the use of self consistent cluster analysis (SCA) among other methods for designing microstructures and process-structure-property relationships in AM.} {\color{red} Must give this article an in-depth read.}
	\item \bibentry{Yap2015} \textbf{A review of SLM across material types.}
	
	\item \bibentry{Yuan2018} \textbf{Uses a convolutional neural network to measure the average and standard deviation of track widths in single track prints of LPBF. \begin{itemize}
		\item In situ monitoring
		\item Computer vision
	\end{itemize}}
	
	\item \bibentry{Zegard2016} \textbf{Proposes tweaks to current topology optimization algorithms so that they are compatible with AM.
	\begin{itemize}
		\item Topology optimization
	\end{itemize}}
	\item \bibentry{Zhang2010} \textbf{A finite element study on temperature history of W-Ni-Fe powders; characterizes thermal history based on layer thickness, scan speed, etc.; compares with experiments.}
	\item \bibentry{Zhao2017} \textbf{They use HEXRD to characterize phase development in powder bed Ti-6Al-4V.}
	
	\item \bibentry{Zhou2009} \textbf{Simulation of power particle packing with different particle size distributions; applications for metrology, rheology of AM.
	\begin{itemize}
		\item Featurization
	\end{itemize}}
	
	
	\item \bibentry{Ziolkowski2014} \textbf{Uses XCT to characterize porosity in SLM parts.}
	
	
	
\end{itemize}
\bibliographystyle{annotate}
\bibliography{../APR.bib}
\end{document}