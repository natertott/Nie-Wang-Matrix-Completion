\subsubsection{Featurization of Qualitative Image Data}
The use of images in studying additive manufacturing is widespread, common throughout all aspects of the manufacturing process, and provides key information about material properties and processing. As with all aspects of AM, the sheer size of image data to be analyzed is profound due to the large design space of AM. The types of information taken from images includes grain characteristics, like size, orientation, and phase, and defect characteristics, like pore size or crack length. When characterizing all of these features for all possible processing conditions and alloys the size of the problem grows quickly.

Computer vision algorithms have been tested for automation of materials science image classification and analysis. Using these algorithms can speed up the experimental characterization process of AM. Furthermore, computer vision techniques can quantify information which may have otherwise only been used qualitatively or measured by approximation. 

It is worthwhile to mention up front that these algorithms have been \textit{tested} on microstructure and, in some cases, additive-specific images. There are few algorithms that can process AM microstructure data `out-of-the-box.' Rather, these algorithms will need to be tailored in order to quantify AM images specifically. However, the algorithms discussed here have been proven on non-AM microstructure datasets, thus they should be extensible to AM datasets. The computer vision approaches which work for microstructure data are often the same approaches discussed in the previous section for in situ monitoring. 

One AM-related application of image characterization is measuring particle size distributions in AM powder feedstock. DeCost and Holm used SIFT with a dictionary classifier, as in template matching, to measure the particle size distribution for a dataset of synthetic powder particles \cite{DeCost2017a}. Particle size distribution plays in several steps across the additive process including energy absorption and part metrology \cite{Zhou2009, Boley2015, Boley2016}. DeCost created datasets with six different particle size distributions. Image features were identified and classified using $k$-means clustering on the features found by SIFT. Then, a classification algorithm known as a support vector machine (SVM) was trained to classify image features into particle sizes. DeCost was able to achieve $89$\% overall classification accuracy in measuring particle size distribution this way. DeCost et al. later improved upon this powder classification method and were able to achieve higher classification accuracies for real powder images \cite{DeCost2017}.

Strides have been made in automatically identifying and quantifying information from metallographs \cite{DeCost2015, DeCost2017b, Ling2017a, Bulgarevich2018}. A good portion of quality control in materials science as a whole, not just AM, involves classifying materials based on metallographs or micrographs of microstructure. Work is being done across materials science to apply machine learning based computer vision to classifying and quantifying information in these microstructural images. Doing so will speed up the process of materials characterization and qualification, while also providing methods of quantifying information which otherwise would have stayed in a qualitative form. Examples include classification of grain stuctures, measurements of grain size, pore size calculations, and more.

An additive-specific image segmentation algorithm was used by Miyazaki et al. \cite{Miyazaki2019}. Five image filters were convolved with microstructure images of selective laser melted Ti-6Al-4V. The features identified by these filters were used in a random forest algorithm to segment the image into regions of $\alpha$ phase grains and $\beta$ phase grains. The algorithm was able to automatically calculate area fraction of primary and secondary $\alpha$ phases that form during cooling. It was also able to calculate the nearest-neighbor distance between grains. Nearest neighbor distance of grains is indicative of grain characteristics like size, morphology, and distribution. 

Chowdhury et al. took a more expansive approach to performing feature identification in microstructures. In particular, they were looking to classify microstructures as either dendritic or non dendritic. Chowdhury employed 8 different feature identification methods for a dataset of images. Classification was performed using an ensemble of ML techniques including support vector machines (SVM), Na\"ive Bayes, nearest neighbor, and a committee of the three previous classification methods \cite{Chowdhury2016}. Chowdhury's wide approach to image classification achieved classification accuracies above 90\%. 

It would be overly burdensome to lay out every \textit{possible} application of computer vision in additive manufacturing. Efforts are underway across materials science to implement computer vision for the automation of materials classification. Rather, the authors would like to refer the reader to reviews on the subject of computer vision for materials science, as well as open libraries listed in Table \ref{data_tools}. The hope is that readers will discover the many possible uses of computer vision and begin applying methods to their own AM problems. 

