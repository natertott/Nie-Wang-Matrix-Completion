\subsubsection{In Situ Automated Feature Detection}
Thus far, in situ control in AM has been consistently ranked as one of the most-needed technologies for advancing the technology \cite{Mani2017, Tapia2014}. If the ideal conditions under which to perform manufacturing can be understood, monitoring the process will ensure the desired conditions are achieved. Or, the conditions under which certain defects form could be ascertained. In this case scientists would want to be able to detect the onset of those conditions to adjust the heat source parameters and fix the problem. All of this will require in situ sensing and a level of robotic control. 

A study by Gobert et al. investigated using X-ray CT data to correlate with images obtained by a DSLR camera during the printing process. Gobert et al. printed a single part, containing several different features, and recorded the print results before and after every layer using a DSLR and eight different lighting conditions. The part was then characterized using X-ray CT to find the location and size of pores and inclusions. A matrix list of pixel intensities $I(x,y,z)$ was convolved using a Gaussian blur, whereby each CT image slice was iteratively blurred in order to increase the contrast of differently sized features. This same type of blurring is common in many image recognition algorithms for images with multiple features of varying size \cite{Lowe2004, Bay2008}. The result of the Gaussian blur identifies pixel intensities as high-intensity pixels (inclusions) or low-intensity pixels (voids), with the remaining identified as part of the bulk. Clustering with $k$-means was then applied to group pixels into single pores.

The information from the convolution operation was then used to train an algorithm that detects defect formation from the DSLR results. One of the underlying assumptions of the study was that, even though the DSLR images did not record during the melting and solidification process, features would appear in each image that could be correlated back to the higher-resolution data in the CT images. The DSLR images were projected into the feature space of the convolved CT data using an affine transformation through linear least squares. This transformation provides a mapping by which \textit{new} image data can be mapped to the feature space, giving an estimate of whether or not the new image contains a defect or not. The study by Gobert et al. was able to accurately predict whether or not an image contained a defect with 85\% accuracy after all the DSLR lighting conditions were combined into an ensemble classifier. Similar studies have also achieved good prediction accuracy for AM using convolutional neural networks \cite{Scime2018} and dictionary methods \cite{Abdelrahman2017}. Kernel ridge regression was implemented successfully for fault detection in general manufacturing systems \cite{Wang2018}.

With this type of in situ image recognition, defect formation can be detected during the printing process, precluding the need for certain post-characterization techniques. If defects form and are detected, then the printing process can be stopped, saving powder/feedstock, operating costs, and operating time. However, an even more desirable result is to first detect defect formation, then correct for it in situ. To do so, information will have to be assembled that relates the detection of defects, the manner in which the defects formed, and the prescribed fix to undo or stop the defects from growing. It is likely that combining all these information sources will require a significant leap in scientists' ability to apply machine learning algorithms to widely varying data sets and types. 