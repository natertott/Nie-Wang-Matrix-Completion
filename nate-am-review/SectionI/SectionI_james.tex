\section{Motivation}

Metals-based additive manufacturing (AM) represents a potential paradigm shift in how products are manufactured, providing versatility in the type and design of parts produced by a single manufacturing facility, decentralizing manufacturing capability, and enabling novelty in the properties and design of the parts, to name but a few benefits AM offers. However, there have been significant roadblocks to fully realize AM's potential, particularly in the control of consistency and quality in part production and in the development of materials amenable to the AM process. Although decades of immense scientific and engineering work, in industry, academia, and government, have produced large advances in making AM a practical manufacturing solution, the metallurgical challenges facing AM persist. Computational materials simulation and the Integrated Computational Materials Engineering (ICME) approach have made strides in accelerating materials development, but the features that make AM such a departure from traditional manufacturing requires more uniquely suited problem-solving methods. In this paper, we argue that machine learning-based materials informatics is such a solution, capable of significantly accelerating the AM development process.

Made famous by his widespread search for a lightbulb filament material in 1880's, Thomas Edison and his name-sake ``Edisonian approach`` represented the most efficient method of materials development in the industrial design process at the time. Edison was searching for materials with a combination of properties unique to his specific engineering problem, such as melting temperature, ductility, and electrical and thermal conductance. The development of materials for specific technologies was a long-standing problem. Long before Edison, for instance, the search for sword alloys was a centuries-long Edisonian approach to optimizing composition and thermo-mechanical processing of steel. 
  
Catalogs of properties of engineered materials produced by typical manufacturing processes were compiled in the 20th century, distilling the materials problem in industrial design to one of selection. If Edison had such a catalog, he would have identified the proper filament material much more quickly.  However, selection from \emph{existing} materials fundamentally limits the capabilities of the manufactured part.  A turbine blade, for instance, will have an operating temperature limited by the melting point and high-temperature strength of the alloys in the catalog. 

To address this problem, the 20th century also saw the maturation of materials science and engineering as a field of study, enabling more rapid development of novel materials and materials manufacturing methods for specific applications. The process-structure-properties-performance paradigm transformed the combinatorial trial-and-error and intuition-driven materials discovery process into a problem of engineering a desired microstructure through designed processing. For instance, the history of turbine blade superalloy development is typified by advancements in control over microstructure through processing, including increasingly complex alloying recipes, multi-step heat treatments, and single-crystal casting. 

In the past decades, materials development has greatly accelerated to match the broader acceleration of general technology advancement. Computational materials science has enabled the prediction of microstructure from processing and of properties from microstructure, reducing the need for costly and time consuming experimentation. The Integrated Computational Materials Engineering (ICME) approach tightly integrates physics-based computational models into the industrial design process, allowing the desired performance requirements of a part to guide the design of a novel material. Examples include low-RE Ni superalloys for higher turbine performance and lower cost and the Ferrium S53 alloy designed for corrosion-resistant landing gears. Both cases took materials development timelines from decades to under ten years, demonstrating the practical capability of designing new materials within an industrial product timeframe. Generalizing this capability to more industries and further accelerating the process is the primary goal of the Materials Genome Initiative (MGI). 

Current AM capabilities are limited by materials-based problems which are uniquely difficult to solve with the above paradigm. The AM process itself is complex relative to traditional casting methods, including rapid solidification, vaporization and ingestion of volatile elements, and a complex thermal history, all of which vary with part location and require advanced computational tools to properly predict microstructure and properties. However, the lower cost and time barrier to entry for performing AM has enabled the rapid accumulation of experimental data, enabling the Edisonian approach for finding optimized AM processing methods and parameters of existing alloys. For AM, the ICME-based tools have been catching up, attempting to bootstrap legacy models to this data, with limited success. As such, current AM materials development is largely combinatorial, as the processing of legacy alloys are optimized with extensive design of experiments and AM-specific alloys with higher performance are just now being effectively developed.

It is within this context we argue that machine learning (ML) can accelerate the application of additive manufacturing. ML as a method for model development has shown wide application in the past years, in industries ranging from finance to social networking. The use of ML in materials has been relatively limited for a variety of reasons, primarily the lack of a robust and large dataset on which ML can operate. The MGI identified this as a primary problem for accelerating materials development, and there has been significant progress recently in infrastructure development for materials databases suitable for informatics tools such as ML. The difficulty in producing physics-based ICME models and the ability to rapidly and efficiently sample processing space in a combinatorial approach makes additive manufacturing an attractive use case for ML. Machine learning as a framework can couple the legacy ICME tools with the experimental data to produce much more accurate AM process-structure-property models and to automate the iteration of designed experiments for model improvement and optimized materials.

In this paper, we present our arguments for the use of ML to resolve AM challenges. We begin by detailing how ML could be applied to AM and the use cases we envision. Then we discuss existing AM models and how they could be integrated into a ML framework. Existing examples of ML as applied to materials and AM specifically are then reviewed. We conclude with a prospective outlook on the potential of ML for AM.

%I.	Motivation: Why would someone want to use ML approaches to solve AM problems?
%A.	The Edisonian Approach: Brute Force Search, Historical MatSci Approach: Trial and error requires luck and a lot of resources, takes a lot of time and money
%	Modern example of Edisonian approach: how long/cost to develop 304 welding wire or an Inconel? 
%B.	More Modern Advancements in Mat Sci Developments: Decades of effort have been made to reduce the time and cost to discover and deploy new materials/processes
%	Use specific development timeline examples
%		Combinatorial Example
%		ICME example Ð maybe Questek steel (Ferrium)? Or something recent? Using ICME and gains made there.
%C.	Machine Learning Provides Means to wrap all of these things together (combinatorial, ICME, traditional experimentation, etc. can be used in tandem):
%Automated Iteration with statistically measurable differences quantified
%	Great for AM b/c of high DOF space Ð feature identification challenging w/o statistics, etc.
%D.	Roadmap: Overview the flow of the rest of the paper.
