\documentclass[letterpaper,12pt]{article}


%%%%%%%%%%%%%%%%%%%%%%%%%%%%%%
%%%%%% LATEX PACKAGES %%%%%%%%%%%%%
% csm-thesis automatically includes the following packages:
%% float
%% setspace
%% geometry
%% graphics
%% textcase
%% subfig

% Note: Two package options exist for your convenience: ``insane'' and ``nolabel''.  To use these options together separate them by a comma, ie. \usepackage[insane,nolabel]{csm-thesis}
% * \usepackage[insane]{csm-thesis}
% Turn off all document sanity checks.  This option can be used to render a ``sub-document'' that is part of the root thesis document.  It is important to note that you should NEVER disable this check on your root thesis document, as important format errors and warnings will be disabled.
% * \usepackage[nolabel]{csm-thesis}
% Disables automatic reference ``labeling'' of figures and tables.  By default the thesis template prepends any reference to a figure or table with ``Figure~'' or ``Table~''.  This option is meant for disabling the labeling behavior when a document already has the appropriate labeling.  It is important to note that if your document DOES NOT have the appropriate labeling (the reference label must EXACTLY MATCH the caption label) then it will not pass the format review.
\usepackage[insane]{csm-thesis}
% For inserting large multi-page tables:
\usepackage{array}
\usepackage{longtable}
% For inserting sideways tables and figures
\usepackage{rotating}

% Packages used by the review article for Additive Manufacturing
\usepackage{amscd}
\usepackage{amsmath}
\usepackage{amssymb}
\usepackage{amsthm}
\usepackage{dcolumn}
\usepackage{booktabs}
\usepackage{dashrule}
\usepackage{bm}
\usepackage[utf8]{inputenc}
\usepackage[T1]{fontenc}
\usepackage{float}
%\usepackage{floatrow}
\usepackage{graphicx}
\usepackage{import}
\usepackage{mathtools}
%\usepackage{natbib}
\floatstyle{plaintop}
\usepackage{subfiles}
%\restylefloat{table}
\usepackage{tabularx}
\usepackage{verbatim}
\usepackage{xcolor}
%\usepackage{subcaption}
%\usepackage[bottom]{footmisc}

% Since hyperref and cite don't completely get along, the template now recommends using natbib:
% For an explanation, see http://www.tex.ac.uk/cgi-bin/texfaq2html?label=citesort
\usepackage[numbers]{natbib}
% If you wish to use ``cite'' instead then your choices are:
% 1) Don't use the hyperref package
% 2) Put ``cite'' before hyperref, resulting in no citation hyperlinks
% 3) Put ``cite'' after hyperref, resulting in ugly looking citations

% If you choose not to use natbib then you can set the standard ``numeric'' style like so:
%\bibliographystyle{unsrt}

% Important Note: math-mode in sections, titles, and other bookmarks will generate warnings with hyperref.  You can work around this by either:
% 1) Not using the hyperref package
% 2) Using \texorpdfstring{TeX Code}{PDF Replacement} to display an alternative bookmark (for example, \texorpdfstring{H$_2$O}{Water}).
% The thesis template will automatically import your document information into hyperref, so if you go to ``File | Properties'' in Adobe Acrobat it will display the title and author.  If you would like to over-ride this option then just change the line below to ``\usepackage[]{hyperref}''.
\usepackage{hyperref}
% For inserting programming code:
\usepackage{listings}
% For inserting landscape-mode objects:
\usepackage{pdflscape} % use ``lscape'' if you are not creating a PDF output
% For matrices:
\usepackage{amsmath}
% For differently colored fonts
\usepackage{xcolor}

%%%%%%%%%%%%%%%%%%%%%%%%%%%%%%%%%%%%%%%%%
%%%%%%%%%%%%%%%%%%%%%%%%%%%%%%%%%%%%%%%%%
%%%%%%%%%%%%%%%%%%%%%%%%%%%%%%%%%%%%%%%


%%%%%%%%%%%%%%% Define Command %%%%%%%%%%%%%%%%%

\DeclareMathOperator*{\argmin}{argmin}
\DeclareMathOperator*{\argmax}{argmax}
\newcommand{\eqn}{\begin{equation}}
\newcommand{\equ}{\end{equation}}

\def\dashedrule#1#2#3{{
	%% #1 is length of dash
	% #2 is length of gap between dashes
	% #3 is number of dashes
	\dimen1=#2 \divide\dimen1 by 2
	
	\def\@ruledash{%
		\rule{\dimen1}{0pt}%
		\rule[0.5ex]{#1}{0.4pt}%
		% line is 0.5ex above the baseline
		% and 0.4pt thick
		\rule{\dimen1}{0pt}}%
		
	\count1=0\loop%
	\ifnum\count1<#3%
		\advance\count1 by 1%
		\@ruledash%
	\repeat}}

%%%%%%%%%%%%%%%%%%%%%%%%%%%%%%%%%%%%%%%%%
%%%%%%%% DOCUMENT START %%%%%%%%%%%%%%%%%%%%%%

% Do not put any carriage returns (\\), all normal LaTeX should function properly.
	\title{{\large \textbf{Ph.D Thesis Working Title:}} In Situ Diffraction Characterization of Phase Transformations and Mechanical Behaviors in Additively Manufactured Titanium Alloys}


\degreetitle{Doctor of Philosophy}
\discipline{Materials Science and Engineering}
\department{Mechanical Engineering}

\author{Nathan S. Johnson}
\advisor{Dr. Craig A. Brice}
\coadvisor{Dr. Aaron P. Stebner}
\dpthead{Dr. John Berger}{Professor and Department Head}

\begin{document}

% Parts of a Thesis
%% Parts of a Thesis - Front Matter
\frontmatter

%%% Parts of a Thesis - Front Matter - Title Page (required)
\maketitle
\newpage

%%% Parts of a Thesis - Front Matter - Copyright Page (optional)
% If the copyright for your document spans multiple years, or does not match the current year, then replace ''\the\year`` below with the appropriate text.
\makecopyright{\the\year}
\newpage

%%% Parts of a Thesis - Front Matter - Signature Page (required)
\makesubmittal
\newpage

%%% Parts of a Thesis - Front Matter - Abstract (required)
\begin{abstract}

This thesis focused on the characterization of additively manufactured titanium alloys using high energy X-ray diffraction.

\end{abstract}
\newpage

%%% Parts of a Thesis - Front Matter - Table of Contents (required)
\tableofcontents
\newpage

%%% Parts of a Thesis - Front Matter - List of Figures (if applicable)
%%% Parts of a Thesis - Front Matter - List of Tables (if applicable)
% NOTE: If you have more than 2 items in either list they must be separate.
% This case is generally handled automatically, but if you are told to separate the lists then comment or remove the two lines below:
\listoffiguresandtables
\newpage

% ... and then uncomment these four lines to force separate lists:
%\listoffigures
%\newpage
%\listoftables
%\newpage


%% NOTE: If included in the front matter, a glossary, a list of abbreviations, or a list of symbols is placed as the last list. If these lists are included in the back matter, they are placed immediately before the REFERENCES CITED.

%%% Parts of a Thesis - Front Matter - Glossary (if applicable)
%\glossary

%%% Parts of a Thesis - Front Matter - List of Symbols (if applicable)

% Place this call before ''\listofsymbols`` to make the symbols appear on the left instead of the right:
%\ShowSymbolFirst
% To call the “List of Symbols” “Nomenclature” instead use:
%\listofsymbols[Nomenclature]
% To autosort the list use a star after the command (ie. \listofsymbols*[Nomenclature] or \listofsymbols*)
\listofsymbols
% With very large symbol lists it is sometimes good to split the list into multiple sub-lists.  To output the lists just use the extended \listofsymbols command (below) and to add an element to the list use the optional parameter to ''\addsymbol``.
%\listofsymbols{General Nomenclature}
%\listofsymbols{Greek Letters}
\newpage

% Note that you may define symbols anywhere in the document, when you re-run LaTeX they
% will be added to the list (just like all other lists)
\addsymbol{alpha}{$\alpha$}

% Example for sub-list symbols (optional parameter specifies which list to use):
%\addsymbol[General Nomenclature]{absorption coefficient}{$\alpha_c$}
%\addsymbol[General Nomenclature]{absorption cross section}{$\alpha_{\sigma}$}
%\addsymbol[Greek Letters]{average radius of cylindrical shell}{$c$}
%\addsymbol[Greek Letters]{activation energy of oxidation reaction of a-C in excited state}{$E^{\ast}_{act}$}

%%% Parts of a Thesis - Front Matter - List of Abbreviations (if applicable)
% To autosort the list use a star after the command (ie. \listofabbreviations*)
\listofabbreviations
\newpage

% Note that you may define abbreviations anywhere in the document, when you re-run LaTeX they
% will be added to the list (just like all other lists)
\addabbreviation{high energy X-ray diffraction}{HEXRD}


%%% Parts of a Thesis - Front Matter - Acknowledgments (optional)
\begin{acknowledgments}

Mom \& Dad \& Noelle\\
Thomas \& Marcella \\
Matt Zappulla \& Jasmin Honegger \\
Maria Strantza \& Bj\"orn Clausen \& Sven Vogel \& Travis Carver \\
Brian Toby \& Robert Von Dreele \\

\end{acknowledgments}
\newpage

%%% Parts of a Thesis - Front Matter - Dedication (optional)
\begin{dedication}
To be determined.
\end{dedication}
\newpage

%~~~~~~~~~~~~~~~~~~~~~~~~~~~~~~~~~~~~~~~~~~~
%~~~~~~~~~~~~~~~~~~Main Text Body~~~~~~~~~~~~~
%% Parts of a Thesis - Body
\bodymatter

% Outline
\chapter{Outline}
This is an outline that I made with Dr. Stebner on February 13th of 2020. It is a rough outline of the chapters of my Ph.D thesis, as well as the potential publications that might result from each chapter.

\begin{enumerate}
	\item Intro 
		\begin{enumerate}
		\item Motivation
		\item Background 
			\begin{enumerate}
			\item Diffraction Techniques
			\item Materials science of metals AM
			\end{enumerate}
		\end{enumerate}
	\item Vision for Machine Learning Applications in Additive Manufacturing
	\item Austenititic Stainless Steel Weld Beads
	\item Ti-6Al-4V Weld Beads 
	\item Mechanical Behavior and Phase Transformations of Additively Manufacture NiTi Elastocaloric Cooling Materials 
	\item Single Unit Cell Lattices
	\item Multiple Unit Cell Lattices
	\item Summary and Future Work
	\item Appendices
		\begin{enumerate}
		\item Matrix Completion
		\item Lockheed Martin Ti-6Al-4V Antler Build Characterization
		\item Scripting with GSASII Scriptable
		\end{enumerate}
\end{enumerate} 
\chapter{Introduction}
\textbf{Continuity Statement:}
There are several outstanding problems in the characterization of rapidly solidified alloys -- particularly additively manufactured alloys -- that require modern characterization techniques to address. These problems include
\begin{itemize}
	\item The determination of stress and strain during solidification
	\item Measurements of temperature during solidification
	\item Quantification of residual stress in samples
	\item Resolving the dynamics of phase transformations during solidification
	\item Quantification of phases, both expected and unexpected, after solidification
	\item Observation of diffusional processes during solidification
	\item The mechanical behavior of unique geometries
	\item The impact of the printing process on mechanical properties
\end{itemize}
High energy X-ray diffraction stands as an important tool to address these problems.

\subsection{Motivation}

\subsection{Background}
%~~~~~~~~~~~~~~~~~~~~~~~~~~~~~~~~~~~~~~~~~~~~~~~~~~~~~~
\subsection{High Energy X-ray Diffraction Characterization Techniques}
Information to cover for the methods section:
\begin{itemize}
	\item Basics of X-ray diffraction \begin{itemize}
		\item Bragg's Law
		\item The extension of Bragg's law to 3DXRD
		\item Sources of intensity in HEXRD \begin{itemize}
			\item The polarization factor
			\item The Lorentz Factor
			\item Thomson Scattering
			\item Compton Scattering
			\item Texture
			\item Debye-Waller Factor
			\item Structure factor
			\item Multiplicity factor
			\item Absorption factor \end{itemize}
		\end{itemize}
	\item Texture in additively manufactured materials
	\item Crystallite size effect
	\item Residual stress
	\item Solidification dynamics \begin{itemize}
		\item Scattering through amorphous materials
		\item Observation of phase changes during solidification
		\item X-rays as a probe to measure temperature, cooling rate
		\item Quantification of liquid phase fraction \end{itemize}
	\item Measurement of stress and strain \begin{itemize}
		\item Noyan's model
		\item Limitations of Noyan's model 
		\item DFA model and its relevance to HEXRD measurements
		\end{itemize}
\end{itemize}

\subsection{Detector Setups and Definition of Laboratory, Detector, and Sample Coordinate Frames}\label{detector_setup}
\begin{figure}
	\includegraphics[width=1\linewidth]{/Users/njohnson/git/thesis/document/chapters/introduction/introduction_images/coordinate_systems}
	\caption{The general setup of a synchrotron X-ray experimental hall. Coordinate systems for the laboratory, sample, and detector are shown.}
	\label{coordinate_systems}
\end{figure}
The general setup of a laboratory experimental hutch at a synchrotron X-ray facility can be seen in \ref{coordinate_systems}. This setup applies to synchrotrons at the Advanced Photon Source (APS) and the Cornell High Energy Synchrotron Source (CHESS), the facilities where data for this thesis was collected.


\subsection{Bragg's Law in 2D and 3D Crystals}
\textbf{Write up, with figures, the derivation of Bragg's Law in a crystal.}


\subsection{X-ray Intensities}
All the sources of X-ray intensity from a diffracted crystal.

\paragraph{Absorption of X-rays by a Material}
Intensity is the flux of energy through a unit surface normal of the average ray of the beam per second. It is typically measured in cgs units. Define $\mu$ to be the coefficient of X-ray absorption. Consider a beam with intensity $I$ incident perpendicular to a sheet of material with mass-to-area ratio $d\rho$. Then the transmitted intensity is 
\eqn
	I + dI,\,dI<0
\equ
for all $dI\sim I,d\rho$. Let $dI = -\mu I d\rho$. Then
\eqn
\frac{dI}{I} = -\mu d\rho.
\equ
Integrate both sides to find
\eqn
	\frac{I}{I_0} = \exp{[-\mu\rho]}.
\equ
Define density to be $p = \rho/x$ where $x$ is the thickness of the plate. Then
\eqn
	\frac{I}{I_0} = \exp{[-\mu px]}.
	\label{intensity}
\equ

\paragraph{The Polarization Factor}
The following was originally derived by JJ Thomson. A re-treatment is discussed here based largely on the derivation presented in Cullity \cite{Cullity}. Scattering by a single electron occurs in part due to the randomly oscillating electromagnetic field of the electron. The electromagnetic field has two components: an electric component $\mathbf{E_y}$ and a magnetic component $\mathbf{E_z}$ that are perpendicular to each other. The intensity of the full electromagnetic wave is given by the sum of the squared amplitude of each wave component
\begin{equation}
	\mathbf{E}^2 = \mathbf{E_y} ^2 + \mathbf{E_z}^2
	\label{electromagneticintensity}
\end{equation}
where $\mathbf{E}$ is the squared amplitude of the total wave vector, or the intensity. The spatial relationship of $\mathbf{E_y}$ and $\mathbf{E_z}$ can be seen in Figure \ref{polarization_factor}.

\begin{figure}
	\centering
	\includegraphics[width=0.5\linewidth]{/Users/njohnson/git/thesis/document/chapters/introduction/introduction_images/polarization_factor}
	\caption{The oscillating electric field of an electron at point $O$.}
	\label{polarization_factor}
\end{figure}

Consider an electron at location $O$ in \ref{polarization_factor} with electric and magnetic wave components oscillating in the $yz$ plane, perpendicular to each other on the $y$ and $z$ axis, respectively. For an incident X-ray traveling along $xO$ consider the reflected X-ray intensity at location $P$ in the $xz$ plane. Thomson derived that the scattered intensity $I$ of the X-ray at a distance $r$ from an electron with mass $m$ and charge $e$ would be
\begin{equation}
	I = I_O \left(\frac{\mu_O}{4\pi}\right) \left(\frac{e^4}{m^2r^2}\right)\sin^2\alpha
	\label{reflected_intensity}
\end{equation}
where $I_O$ is the intensity of the incident beam, $\mu_O$ is the permittivity of free space, and $\alpha$ is the direction of the acceleration of the electron. \ref{reflected_intensity} can be simplified by condensing the constant terms into $K$, given by
\begin{equation}
	I = I_O \frac{K}{r^2}\sin^2\alpha.
	\label{condensed_reflected_intensity}
\end{equation}

In general, the amplitude of the electric and magnetic components of $\mathbf{E}$ will be equal since the direction of their oscillation is random. Therefore,
\begin{equation}
	E_y^2 = E_z^2 = \frac{1}{2}E^2
\end{equation}
and because the intensity of the reflected X-ray is dependent upon the magnitude of the oscillation of the electron's wavefunction 
\begin{equation}
	I_{Oy} = I_{Oz} = \frac{1}{2}I_O.
	\label{electron_intensity}
\end{equation}
The incident X-ray will accelerate the electron at an angle of $\pi/2$ in the $yOp$ direction and thus the contribution to the reflected intensity will be
\begin{equation}
	I_{y0p} = I_O \frac{K}{r^2}.
	\label{intensity_yOp}
\end{equation}
Similarly, the electron will be accelerated in the $zOp$ direction at an angle of $\pi/2 - 2\theta$ and the reflected intensity will be
\begin{equation}
	\begin{split}
		I_{zOp} &= I_O \frac{K}{r^2}\sin^2\left({\pi/2 - 2\theta}\right) \\
		& = I_O \frac{K}{r^2}\cos^22\theta.
	\end{split}
	\label{intensity_zOp}
\end{equation}
Putting together \ref{electron_intensity} with \ref{intensity_yOp} \& \ref{intensity_zOp} the reflected intensity at a distance $r$ will be
\begin{equation}
	\begin{split}
		I_p &= I_y + I_z \\
		& = I_O\frac{K}{r^2} + I_O\frac{K}{r^2}\cos^22\theta \\
		& = I_O \frac{K}{r^2}\left(\frac{1 + \cos^22\theta}{2}\right). \\
	\end{split}
	\label{polarization}
\end{equation}
The above term is called the Thomson equation or the polarization factor and accounts for intensity loss due to interaction of incident X-rays with electrons in the unit cell.

\paragraph{Scattering by an Atom}
The scattering intensity of X-rays incident on an atom can be described as the collective behavior of the scattering of all electrons bound to the atom. \ref{reflected_intensity} describes the scattering by one atom. The scattering  intensity by many atoms is the ratio of the intensity of the X-rays scattering by the atom to the intensity scattered by one electron or
\begin{equation}
	f = \frac{I_{\text{atom}}}{I_{\text{electron}}}
	\label{form_factor}
\end{equation}
which is sometimes called the \textit{form factor} because the scattered intensity is related to the shape of the electron cloud around the atom. When electrons scatter X-rays in the forward direction, or $\alpha = 0$ in \ref{reflected_intensity} the scattered X-rays are in phase and interfere constructively. However, when electrons scatter X-rays at any other angle there will be some amount of interference. The interference is strongest in the back-scattering condition. Thus, the scattered intensity of an atom is the combination of all scattering events that occur in all directions.

\paragraph{The Debye-Waller Factor}
The Debye-Waller factor describes how the intensity of diffracted X-ray beams decreases with increasing thermal motion of atoms in a crystal \cite{Cullity}. For an atom with mean displacement $\bar u$ the Debye-Waller factor predicts that X-ray intensity decreases with
\eqn
	f = f_0 \exp{\left[-2M\right]}
	\label{debyewaller}
\equ
where $f_0$ is the scattering intensity of the atom and $M$ is given by
\eqn
	M = 8\pi\bar u^2 \left(\frac{\sin\theta}{\lambda}\right). 
	\label{uiso}
\equ
The term $\bar u$ is also sometimes referred to as $U_{\text{iso}}$ because it accounts for isotropic movement of atoms about their location on the crystal lattice.

The $U_{\text{iso}}$ becomes very important when studying materials at temperature because the analysis of X-ray diffraction data often involves interpretation of peak intensities, which are lessened at heightened temperature. In particular, for a material that is transforming through a liquid-solid-solid phase transformation, characterizing the phase fractions -- which are calculated from relative peak intensities of each phase -- can be disrupted by the decrease in X-ray intensity due to temperature effects. 

$U_{\text{iso}}$ was fit for every X-ray diffraction study which examined materials at high temperatures and was found to be negligible in comparison to other factors impacting peak intensity, namely the phase transformation which was observed. 

\paragraph{The Full Intensity of a Diffracted Beam}
By combining all of the preceding factors for X-ray intensity, the full intensity of a diffracted beam can be described by
\begin{equation}
	I_{hkl} = \left(\frac{I_0A}{32\pi r}\right) 
	\left[ 
	\left(	\frac{\mu_o}{4\pi}	\right)
	\frac{e^4}{m^2}
	\right] 
	\left( 	\frac{1}{v^2}	\right) 
	\left[ 
	|F(hkl)|^2p
	\left( 	\frac{1+\cos^22\theta}{\sin^2\theta\cos\theta}	\right)
	\right] 
	\left(	 \frac{e^{-2M}}{2\mu}	\right)
	\label{full_intensity}
\end{equation}
which is composed of the following terms:
\begin{itemize}
	\item $I_0$ -- the intensity of the incident beam
	\item $A$ -- the cross-sectional area of the illuminated beam
	\item $r$ -- the radius from the diffracted volume to the camera/detector
	\item $\mu_0$ -- the permittivity of free space
	\item $e$ -- the charge of an electron
	\item $m$ -- the electron mass
	\item $v$ -- the unit cell volume
	\item $|F(hkl)|$ -- the structure factor
	\item $p$ -- the multiplicity factor for plane $(hkl)$
	\item $\theta$ -- the Bragg angle
	\item $M$ -- the Debye-Waller factor
	\item $\mu$ -- the linear absorption coefficient of the material
\end{itemize}
The above equation combines all intensity correction factors that have been discussed in this section. Starting from \ref{full_intensity} one can begin to derive the calculation of phase fractions in a material, a topic which will be important to this thesis.
%%%%%%%%%%%%%%%%%%%%%%%%%%%
\subsection{Calculation of Phase Fractions}
The accurate calculation of phase fractions in a material is dependent upon several key assumptions. In particular, it is assumed that
\begin{itemize}
	\item The orientation of grains in the sample is random;
	\item The diffracted specimen can be regarded as a flat plate with infinite thickness relative to the size of the beam;
	\item Diffraction conditions are met at all angles $\theta$ in the material.
\end{itemize}
If the above conditions are not met then the following derivation is dubious. This derivation is based on the treatment provided by Cullity in Chapter 12 of \textit{Elements of X-ray Diffraction} \cite{Cullity}. 

The intensity diffracted by an individual phase in solution with other phases is dependent on the concentration $c$ of that phase in the solution. The concentration in solution for a phase $\alpha$ can be calculated as 
\begin{equation}
	c_\alpha = \frac{w_\alpha \rho_m}{\rho_\alpha}
	\label{concentration}
\end{equation}
where $w_\alpha$ is the weight fraction of $\alpha$ in the mixture, $\rho_m$ is the density of the mixture and $\rho_\alpha$ is the density $\alpha$. The intensity diffracted by $\alpha$ is directly related to the concentration of $\alpha$ in the mixture. Setting all factors constant for a given temperature in \ref{full_intensity}, the diffracted intensity of phase $\alpha$ is
\begin{equation}
	I_\alpha = \frac{K_2 R_\alpha c_\alpha}{\mu_m}
	\label{intensity_alpha}
\end{equation}
where $\mu_m$ is the linear absorption coefficient of the mixture, $K_1$ is a constant, and $R$ is dependent upon $\theta, (hkl)$. The relationship between $I_\alpha$ and $c_\alpha$ is generally nonlinear because $\mu_m$ will change as the concentration of phases changes in the material. 

We will discuss one possible method to calculate phase fraction referred to as the \textit{direct comparison method}. In this case the phase fraction will be calculated from the relationship of intensities diffracted by each phase in the material. There are other methods possible, such as comparing intensities to a known reference standard or doping the material with a known pure material.

Similar to \ref{intensity_alpha} the intensity of another phase $\beta$ becomes
\begin{equation}
	I_\beta = \frac{K_2R_\beta c_\beta}{\mu_m}.
	\label{intensity_beta}
\end{equation}
Division of \ref{intensity_alpha} and \ref{intensity_beta} yields
\begin{equation}
	\frac{I_\alpha}{I_\beta} = \frac{R_\alpha c_\alpha}{R_\beta c_\beta}.
	\label{divided_intensity}
\end{equation}
If the structure factor and lattice parameters of each phase are known then $R$ can be solved for and \ref{divided_intensity} reduces to a function of concentration and intensity only. By introducing a third equation
\begin{equation}
	c_\alpha + c_\beta = 1
	\label{phase_unity}
\end{equation}
the problem becomes a full-rank system of linear equations and the concentration can be solved for from measurements of $I$. 

When calculating phase fractions it is important to use the integrated intensity $\int I dI$ instead of the absolute intensity $|I|$. Broadening of peaks will impact the value of $|I|$ but will not impact the value of $\int I dI$. It is also important to consider the effects of preferred orientation of grains on the results. Preferred orientation will reduce the peak intensity of certain peaks based on the orientation of grains. This can be overcome by rotating the sample during measurement and collecting diffraction images at every rotation. Then, the images at every rotation angle can be summed to wash out the effect of preferred orientation.


%%%%%%%%%%%%%%%%%%%%%%%%%%%
\subsection{Scattering Through Amorphous Materials}
Guinier et al. provide a comprehensive model for the scattering of X-rays through amorphous materials \cite{Guinier1994}. Amorphous materials in general -- but specific to this discussion, liquids -- can be described as a monatomic gas scattering X-rays. The pair correlation of the material determines the scattering specifics of the amorphous material.

As shown in \ref{amorphous_scatter} amorphous scattering displays a characteristic length in $d$-spacing at which intensity is maximized. In some cases, two or more peaks can be observed such as in 308L stainless steel \textbf{get an image of the two peaks in SS}. 

The density of liquid scattering X-rays can be determined if the pair correlation function of the material is known. Unfortunately,, the pair correlation function is difficult to determine experimentally and has to be modeled.

\begin{figure}
	\includegraphics[width=1\linewidth]{/Users/njohnson/git/thesis/document/chapters/introduction/introduction_images/amorphous_scatter.png}
	\caption{Scattered intensity of liquid Ti-6Al-4V}
	\label{amorphous_scatter}
\end{figure}

%%%%%%%%%%%%%%%%%%%%%%%%%%%%%%%%%%%%%%%
\subsection{Measurement of Stress and Strain}\label{diffraction_stress_strain}
High energy X-ray diffraction can be used for the measurement of elastic strains in materials. Strain is measured as
\begin{equation}
	\epsilon = \frac{a_i - a_0}{a_0}
	\label{diffraction_strain}
\end{equation}
where $\epsilon$ is the strain on the crystal, $a_i$ is the lattice parameter at a given load step, and $a_0$ is the reference lattice parameter. Choices of $a_0$ are many and varied and depend on both the application being studied and the desired outcome for measurement. A common choice of $a_0$ is the lattice parameter in an unstressed material, though the definition of `unstressed' in a material is not always clear. For some cases it may make sense to compare to a stress annealed sample. In the case of materials being loaded the lattice parameter in the unloaded state may be used. 

Equation \ref{diffraction_strain} is useful in the case of an isotropic material with no directional dependence on lattice parameter. For rapidly solidified materials with directional cooling there is often an orientation dependence in lattice parameter. In this case, it may make sense to use an orientation-dependent measurement of strain
\begin{equation}
	\epsilon_{\eta,\omega,i} = \frac{a_{\eta,\omega,i} - a_{\eta,\omega,0}}{a_{\eta,\omega,0}}
	\label{directional_strain}
\end{equation}
where $\eta$ and $\omega$ are defined in Section \ref{detector_setup}. 

The determination of orientation dependent strains can be related to a strain tensor using the theory of elasticity. The strain tensor is given by
\begin{equation}
	\begin{split}
	\epsilon_{\eta,\omega} &  = \epsilon_{xx}\cos^2\omega\sin^2(\eta) + \epsilon_{yy}\cos^2(\eta) + \epsilon_{zz}\sin^2\omega\sin^2(\eta) \\
	& + \epsilon_{xz}\sin2\omega\sin^2(\eta) + \epsilon_{yz}\sin\omega\sin(2\eta) + \epsilon_{xy}\cos\omega\sin(2\eta).
	\label{diffraction_strain_tensor}
	\end{split}
\end{equation}
Equation \ref{diffraction_strain_tensor} requires the measurement of $\epsilon$ at multiple values of $\eta$ and $\omega$. For a measurement at a fixed rotation $\omega$ only two components of $(x,y,z)$ can be measured at a time. Therefore, the sample must be rotated in $\omega$ to determine the third strain component. However, for any meaningful measurement of strain the same grains must be sampled at all $\omega$ rotations. Methods such as the strain slit method can be used to ensure that the same volume is illuminated during measurement at all rotations \cite{Strantza2018}.

Equation \ref{diffraction_strain_tensor} can be re-written in terms of principal strains $\epsilon_1, \epsilon_2$ by
\begin{equation}
	\epsilon_{\eta} = \epsilon_1 \sin^2(\eta + \psi) + \epsilon_2\cos^2(\eta + \psi)
	\label{diffraction_principal_strains}
\end{equation}
where $\psi$ is the orientation of the principal coordinate system.

\begin{figure}
	\includegraphics{/Users/njohnson/git/thesis/document/chapters/introduction/introduction_images/principal_strain_diagram.png}
	\caption{An exaggerated diagram of the distortion of a diffraction ring due to stress applied to a sample. The principal coordinate direction $\psi$ is shown relative to $\eta = 0$ in the detector coordinate system. The principal strains $\epsilon_{11}$ \& $\epsilon_{22}$ are the primary compressive and tensile strains, respectively.}
	\label{principal_strain_diagram}
\end{figure}

Figure \ref{principal_strain_diagram} demonstrates how principal strains develop in a sample under load. Under assumptions of isotropy the diffraction ring will deform into an ellipse. The minor axis of the ellipse corresponds to $\epsilon_1$ and the major axis corresponds to $\epsilon_2$ and $\psi$ is the orientation of the minor axis relative to $\eta =0$ on the detector. When $a_{\eta}$ is plotted as a function of $\eta$ it appears as a sinusoidal function where the maximum amplitude of the wave is $\epsilon_1$, the minimum amplitude is $\epsilon_2$ and the phase of the wave is $\psi$. 

%%%%%%%%%%%%%%%%%%%%%%%%
\subsection{Preferred Orientation of Samples}
Additive manufacturing, or rapid solidification specifically, is known to cause preferred orientation of crystals in the sample. Preferred orientation is most often observed with the close packed direction of the alloy oriented in the direction of maximum heat flow during solidification \textbf{Find a citation}. Table \ref{orientation_directions} outlines the close packed direction of each phase studied in this thesis.

\begin{table}\caption{Close packed directions for the crystal structure of each phase studied in this thesis. The close packed direction is often found to be oriented in the direction of maximum heat flow during solidification for additively manufactured alloys.}
	\label{orientation_directions}
	\begin{center}
		\begin{tabular}{c|ccccc} \hline
		Phase & $\alpha$ Ti-6Al-4V & $\beta$ Ti-6Al-4V & $\beta$ Ti-5553 & $\delta$ SS 308L & $\gamma$ SS 308L \\
		Crystal Structure & HCP & BCC & BCC & FCC & BCC \\
		Close Packed Direction & [0001] & [100] & [100] & [111] & [100] \\
		\end{tabular}
	\end{center}
\end{table}

%%%%%%%%%%%%%%%%%%%%%%
\subsection{X-ray Diffraction as a Probe for Measurement of Temperature}
When the coefficient of thermal expansion is known in a material it can be possible to measure temperature of the material from the $d$ spacing of the crystal lattice.

As a material is heated the $d$ spacing of the crystal increases due to increased thermal energy. The amount of $d$ spacing change can be quantified using strain. Strain $\epsilon$ is measured in the material by taking the lattice parameter $a_i$ at temperature $i$ and normalizing it to a reference parameter $a_0$ through
\begin{equation}
	\epsilon_i = \frac{a_i - a_0}{a_0}.
	\label{crystal_strain}
\end{equation}
The determination of $a_0$ is not straightforward and several different values can be used. In general, it must be assumed that temperature is the only source of strain for temperature to be measured in a physically meaningful way. 

In the case of an isotropic material experiencing even cooling in all directions the lattice parameter at room temperature can be used. For this case the strain is related to temperature by
\begin{equation}
	\epsilon = \alpha \Delta T
	\label{isotropic_temperature}
\end{equation}
where $\alpha$ is the coefficient of thermal expansion of the material. Equation \ref{isotropic_temperature} is analogous to Hooke's law, where the only stress on the material is a hydrostatic stress due to temperature. The value of $\alpha$ is sometimes considered to be a single scalar value although this assumption falls apart at high temperatures. Instead, a fit of the strain as a function of temperature must be used. For the case of a polynomial
\begin{equation}
	\epsilon = \alpha_0 + \alpha_1 T + \alpha_2 T^2 + \ldots + \alpha_n T^n
	\label{polynomial_temperature}
\end{equation}
For example, Touloukian \cite{Touloukian1975} provides a measurement of the strain on SS 308L as a function of temperature, given by
\begin{equation}	
	\epsilon = -0.358 -2.978\times10^{-10}T^3 + 1.031\times10^{-6}T^2 +  9.472\times10^{-4} T 
	\label{SS_temperature}
\end{equation}
The measured strain can be fit to temperature by finding the polynomial roots of Equation \ref{SS_temperature} and solving for $T$. 

The use of $a_0$ at room temperature can be misleading if certain assumptions are not met. First of all, fits such as Equation \ref{SS_temperature} are made for stress-annealed macroscopic samples of SS 308L. Therefore, if the sample being measured has residual stresses present at room temperature the datum may cause an offset in the measured temperature. Furthermore it is assumed that the development of stress during heating or cooling is only due to temperature. This assumption can readily fall apart for several types of materials. In particular dual phase alloys that have significantly crystal structures or lattice parameters between the phases can cause interfacial stresses to develop at grain boundaries. These stresses need to be isolated and quantified before physically meaningful temperatures can be measured. 

Equation \ref{isotropic_temperature} also assumes even cooling in all directions. That is, it assumes that all heat transfer modes -- conduction, convection, radiation -- are occurring evenly in all directions. For most weld and additive manufacturing samples this is a poor assumption. Welds are often deposited on a water cooled substrate that acts as a heat sink for the sample. Therefore directional cooling occurs and the isotropy assumption is invalidated. In this case it may make sense to measure a direction-dependent strain
\begin{equation}
	\epsilon_{\eta,\omega,i} = \frac{a_{\eta,\omega,i} - a_{\eta,\omega,0}}{a_{\eta,\omega,0}}
	\label{directional_strain}
\end{equation}
where $\eta$ and $\omega$ are defined in Section \ref{detector_setup}. The orientation dependent strains can be fit to principal strains as described in Section \ref{diffraction_stress_strain}. Once the orientation of the principal tensile and compressive directions $\psi$ are found temperature fits can possibly be used. In general, for a material being heated it may make sense to use the tensile strains and for a material being cooled the compressive strains can be used.

%%%%%%%%%%%%%%%%%%%%%%%%%%%%%%%%%%%
\subsection{Diffraction Information Specific to $\alpha + \beta$ Ti alloys and SS 308L}
Ti-6Al-4V has two stable phases at room temperature. The first phase, called $\alpha$ phase in the literature, is a hexagonal close packed structure with a basis of 2 and a space group of P6$_3$/mmc. In traditionally manufactured (cast and wrought) Ti-6Al-4V the $\alpha$ phase is present in weight percents between $90-94\%$. The other phase, termed the $\beta$ phase is a body centered cubic structure with a space group of Im$\bar3$m.

Stainless Steel 316L is an austenitic, two-phase stainless steel alloy. Its primary phases are referred to as austenite $\gamma$ and ferrite $\delta$. The austenite phase is a face centered cubic phase with space group Fm$\bar3$m. The ferrite phase is a body centered cubic phase with space group Im$\bar3$m.

Ti-5Mo-5V-5Hf-3Cr (Ti-5553) is a single phase titanium alloy with a crystal symmetry of body centered cubic and a space group of Im$\bar3$m. 

%%%%%%%%%%%%%%%%%%%%%%%%%%%%%%%%%%%
\subsection{Refinement Parameters in GSAS and GSASII}
Topics to cover:
\begin{itemize}
	\item models laid out in the GSAS handbook
	\item Image integration
	\item Calibration
	\item The refinement parameters used \begin{itemize}
		\item Lattice parameter
		\item Broadening
		\item Preferred orientation
		\item Phase fractions
		\end{itemize}
	\item Automated image processing
	\item Automated refinement (?)
\end{itemize}

\paragraph{Image Integration}
The type of image integration used depended largely on the desired outcome for refinement. There were two primary outcomes pursued in this thesis: the measurement of stress and strain and the calculation of phase diagrams.

The integration of images is dependent upon good detector calibration. All calibrations were performed by using CeO$_2$, a common standard with a known crystal structure. Diffraction images of CeO$_2$ were taken at a fixed position $z$ in the laboratory coordinate frame, with the distance from the sample to the detector being known roughly. GSASII was used to perform image calibration. The known wavelength $\lambda$ of the beam and the rough sample-to-detector distance are input into GSASII along with a guess as to the image center. Then, refinement of the image is performed until a simulated pattern of CeO$_2$ matches well with the observed pattern on the image. The image refinement process refines the sample-to-detector distance, the image center position, as well as the tilt and rotation of the detector relative to the sample. A new CeO$_2$ calibration image is taken for every new sample that has been moved in the laboratory $(x,y,z)$ coordinate frame.

Once image calibration is performed adequately, integration of diffraction images for the sample of interest can be performed.

For the measurement of stress and strain data needs to be `binned' into slices of $\eta$ so that the expansion and contraction of the diffracted cone can be observed. An example of one such image binning is shown in \ref{image_integration}. For all studies herein the images were binned into 24 bins of 15$^\circ$ each. The range of integration in the $2\theta$ direction depended on the size of the image, the crystal structure, and the type of detector used. Once images were integrated into slices of $\eta$ Equation \ref{diffraction_principal_strains} was used to determine the major and minor axes of the diffracted ellipse, and thus the principal strain directions.

\begin{figure}
	\includegraphics[width=1\linewidth]{/Users/njohnson/git/thesis/document/chapters/introduction/introduction_images/image_integration}
	\caption{A diffraction image taken on a Dexela detector showing the diffracted rings and the bounds of the integration. The image was binned into 15$^\circ$ increments, shown as the `slices' between dashed lines.}
	\label{image_integration}
\end{figure}

Phase analysis of diffraction images requires good grain statistics for each diffracted ring, sometimes referred to as a `powder diffraction' condition. The determination of phase fractions follows from Equations \ref{divided_intensity} and \ref{phase_unity} and thus requires good measurement of the diffracted intensity of each phase. Thus, full integration in both $\eta$ and $2\theta$ is required. After integration, Rietveld Refinement was used to measure the diffracted intensity of each phase and calculation was performed by GSASII. 

\paragraph{Refinement Parameters}
Refinement was primarily performed using the General Structure Analysis Software (GSAS) II. The Rietveld refinement parameters chosen have a major impact on the success and results of the refinement and should be discussed.

The first parameter refined was the lattice parameter of the crystal unit cell. GSASII allows independent movement of the unit cell lattice parameters through the parameters $D_{ii}$ where $ii$ is an entry in the unit cell metric tensor. For a cubic material only one parameter $D_{11}$ is refined, while for HCP materials two parameters $D_{11}$ and $D_{33}$ are refined. 

Once the lattice parameters were reasonably fit the next step was to refine the peak broadening. The sources of peak broadening are many and varied and broadening is often present in rapidly solidified materials due to the nature of rapidly solidified crystals. Some sources of broadening include crystallite size effects, dislocations, and mosaicity in the crystal. Dislocations are often present in a high density in rapidly solidified materials. Crystallite size effect was a major factor Ti-6Al-4V because the $\beta$ crystals were often on the order of nanometers. There is a separate parameter for crystallite size broadening in GSASII from general peak broadening. Both were refined when necessary.

When images were binned into increments there was often intensity differences in the histograms due to preferred orientation in the samples. When the data was being analyzed for stress and strain an individual preferred orientation model was fit to each histogram. Doing so allowed for the peak intensities of each histogram to vary independently and provide better fits of the integrated intensity. 

Spherical harmonic functions were used to fit the intensity changes due to preferred orientation. 




%~~~~~~~~~~~~~~~~~~~~~~~~~~~~~~~~~~~~~~~~~~~~~~~~~~~~~~
\subsection{The Materials Science of Rapidly Solidified Alloys}
%%%%%%
\paragraph{Material Properties of Ti-6Al-4V}
This is where we include important background information on Ti-6Al-4V including things like:
\begin{itemize}
	\item Composition
	\item Phase diagram
	\item Thermomechanical properties including CTEs and moduli
	\item Microstructural characteristics, both the possible microstructures and the ones we observe
	\item Deviations in material properties between traditional manufacturing and wire feed DED
\end{itemize}

\paragraph{Composition}
The nominal composition of Ti-6Al-4V is shown in Table \ref{Ti64_comp}.
\begin{table} \caption{The composition of Ti-6Al-4V in weight percent, taken from \cite{Ti64matweb}.} \label{Ti64_comp}
	\begin{center}
		\begin{tabular}{cccccccc}\hline
		Ti & Al & V & C & H & Fe & N & O\\ \hline
		Balance & 6 & 4 & 0.08 & 0.015 & 0.4 & .03 & 0.2 \\
		\end{tabular}
	\end{center}
\end{table}
			

\paragraph{Phase Diagram}
A pseudobinary TiAl-V phase diagram can be seen in Figure \ref{Ti64phasediagram}. Ti-6Al-4V is a two-phase $\alpha + \beta$ alloy. It has a melting point around 1650$^\circ$ C and a solidus at 950$^\circ$ C where the $\beta \to \alpha$ transition occurs. Ti$_3$Al, shown at room temperature in Figure \ref{Ti64phasediagram} has not been observed in the samples studied herein. 

Some phase diagrams of Ti-6Al-4V also include a transition temperature for a martensitic phase transformation. The existence of this type of phase transformation in Ti-6Al-4V is under contention in this thesis and will be discussed in greater detail later.

\begin{figure}
	\includegraphics[width=1\linewidth]{/Users/njohnson/git/thesis/document/chapters/introduction/introduction_images/ElmerPalmerPhaseDiagramTi64}
	\caption{A pseudobinary phase diagram for the TiAl-V system. Ti-6Al-4V is shown as the blue line. Image taken from Elmer, Palmer \cite{Elmer2005}}
	\label{Ti64phasediagram}
\end{figure}

\paragraph{Microstructures of Ti-6Al-4V}
Sources to cite on the microstructures of Ti6Al4V and what they discuss
\begin{itemize}
	\item Plichta et al. \cite{Plichta1977} this is one of the earliest sources discussing the `massive' transformation form beta to alpha Ti in the systems Ti-Au, Ti-Ag, and Ti-Si. From 1977. Likely has some information that I would now disagree with
	\item Xu et al. 2015 \cite{Xu2015}. This article discusses `in situ martensite decomposition' during the SLM process. 
	\item Xu et al. 2017 \cite{Xu2017}. This is an important article because they do EDS analysis and show that even in rapidly cooled (SLM) Ti-6Al-4V there is still a higher V content in small $\beta$ laths. This indicates that the $\beta \to \alpha$ diffusional transformation occurs even under rapid solidification conditions, contradicting the idea that a martensitic phase transformation occurs in Ti-6Al-4V.
	\item Lim and Rollett 2020 \cite{Lim2020}. This is Tony Rollett's student who measured the CTEs of Ti-7Al. Could be relevant to our investigations.
	\item Dumontet 2019 \cite{Dumontet2019}. This paper measures the elastic properties of the martensitic phase of Ti-6Al-4V. One important aspect of this is that they use laboratory XRD to quantify the amount of $\beta$ phase -- this is NOT an acceptable method of quantification. It is very likely that the $\beta$ phase was present in the as-printed condition but the crystallite size was so small that peaks could not be detected (especially by a lab XRD). They made the explicit assumption that the as-printed microstructure was martensite and that a heat treated microstructure was $\alpha + \beta$. They did no microscopy to confirm this.
	\item Ter Harr and Becker \cite{TerHarr2018}. This article goes into a DEEP dive on transformation mechanisms of selective laser melted Ti-6Al-4V microstructures during heat treatment. It talks a lot about the $\alpha'$ martensitic phase and its decomposition. It makes the argument that there is a higher V content in the martensitic phase as observed in XRD data. A very good reference to come back to, especially for transformation mechanisms.
	\item Just a note for myself -- a lot of people identify martensitic $\alpha'$ as a brighter version of the $\alpha$ phase in microscopy. They argue that the higher V content leads to this discoloration. Isn't it possible they are just looking at $\beta$ phase?
	\item Wang et al 2020 \cite{Wang2020a}. Need to come back to this one. Can't quite concentrate at the moment. It focuses on heat treatments of Ti-6Al-4V and the transformation mechanisms resultant.
	\item Hennig et al. \cite{Hennig2008} describes a DFT method for assessing martensitic transformation pathways from $\beta \to \alpha \to \omega$. Not super relevant.
	\item Zhang et al. \cite{Zhang2019}. Figure 5 shows compositional segregation of V and Al near the heat affected zone and fusion zone of an EBM weld. There is significant segregation of the V in the baseplate (two phase microstructure) and less segregation in the near HAZ and the fusion zone -- BUT there is still segregation nonetheless. They also demonstrate in Figure 6 that the V concentration in the fusion zone goes all the way up to 10 wt pct for certain (likely beta) phases. They cite an interesting statistic from Luterjing et al that says that V has a low solubility in the alpha phase
	\item Mishin and Herzig \cite{Mishin2000} show many diffusion rates and constants for species in the Ti-Al system --- V does not seem to be represented.
	\item Williams \cite{Williams1972} provides a review of martensites in titanium alloys. This is the OG paper on martensite in titanium. He identifies many different types of martensites including hexagonal and orthorhombic. This is where the terminology $\alpha'$ is solidified for the first time in the literature. This is a seminal paper that doesn't provide too much useful information but should be cited for its historical significance.
	\item Stanford and Bate \cite{Stanford2004} both measure and calculate that significant crystallographic variant selection is occurring during diffusional transformation of Ti-6Al-4V, showing that the Burger's OR does not happen with equal probability for all variants. This study found that when two adjacent $\beta$ grains share a common orientation (low misorientation) the variants produced from the $\beta \to \alpha$ transition are also going to be nearly the same, i.e. the alpha variants within both prior beta grain boundaries will be nearly the same.
	\item Qazi et al. \cite{Qazi2003} is a hard article to critique because it starts with the assumption that $\alpha'$ martensite exists and forms upon quenching from the $\beta$ phase field. It looks at martensite decomposition as a function of hydrogen content, aging time, and aging temperature. The results for 0$\%$ H are not that useful or promising.
	\item Stapleton et al. \cite{Stapleton2008} ran an EPSC model to compare to diffraction-measured lattice strains. They looked at slip and critical resolved shear stress in basal, pyramidal, and prismatic slip systems. Not a ton of information relevant to my thesis.
	\item The article by A. Safdar et al. \cite{Safdar2012} involves considerable conjecture and provides little new information to the discourse. It discusses phase transformation mechanisms in EBM. The article is mostly citations of other's work and conjecture about how it applies to microstructure formation in EBM.
	\item Farabi et al. \cite{Farabi2018} has some good information but you're a little distracted right now. At the very least you should reference Figure 9 in this article when discussing the slip/shear pathways for the martensitic transformation
	\item M. Ali et al \cite{Ali2020} wrote an article about the texture effect of the $\beta$ phase after cyclic heat treatments above the $\beta$ transus. Not sure that there is a lot here relevant to my research.
	\item Wang et al \cite{Wang2019} just reports some things they observed for WAAM Ti-6Al-4V...not really a ton of `research' in this article.
	\item Katzarov, Malinov, and Sha \cite{Katzarov2002} present an interesting model for the diffusion of V during $\beta \to \alpha$ transitions in Ti-6Al-4V. They point out some facts that I haven't seen elsewehere; Figure 1 is particularly useful, it provides concentrations of Al and Ti in the various phases as a function of temperature. It also points out that the major driving force of the $\beta \to \alpha$ transition is the diffusion of V, not the diffusion of Al. The Al concentration is roughly the same for the $\beta$ and $\alpha$ phases. They only look at cooling rates up to $40^\circ$C/s unfortunately. Their model predicts and equilibrium V concentration near 13\% for the final $\beta$ phase.
	\item Banerjee and Williams \cite{Banerjee2013} provide a fairly comprehensive review article on the materials science of Titanium alloys. They describe the process of martensite formation in $\beta$-stabilized Ti as being 1) Bain transformation, 2) shear into the hcp structure; and 3) slip/glide of a plane so that all atoms are sitting in the correct site. But how do they know this?? I still don't understand what evidence exists to support these claims. The article they cite is another article by Banerjee on the topic; I would like to see other authors being cited.
	\item Yang et al \cite{Yang2016} provide a good argument FOR the existence of the martensitic phase transformation in Ti-6Al-4V. They show some TEM micrographs which they claim have evidence of twinning, a sure sign (right?) of the martensitic phase transformation. They do not do compositional analysis. Figure 7 shows the phase transformation pathway for SLM and is a good resource to cite.
	\item Krakhmalev et al. \cite{Krakhmalev2016} did not see any $\beta$ phase in their TEM investigation of additively manufactured Ti-6A-4V. They did observe twinning of $\alpha$ grains however. An interesting counterpoint to our hypothesis.
	\item Murr et al. \cite{Murr2009} compares microstructures of cast, wrought, EBM, and SLM Ti-6Al-4V biomedical implants. Claims to see martensite based on optical metallography and TEM but does not go far enough to perform compositional analysis
	\item Compositional analysis! \cite{Matsumoto2011} Matsumoto et al. found a V content in acicular HCP of around 4.4 percent while equiaxed grains had a V enrichment of around 2.1 percent. The Al content is the same between the two.
	\item Sun et al. \cite{Sun2019} discuss the microstructure of SLM Ti-6Al-4V. Using the language of Ahmed and Rack they identify primary, secondary, tertiary, and quartic $\alpha'$ phases. Their language is loose and they discuss precipitation of the secondary, etc., $\alpha'$ on the grain boundary of primary $\alpha'$. However, a martensitic phase cannot `precipitate' because that implies nucleation and growth which invalidates the assumption that the phase formed through a martensitic transformation
	
\end{itemize}
\subsubsection{Material Properties of Stainless Steel 308L}

\begin{figure}
	\includegraphics[width=1\linewidth]{/Users/njohnson/git/thesis/document/chapters/introduction/introduction_images/SS_phase_diagram}
	\caption{Pseudobinary phase diagram for the FeCr-Ni system.}
	\label{SS_phase_digram}
\end{figure}

Information to include in this section"
\begin{itemize}
	\item Crystal structures and allotropes of the 308L
	\item phase digram
	\item The fits used for the temperature calculation
	\item Composition
\end{itemize}

The composition of 308L stainless steel can be seen in Table \ref{SS_compo}.
\begin{table}\caption{Composition of 308L Stainless Steel in weight percent, taken from \cite{SSMatweb}.}\label{SS_compo}
	\begin{center}
		\begin{tabular}{cccccccc} \hline
		Fe & Cr & Ni & C & Mn & P & Si & S  \\ \hline
		Balance & 20 & 11 & 0.08 & 2 & 0.045 & 1 & 0.03 \\
		\end{tabular}
	\end{center}
\end{table}

The temperature fits used for the SS temperature calculation is
\begin{equation}
	\frac{\delta \ell}{\ell_o} = -0.358 + 9.472\times10^{-4}T + 1.031\times10^{-6}T^2 - 2.978\times10^{-10}T^3
	\label{SS_temperature_calculation}
\end{equation}
and is taken from Touloukian et al \cite{Touloukian1975}.
\subsubsection{Material Properties of Ti-5553}
The composition of Ti-5553 can be seen in \ref{TiFi_compo}. It is a $\beta$ Ti alloy with a BCC crystal structure and a space group of Im$\bar3$m. The Ti-5553 used in this investigation was additively manufactured using laser powder bed fusion. 

\begin{table}\caption{Composition of Ti-5553 in atomic percent.}
	\label{TiFi_compo}
		\begin{center}
			\begin{tabular}{ccccc}\hline
				Ti & V & Hf & Mo & Cr \\ \hline
				Balance & 5 & 5 & 5 & 3 \\
			\end{tabular}
		\end{center}
\end{table} 

\chapter{Scientific Approaches to Researching, Modeling, and Engineering Additively Manufactured Materials}
This section probably needs to come first because (at least at the moment) it contains a lot of literature review that is relevant to the later chapters.

\chapter{In Situ High Energy X-ray Diffraction of Cold Metal Transfer Welded SS308L}


\begin{itemize}
	\item Experimental goals
	\item Experimental setup
	\item Liquid phase fraction measurement
	\item Temperature measurement
	\item Results
	\item Discussion
\end{itemize}

\begin{figure}
	\includegraphics[width=1\linewidth]{/Users/njohnson/git/thesis/document/chapters/ssweldbead/Images/weldsetup1}
	\caption{}
	\label{weldsetup1}
\end{figure}

\subsection{Experimental Goals}
The original goal of this investigation was to measure temperature based on lattice parameters in rapidly solidifying stainless steel 308L weld beads. A picture of one such weld bead can be seen in Figure \ref{weldsetup1}. 

\subsection{Experimental Setup}
The welder setup can be seen in Figure \ref{weldsetup1}. Single beads of stainless steel 308L were deposited onto a stainless steel 304L water cooled substrate. Weld deposits were made using a Fr\"onius Cold Metal Transfer welder. Each weld deposit was approximately 2.5mm in width and 2.5mm in height. 

X-ray diffraction experiments were performed at the 1ID beamline of the Advanced Photon Source, Argonne National Laboratory. A 71 keV X-ray beam was used. As samples were being deposited the X-ray beam was shone on the liquid sample. Data was collected through deposition and until the beads were fully solidified. Data was collected using a dual-detector setup. One detector was and ASI Lynx detector \cite{LynX} with a refresh rate of 200 Hz and a small azimuthal coverage in the diffraction cone. The other detector was a GE detector with a refresh rate of 10 Hz and an azimuthal coverage of around 180 degrees. 

After deposition weld beads were cut from the substrate and EBSD microscopy was performed. An example EBSD image can be seen in Figure \ref{SS_EBSD_weldbead}. \textbf{write up grain size and orientation information}.

Data was analyzed using the General Structure Analysis Software II (GSASII) \cite{GSASII}. Data was fully integrated on both detectors in order to provide maximal intensity and coverage for analysis. The first 1000 images of the ASI detector were analyzed while the full dataset for the GE detector, 120 images, were analyzed. The first 1000 images of the ASI were analyzed because data was repetitious after that point and the number of images taken was too large to be analyzed in a reasonable amount of time. 

The background function of the data needed to be fit in order to analyze the X-ray scattering through the liquid phase. The amorphous phase scattering can be described by the theory of Guinier explained in Section \cite{Guinier1994}. A Gaussian function was fit to the background to capture the dynamics of the amorphous phase. 

Rietveld refinement was used to fit the lattice parameters and intensity of the two phases of the SS308L, the ferrite ($\delta$) and austenite ($\gamma$) phases. The phase fraction of the liquid, ferrite, and austenite phases was fit by using
\begin{equation}
	f_{p} = \frac{p}{\delta_i + \gamma_i + A*B_i}
	\label{ferritefrac}
\end{equation}
where $f_{p}$ is the scale factor of the phase being calculated, $\delta_i$ is the intensity of the ferrite phase at time $i$, $\gamma_i$ is the intensity of the austenite phase at time $i$, and $B_i$ is the intensity of the Gaussian function fit to the background at time $i$. The prefactor $A$ is a multiplier to scale the background function to the same magnitude as the intensity of the phases. 

Temperature measurements were made by fitting the lattice parameter expansion during cooling to the thermal expansion of SS308L as reported by Touloukian et al. \cite{Touloukian1975}. The fit is given by
\begin{equation}
	\frac{\delta \ell}{\ell_0} = -0.358 + (9.472\times10^{-4})T + (1.031\times10^{-6})T^2 - (2.978\times10^{-10})T^3
	\label{touloukian}
\end{equation}

\begin{itemize}
	\item to write up:
	\item integration of the images
	\item fitting of the background function
	\item rietveld refinement
	\item what was calculated
	\item calculation of the liquid phase fraction
	\item measurement of temperature from lattice parameter
\end{itemize}

\subsubsection{Fitting of Background Function}
Signal acquisition began with the deposition of molten (liquid) stainless steel onto a stainless steel or titanium substrate, depending on the sample. The monochromatic X-ray beam was trained on the location of deposition as manufacturing began resulting in X-ray scattering through the initially liquid material. Scattering of X-rays in a liquid follows the same physics as scattering of materials in an ideal gas. Guinier describes this phenomenon in his book \textit{Scattering of X-rays in Crystals, Imperfect Crystals, and Amorphous Materials} \cite{Guinier1994}. 

An example of the signal observed when X-rays were scattered through the molten liquid can be seen in \ref{solidification_steps}. An amorphous ring of diffracted light with a wide spread in the $2\theta$ direction is observed. The integrated diffraction histogram can likewise been seen in \ref{solidification_steps}. The intensity and full width at half maximum (FWHM) are described by a pair correlation function of the liquid. The distribution of scattered intensity from the liquid can be fit using a pseudo-Voigt function. 

The background function was fit for every time step of the acquired data and used to estimate the amount of liquid that was present in the weld. The liquid phase fraction was fit using Equation \ref{ferritefrac}.

\begin{figure}
	\includegraphics[width=1\linewidth]{/Users/njohnson/git/thesis/document/chapters/ssweldbead/Images/weld_setup}
	\caption{Setup of the weld rig, a single weld bead, and the water cooled substrate.}
	\label{weld_setup}
\end{figure}

\begin{figure}
	\includegraphics[width=1\linewidth]{/Users/njohnson/git/thesis/document/chapters/ssweldbead/Images/SS_solidification_steps}
	\caption{Observed diffraction patterns during solidification of a stainless steel weld bead. Diffuse scattering through the liquid melt is observed first, followed by coarse, spotty grains that arise when the high temperature phase forms. After a while many grains start to nucleate and grow in the weld leading to a powder-like pattern observed in the last panel. Integrated diffraction histograms can be seen below each image.}
	\label{solidification_steps}
\end{figure}


\subsubsection{Detectors Used}
Two detectors were used in this experiment. An ASI Lynx detector with a pixel size of 78$\mu$m and detector size of \textbf{look up detector size}. The refresh rate of the ASI detector was 200 Hz, giving an acquisition resolution of $0.005$s per time step. The azimuthal coverage of the Debye ring for the ASI detector was approximately 15$^\circ$ \textbf{double check that value}.

A General Electric $\alpha$-Si detector was also used. It has a pixel size of $200\mu$m and a detector size of $2048 \times 2048$ pixels. The refresh rate of the GE detector was 10 Hz, significantly slower than the ASI detector and likely unable to fully capture the dynamics of the solidification process of SS and TI-6Al-4V. However, the Debye ring azimuthal coverage of the detector is 180$^\circ$ giving much better signal coverage and therefore grain statistics. 

\subsubsection{Rietveld Refinement}
Rietveld refinement was performed using the General Structure Analysis Software II (GSASII). Images were integrated along the full Debye ring represented by the detector. The reason for full integration was to achieve good enough grain statistics for accurate fitting of phase fractions. For the case of the GE detector this was a decent assumption. The spottiness of the data on the ASI detector means that trusting phase fraction measurements from these images is dubious at best. 

Data was fit using a full suite of parameters in GSASII. First the background function and background amorphous scattering was fit for each time step. Once the background function was adequately fit then the ferrite and austenite phases were introduced for fitting. The scale factors and microstrain for each phase were the first parameters to be fit. The strain on the lattice was particularly important to fit because peak shift occurred as the sample cooled and the lattice contracted. Equally important was fitting the $U_\text{iso}$ parameter, or the Debye-Waller factor. This parameter, described in Section \ref{DebyeWaller}, accounts for loss of intensity in the peaks as a function of temperature. For all samples fit the Debye Waller factor was not found to be particularly important for accurate fitting of the intensities but was important to fit as a sanity check. 

After the above parameters were fit then the strain broadening was allowed to vary. Peak broadening can occur for a number of reasons in the sample but was most likely associated with the development of dislocations in the lattice during cooling and contraction.

In some cases texture models were fit to the diffraction histograms to account for anisotropy in peak intensity. However, these texture fits had very little physical meaning because data was only acquired from a single $\omega$ direction and therefore cannot account for texture variations within the sample. \textbf{Does that make any sense?}

\subsection{Results}
\begin{figure}
	\includegraphics[width=1\linewidth]{/Users/njohnson/git/thesis/document/chapters/ssweldbead/Images/raw_scale_factors}
	\caption{Raw scale factors for the austenite and ferrite phases of SS308L as it is solidifying.}
	\label{raw_scale_factors}
\end{figure}

\begin{figure}
	\includegraphics[width=1\linewidth]{/Users/njohnson/git/thesis/document/chapters/ssweldbead/Images/converted_liq_phase_fraction}
	\caption{Scale factors of the austenite and ferrite phase, together with the background function peak intensity, scaled using Equation \ref{ferritefrac} to compute phase fractions during solidification.}
	\label{SS_phase_fractions}
\end{figure}

\begin{figure}
	\includegraphics[width=1\linewidth]{/Users/njohnson/git/thesis/document/chapters/ssweldbead/Images/raw_SS_strain}
	\caption{Raw strain values measured for the ferrite and austenite phases during cooling.}
	\label{raw_SS_strain}
\end{figure}

\begin{figure}
	\includegraphics[width=1\linewidth]{/Users/njohnson/git/thesis/document/chapters/ssweldbead/Images/converted_SS_temp}
	\caption{Lattice strains converted into temperatures using Equation \ref{temperature}, calculated from the strains shown in Figure \ref{raw_SS_strain}.}
	\label{converted_SS_temp}
\end{figure}

%%%%%%%%%%%%%
\paragraph{Microstructure}
The microstructure of an example SS weld bead can be seen in Figure \ref{SS_EBSD_weldbead}. The microstructure is quite heterogeneous both in terms of grain size and orientation. The grain size near the bottom of the weld is on the order of single microns and has a wide distribution of grain orientations. The grain statistics obtained in this region for diffraction are likely high and therefore HEXRD phase fraction characterization is more trustworthy. There is a transition around $0.4$mm in height where the grain size changes and the morphology becomes less equiaxed and more lamellar. Furthermore there is less of a wide distribution of grain orientations in this region. The phase fraction calculations here are less trustworthy because fewer grains would diffract and there are fewer orientations likely to be in the diffraction condition.

\begin{figure}
\begin{center}
	\includegraphics[width=0.5\linewidth]{/Users/njohnson/git/thesis/document/chapters/ssweldbead/Images/SS_EBSD_weldbead}
	\caption{Electron backscatter diffraction image of a weld bead. Notice the variable grain sizes from top to bottom of the bead.}
	\label{SS_EBSD_weldbead}
\end{center}
\end{figure}

\chapter{In Situ High Energy X-ray Diffraction of Cold Metal Transfer Welded Ti-6Al-4V}


%%%%%%
\subsection{The martensitic phase transformation in Ti-6Al-4V}
\begin{figure*}[t]
    \centering
    \includegraphics[width=\linewidth]{/Users/njohnson/git/thesis/document/chapters/tiweldbead/tiweldbead_images/plane_spacing.pdf}
    \caption{The lattice parameter of the $\beta$ and $\alpha'$ phases from initial solidification of the weld. Initially, the $\beta$ phase lattice parameter decreases with cooling until the $\alpha'$ phase forms. At this point, the $\beta$ phase lattice parameter is constant due to constraint by the $\alpha'$ phase forming. When the $\alpha'$ phase forms its $(0001)$ plane shares a d-spacing with the $\beta (110)$. However, it rapidly relaxes to a more equilibrium d-spacing, as shown by the change in the $a$ and $c$ parameter.}
    \label{fig:latparams}
\end{figure*}

Rapid solidification of alloys has recently become a central topic in materials science and engineering due to the adoption of new advanced manufacturing techniques. Additive manufacturing and advanced robotic welding have enabled manufacturing of advanced engineering components with geometries and properties that are un-achievable with traditional casting and forging. Many of the boons of advanced manufacturing are in additive manufacturing of aerospace components where unique lightweight geometries can be created to reduce fuel consumption during flight. Ti-6Al-4V is one of the most commonly used aerospace alloys due to its high strength-to-density ratio and relatively high thermal stability. 

Ti-6Al-4V is also one of the most commonly scrutinized additive alloys because of the wide range of microstructures that can form during manufacturing. Rapid solidification of Ti-6Al-4V combined with high thermal gradients and repeated thermal cycling results in a wide range of microstructural characteristics related to grain morphology \cite{Semiatin1997, Ahmed1998, Plichta1977, Beladi2014, Katzarov2002}, alloy partitioning \cite{Szkliniarz1995, Fan2005, Boivineau2006, Sridharan2019}, phase fractions \cite{Tan2016}, and more. High variance in microstructure causes uncertainty in mechanical performance. Engineering the microstructure of additively manufactured Ti-6Al-4V is predicated on an understanding of how the microstructure forms, especially understanding transformation pathways to different phases and morphologies of grains. Ti-6Al-4V is an $\alpha/\beta$ alloy containing an HCP $\alpha$ and a BCC $\beta$ phase. The mechanisms of phase transformation and the final morphologies of grains is dependent on the thermal history a part experiences during manufacture. The rapid solidification process has been documented to produce a martensitic (diffusionless) transformation upon rapid cooling.

For the sake of further discussion, the low temperature HCP phase in Ti-6Al-4V will be referred to as the $\alpha$ phase. Any BCC phase of Ti-6Al-4V, whether at low temperature or elevated temperature, will be referred to as the $\beta$ phase. When the HCP phase is formed through a martensitic transformation it will be referred to as the $\alpha'$ phase.

Elmer and Palmer largely pioneered the study of Ti-6Al-4V phase transformations using in situ high energy X-ray diffraction. In a series of papers published in the mid 2000s they characterized the transformation from $\alpha$ to $\beta$ at relatively slow cooling rates ($1^{\circ}$ C/s - $10^{\circ}$ C/s) as well as in rapid solidification conditions ($1000^{ \circ}$ C/s or more). Several aspects of the transformation were characterized. Elmer and Palmer claim that a change in the $\beta$ lattice parameter is due to element partitioning during the formation of $\alpha$ phase. It has been well characterized that during diffusional formation of $\alpha$ from a parent $\beta$ grain, V partitions to the $\alpha$ grain boundary while Al remains trapped in the $\alpha$ phase \cite{Williams1967, Sridharan2019, Tan2015, Nandwana2019}. Elmer and Palmer claim that the movement of V to the $\beta$ phase can be observed by a shrinking of the $\beta$ lattice parameter at a temperature of around $600^\circ$ C/s \cite{Elmer2005}. This is consistent with other findings that an increase of V in BCC Ti alloys shrinks the lattice parameter \cite{Xu2015}.

Elmer and Palmer also characterized phase transformations of Ti-6Al-4V welds using both spatially and temporally resolved high energy X-ray diffraction techniques \cite{Elmer2004, Elmer2003}. Their characterization of phase transformations at different locations in welds was coupled with the Johnson-Mehl-Avrami-Kolmogorov (JMAK) equation to fit parameters such as temperature and transformation coefficient. In this series of papers they claim to observe a combination of diffusionally formed $\alpha$ phase and martensitically formed $\alpha'$ phase. The cooling rates reported in \cite{Elmer2003} indicate that they were in the martensitic phase transformation regime, however their calculation of transus temperature sometimes places the $\beta \to \alpha$ transformation below the martensite start temperature. Microscopy of the welds revealed a decrease in $\alpha'$ fraction away from the center of the weld indicating regions of slower cooling rate or different transformation temperature.

Accurate modeling of this phase transformation is important for engineering manufacturing conditions of Ti-6Al-4V that prevent $\alpha'$ formation, as well as understanding the impact of $\alpha'$ on mechanical performance. The titanium martensitic transformation that occurs during rapid solidification causes embrittlement of the material \cite{Xu2015}. Many studies report observing martensitically-formed HCP phases in rapidly cooled Ti-6Al-4V but there has been little characterization of the transformation in situ. This paper presents observations of the martensitic phase transformation as it occurs using high energy X-ray diffraction. The formation of the HCP martensitic phase can be observed as a transformation of the high temperature $\beta$ phase $(110)$ peak into the low temperature $\alpha$ phase $(0002)$ peak. 

This letter provides measurements of $\beta$ phase and $\alpha'$ phase lattice parameters during the transformation. Relaxation of the $\alpha'$ phase was observed during its initial formation, followed by further relaxation due to cooling of the material.

This study builds on the previous high energy X-ray diffraction characterization of Ti-6Al-4V work of Palmer and Elmer \cite{Elmer2003, Elmer2004, Elmer2005} as well as the work of Kenel et al. \cite{Kenel2017} and Malinov et al. \cite{Malinov2002}. In the previous investigations, phase changes from the $\beta$ phase field to $\alpha + \beta$ phase field were performed under slow cooling conditions or on weld beads large enough to have a variable cooling rate. This letter advances upon that work by comparing and contrasting phase change mechanisms under rapid cooling conditions in small volumes, i.e. a sample that only experienced a martensitic transformation.

%%%% Figure
\begin{figure*}
    \centering
    \includegraphics[width=\linewidth]{/Users/njohnson/git/thesis/document/chapters/tiweldbead/tiweldbead_images/peak_split_plot2.pdf}
    \caption{Formation of the $\alpha'$ $(0002)$ peak from the $\beta$ $(110)$ peak around $t = 2.945$s after the welder was turned off.. Upon initial formation, the $(0002)_{\alpha'}$ forms at a d-spacing near the $(110)_\beta$ peak. However, it quickly shifts to a more stable unit cell dimension. As the transformation proceeds, the intensity of the $\alpha'$ peak grows and the $\beta$ peak decreases. By the end of the experiment, $t=57.15$s when the sample had reached room temperature there were no observable $\beta$ peaks remaining.
    }
    \label{peaksplit}
\end{figure*}

Phase changes were characterized using high energy X-ray diffraction at the Advanced Photon Source, Argonne National Laboratory, beamline 1ID. Weld deposits were made using a Fronius Cold Metal Transfer welder depositing on a water cooled substrate of Ti-6Al-4V. The welder was controlled robotically and liquid depositions were made on the substrate in the path of the X-ray beam. The wire feed rate was 2.5 mm/s, the electrode current was 50 A on average, and the accelerating voltage was 12 V. Welds beads were deposited with an approximate height and diameter of 2.5 mm. Approximating the weld bead as a sphere results in a volume of 22.45 mm$^3$. 

Data collection began with the unobstructed X-ray beam shone in the direction of the detectors. Then, the CMT welder deposited liquid Ti-6Al-4V in the path of the beam. X-ray scattering occurred through the amorphous material at first, followed by X-ray diffraction as the material solidified. Variable cooling rates throughout the bulk of the sample were a concern, as it is possible that cooling rate decreases away from the water cooled substrate. Repeated measurements were made at different heights above the substrate to observe any changes in cooling rate. No cooling rate differences were observed, in contrast with similar studies performed on other alloys \cite{Brown2019}.

A Lynx ASI detector was used to characterize solidification at a rate of 200 Hz. The ASI detector had a $d$-spacing coverage of 1.58-2.6 $\textup{\AA}$, allowing observation of the $(1000)_\alpha$,$(0002)_\alpha$ $(10\bar{1}1)_\alpha$, $(10\bar{1}2)_\alpha$, $(110)_\beta$ and $(200)_\beta$ peaks. All possible peaks were rarely observed simultaneously due to the orientation of grains and the low number of grains in the sample. The detector had an azimuthal coverage of about 15$^\circ$. The diffraction images were integrated and analyzed using the General Structure Analysis Software II (GSASII) \cite{GSASII}. Single peak fits were performed on the $(1000)_\alpha$ and $(0002)_\alpha$ peaks to get $\alpha$ lattice parameters. 

The ASI detector had too small of an azimuthal/d-spacing coverage to ensure good statistics for phase fraction calculations. A 2-dimensional GE 41RT detector was used simultaneously for larger signal coverage, but with a lower refresh rate of 10 Hz. The GE detector was a 2048 pixel by 2048 pixel (200 $\times$ 200 $\mu$m$^2$ pixel size) area detector with a d-spacing range of 0.66 - 3.31 $\textup{\AA}$ and an azimuthal coverage of nearly $180^\circ$. The GE images were integrated over the full $d$-spacing and azimuthal range. Rietveld Refinement was performed to calculate phase fraction during solidification. In all cases, the final $\beta$ phase fraction was zero or close to zero within the error bars of the Rietveld refinement fit. The lattice parameter measurements performed using the ASI detector were compared to a Rietveld refinement fit on the GE detector. Both fits produced comparable values within each other's range of uncertainty.

Volume-averaging effects need to be taken into account because the experiment operated in transmission mode. Transmission X-ray diffraction produces data from all locations in the sample meeting the Bragg condition. This means that scattering/diffraction signals will be simultaneously measured for volumes near the surface as well as in the bulk of the sample. If different phenomenon are occurring at different depths from the sample surface -- such as differences in transformation temperature, differences in cooling rate, etc. -- then the signal observed will contain information from all these locations; however, the signals overlap and obscure the behavior of individual grains.

Furthermore, Rietveld refinement assumes a certain crystal model and cannot always account for anisotropies in the results. For example, if the $\alpha$ phase has different coefficients of thermal expansion for the $a$ axis and $c$ axis then the $\left(1000\right)$ and $\left(0001\right)$ peaks will move in $d$ spacing at different rates.

Along with thermal anisotropies, full integration of the images obscures the behavior of different grains. Two grains may have lattice planes in the diffraction condition but at different locations in the sample. This likely means that spots will appear near the same $d$ spacing range but at a different azimuthal angle. If the internal temperature of the two grains is different, they will be at slightly different $d$ spacing values. Integrating over the azimuth will produce a wide peak since it is taking the sum of two peaks that almost, but not exactly, overlap.

To account for both anisotropy in the thermal expansion and thermal gradients within the sample, the ASI images were integrated over regions corresponding to individual grains. Individual grains were identified by individual spots which appeared on the detector. Figure \ref{peaksplit} demonstrates one such region in the first row. Each image is actually a subregion of a larger detector image which showed multiple spots. The histograms in the second row were produced by only integrating over this part of the image. Single peak fits were made for these histograms to monitor the behavior of individual grains.

\begin{figure}
    \centering
    \includegraphics[width=1\linewidth]{/Users/njohnson/git/thesis/document/chapters/tiweldbead/tiweldbead_images/APL_figure_microscopy.pdf}
    \caption{Optical and scanning electron microscopy of the weld bead characterized. Optical microscopy in (a) and (b) show the fine, acicular (needlelike) structure of $\alpha'$ grains contained within the former $\beta$ grain boundary. }
    \label{microscopy}
\end{figure} 

Of particular interest to this study was observing the mechanism of phase change from high temperature to low temperature because it impacts grain morphology and, therefore, the mechanical performance of the material. 

Figure \ref{fig:latparams} shows the d-spacing of $\left(120\right)_\beta$, $\left(1000\right)_\alpha$, and $\left(0001\right)_\alpha$. After the $\beta$ phase formed from solidification of the liquid melt its lattice parameter decreased due to cooling. Upon formation of the $\alpha'$ phase, however, the $\left(110\right)_\beta$ plane exhibited a knee in its $d$-spacing.  The $\left(0001\right)_\alpha$ initially forms with a d-spacing near that of the $(110)_{\beta}$ peak. However, the geometry of the $\alpha'$ unit cell quickly shifts away from the $(110)_\beta$ equilibrium value to include a longer $c$ axis and shorter $a$ axis. This is likely a more stable configuration for the $\alpha'$ unit cell. 

During this shift, the $\beta$ lattice parameter is constant. As mentioned previously, $\beta$ Ti-6Al-4V has demonstrated a decrease in lattice parameter when its $V$ content increases. Thus, if diffusion of $V$ into the $\beta$ unit cell is occurring there must be a separate external force acting against unit cell contraction. It is possible that the $\beta$ unit cell is kept at a constant volume by changes in the $\alpha'$ unit cell which shares a boundary according to the $(110)_\beta||(0001)_\alpha$ orientation relationship. This orientation relationship is demonstrated in Figure \ref{peaksplit}, where the $(0002)_\alpha$ peak can be seen splitting off of the $(110)_\beta$ peak.

However, the authors believe that diffusion did not occur, as evidenced by the lack of $\beta$ phase at room temperature and the lack of a shrinkage in $\beta$ lattice parameter upon formation of the $\alpha$ phase. Rather, $\beta$ and $\alpha$ coexistence occurs because different parts of the sample are experiencing a martensitic transformation at different times. The $\beta$ phase can coexists with $\alpha$ inside a prior $\beta$ grain as the shear transformation progresses throughout the prior $\beta$ crystal.

To further confirm the hypothesis that the solidified material was entirely martensitic $\alpha'$, optical microscopy and electron beam backscatter detection (EBSD) were used to characterize the microstructure. These images can be seen in Figure \ref{microscopy}. Martensitic HCP Titanium is often characterized by its morphology: long, acicular (needlelike) grains. Optical microscopy in subpanels (a) and (b) of Figure \ref{microscopy} reveal the needlelike structure inside of prior $\beta$ grains. While there is a transition to equiaxed $\alpha$ grains near the substrate, diffraction images were not taken here.

Scanning electron microscopy reveals a grain size of roughly $1-2 \mu$m in width and $5-10 \mu$m in length. A low confidence index was achieved during microscopy, which can be due to a wide variety of reasons including a large amount of strain on the $\alpha'$ lattice which would cause a poor fit with a structure match of equilibrium HCP Ti-6Al-4V. EBSD was also used to index the crystal structures present in the weld beads. No $\beta$ phase was identified with any reasonable confidence index in the samples analyzed, further indicating that it was not present in the regions examined. The lack of $\beta$ at room temperature in repeated HEXRD measurements combined with its lack of presence in EBSD scans indicates that the structure may be entirely martensitic $\alpha'$, with no diffusional transformation to $\beta$ possible.

The $\beta \to \alpha$ transition measured is different than that typically published for martenstitic transformations of $\alpha/\beta$ alloys. The mechanism of transformation from BCC to HCP by a shear transformation was first characterized by Burgers in 1934 \cite{Burgers1934}. Normally, only the shear transformation along the $\left[111\right]$ direction is described without the secondary relaxation along the $a$ and $c$ axes. What we have observed constitutes a two part unit cell transformation; first, the BCC unit cell shears into the HCP unit cell in a martensitic transformation. Then, the HCP unit cell distorts. The reason for the secondary HCP unit cell distortion is unknown, but conjectures about its origin can be made.

Upon initial formation, the HCP unit cell has a c/a ratio of 1.566. This is far from the ideal close packed hexagonal c/a ratio of $\sqrt{8/3} \approx 1.63$. The HCP unit cell undergoes an extension along the $c$ direction and shrinkage along the $a$ direction, distorting the unit cell to a $c/a$ ratio of 1.59, closer to the ideal ratio. Discussing the equilibrium $c/a$ ratio for Ti-6Al-4V can be a somewhat confusing because the $c$ and $a$ lattice parameters depend on a multitude of factors including composition of the $\alpha$ phase, grain morphology, and residual stresses in the sample. Thus, it makes more sense to compare the $c/a$ ratio observed to $c/a$ ratios measured in samples of Ti-6Al-4V manufactured other ways. Stapleton et al. used HEXRD to measure $c/a = 1.597$ for forged Ti-6Al-4V \cite{Stapleton2008}. Tan characterized Ti-6Al-4V manufactured through electron beam melting and found $c/a = 1.595$ \cite{Tan2015}. Xu et al. published a wide variety of $c/a$ ratios depending on the $V$ concentration of the $\beta$ phase. They found values ranging from $c/a = 1.558$ for V-rich HCP phases and $c/a=1.6$ for V lean HCP phases. Since no $\beta$ phase was present at room temperature for these samples it is impossible to comment on the $V$ content of the two phases based on lattice parameter alone. 

High energy X-ray diffraction revealed lattice parameter changes in the martensitic formation of HCP $\alpha'$ Ti-6Al-4V. When the transformation occurs the $\alpha'$ phase initially forms with a $(0001)_{\alpha'}$ d-spacing equal to the $(110)_{\beta}$ d-spacing. However, the unit cell quickly shifts to a more equilibrium geometry, with an extended $c$ axis and shrunk $a$ axis. Following this initial shift both phases continue to shrink in volume due to cooling. This observation has implications for modeling phase transformations of Ti-6Al-4V during additive manufacturing. In particular, it reveals that the microstructure of rapidly solidified small volumes is entirely martensitic. It also indicates that constrains on the $\beta$ unit cell may be due to the $(0001)_{\alpha}||(110)_\beta$ orientation relationship. 



\subsection{\textit{in situ} Characterization of Ti-6Al-4V Welds}
\subsubsection{High Energy X-ray Diffraction Characterization of Single Ti-6Al-4V Weld Beads}
\begin{figure}
	\includegraphics[width=1\linewidth]{/Users/njohnson/git/thesis/document/chapters/tiweldbead/tiweldbead_images/all_alpha_a_lat}
	\caption{All $a$ lattice parameters for the HCP $\alpha$ phase. Time is taken from the point that the heat source was turned off.}
	\label{all_alpha_a_lat}
\end{figure}
%
\begin{figure}
	\includegraphics[width=1\linewidth]{/Users/njohnson/git/thesis/document/chapters/tiweldbead/tiweldbead_images/all_alpha_c_lat}
	\caption{All $c$ lattice parameters for the HCP $\alpha$ phase. Time is taken from the point that the heat source was turned off.}
	\label{all_alpha_c_lat}
\end{figure}
%
\begin{figure}
	\includegraphics[width=1\linewidth]{/Users/njohnson/git/thesis/document/chapters/tiweldbead/tiweldbead_images/all_beta_lat}
	\caption{All $a$ lattice parameters for the BCC $\beta$ phase. Time is taken from the point that the heat source was turned off.}
	\label{all_beta_lat}
\end{figure}
%
\begin{figure}
	\includegraphics[width=1\linewidth]{/Users/njohnson/git/thesis/document/chapters/tiweldbead/tiweldbead_images/all_ca_ratios}
	\caption{The ratio $c/a$ for the HCP $\alpha$ phase from the point of the $\alpha$ peaks first appearing.}
	\label{all_ca_ratios}
\end{figure}

What do I want to say about the Ti-6Al-4V weld bead experiments?

\subsubsection{\textit{in situ} High Energy X-ray Diffraction Characterization of Single and Double Weld Tracks}
\begin{figure}
	\includegraphics[width=1\linewidth]{/Users/njohnson/git/thesis/document/chapters/tiweldbead/tiweldbead_images/weldtrack_measurement_locations}
	\caption{}
	\label{weldtrack_measurement_locations}
\end{figure}
%
\begin{figure}
	\includegraphics[width=1\linewidth]{/Users/njohnson/git/thesis/document/chapters/tiweldbead/tiweldbead_images/beam_distance}
	\caption{}
	\label{beam_distance}
\end{figure}

\subsection{Differences between point depositions and line welds}
Talk about the observed differences in results between the two deposition geometries. What does this indicate about differences in the processes? Why are they different? What implications does this have for the modeling/research of AM Ti-6Al-4V?

%%%%%
\subsection{CHESS samples}
A build wall of Ti-6Al-4V was manufactured by a Sciaky E-beam DED process. Multiple build layers were manufactured on top of one another to form a build wall of about \textbf{how many?} layers. The sample was characterized using HEXRD and TEM to discover differences in the microstructure from the top to the bottom of the wall.

\textbf{This data is stored on one of your work computers, you need to grab it from there.}


\chapter{Mechanical Behavior and Phase Transformations of Additively Manufactured NiTi Elastocaloric Cooling Materials}

\chapter{High Energy X-ray Diffraction Characterization of Additively Manufactured Ti-5553 Octet Truss Lattices}
%~~~~~~~~~~~~~~
\subsection{Introduction}
Continuous metal micro lattice structures have been theorized and described since the 1990s. These structures are able to provide specifically engineered mechanical properties, like stiffness and strength, at a fraction of the density of bulk metals. Early adoptions of the lattice structure include the related open and closed cellular foam materials \cite{Bonatti2017} as well as the snap-fit and weld joined lattices \cite{Dong2015}. 

In 2001, Deshpande, Fleck, and Ashby (DFA) presented a continuum theory for one type of lattices structure: the octet truss \cite{Deshpande2001}. Octet truss lattice structures are analogous to face centered cubic crystal structures, where the nodes of the lattice correspond to atom sites. The DFA model provides predictions for macroscopic stress and strain in the trusses, as well as calculations of important mechanical properties like yield strength, buckling strength, and failure modes. In particular, the DFA model predicts that the mechanical deformation in octet trusses should be stretch-dominated first, with buckling occurring as a secondary mechanism. While the DFA model is robust, the technology at the time of its introduction was limited to characterize such complex structures. Advancements in imaging analysis and computing have made characterization of these structures possible. Furthermore, advancements in additive manufacturing (AM) have allowed for these complex structures to be manufactured with little machining and for arbitrarily large and complex geometries to be built.

The combination of a unique geometry with the octet truss and the additive manufacturing process requires multiple approaches for characterization. The geometry of the octet truss means that models like the DFA model can be used to assess mechanical behavior, specifically failure of the part. These continuum models can be compared against computational models using finite element analysis or against macroscopic stress-strain response observed. However, the additive manufacturing process adds another complication into the mix. The additive process is known to create unique microstructures. Additive manufacturing impacts grain morphologies \cite{Tan2015, Zhu2018}, dislocation density \cite{Zhang2015, Wang2017, Gallmeyer2020} crystallographic texture \cite{Wang2019}, phase fractions \cite{Gallmeyer2020}, and defect structures \cite{Yang2017,Matthews2016}, all of which impact the material structure-property relationship and therefore the mechanical performance. 

Several studies have performed mechanical testing or finite element modeling of octet trusses and, in some cases, both. Dong et al, examined the macroscopic response of snap-fit octet truss lattices and compared the results to the DFA model, finding that snap-fit octet truss lattices moduli are well predicted by the DFA model \cite{Dong2015}. Latture, Begley, and Zok evaluated the mechanical performance of ideal lattice structures with octet trusses being one amongst many studied \cite{Latture2017}. Bonatti and Mohr looked at macroscopic mechanical measurements of large (5x5x5 unit cell) octet truss structures and characterized the failure mechanisms \cite{Bonatti2017}. Tancogne-Dejean did a comprehensive study that compared macroscopic stress-strain results for an octet truss to finite element modeling \cite{Tancogne-Dejean2016}. Their approach examined different unit cell densities, strut thicknesses, and strut shapes and the impact on mechanical response. Tancogne-Dejean was able to achieve decent agreement in failure modes for different densities between a finite element model and additively manufactured samples. 

Other studies have examined how additive manufacturing impacts lattice materials. Yan et al. investigated additively manufactured gyroid structures, a different kind of period lattice material \cite{Yan2012}. Yan was primarily concerned with the manufacturability of the structure and its impact on surface roughness, shape, and deformation. They found that some cracking occurred in the samples during printing due to residual stress buildup. Tancogne-Dejean used selective laser melting to build and assess the macroscopic response of octet truss samples \cite{Tancogne-Dejean2016}. They found that changes in morphology and texture of the sample can result in mechanical property differences up to 20\%. Liu et al. studied the role of geometric imperfections of the struts due to the SLM process on octet truss lattices \cite{Liu2017}. They counted a statistical distribution of strut deformities based on how far the strut deviated from a perfect circle. These distributions of imperfections were then introduced into a finite element model which varied the beam width and thickness in order to capture imperfections. They found a decrease in Young's modulus in all directions along the struts as a result of the imperfections. This was linked back to the manufacturing process by the realization that struts in the orientation of the build plane were oversized.

These studies are all concerned with comparing continuum level models of octet trusses to macroscopic responses in the parts. What is often unclear is how the behavior of an individual unit cell impacts the macroscopic behavior of the samples. To this end, several authors have implemented homogenization theories which connect unit-cell-level mechanical properties to macroscopic mechanical properties. Park et al. developed a homogenization scheme which takes into account deformations at the unit cell scale due to build processes in fused deposition modeling and predicts mechanical performance at the macroscale \cite{Park2014}. Vigliotti et al. used the representative volume element approach to develop a homogenization scheme which linked discrete elements (unit cells) to the macroscopic behavior of arbitrarily shaped lattice structures \cite{Vigliotti2014}. Mohr used a similar approach which treated the macroscopic part as a homogenous medium and calculated mechanical properties based on the mechanics of unit cells \cite{Mohr2005}. In all cases, however, the unit cell is treated as a  a perfect material.

The success of homogenization schemes is built upon having accurate properties for the representative volume elements. Consideration of the additive manufacturing process including deformation, defects, microstructure, texture, and more must be made for models to accurately predict macroscopic deformation behavior of lattice materials. To this end, it is beneficial to examine how additive manufacturing impacts an individual unit cell.

In this case, samples were manufactured out of Ti-5V-5Mo-5Hf-3Cr (Ti-5553). Some studies have looked at the impact of additive manufacturing on Ti-5553 specifically, such as Schwab et al. \cite{Schwab2016}. Characterization of the phases resulting from the AM process will be especially important because Ti-5553 is known to have unexpected secondary phases form \cite{Dehghan-Manshadi2011, Zheng2016}. Post-processing heat treatments will be equally important to characterize, as the heat treatment schedule can significantly vary mechanical properties like Young's modulus and yield strength \cite{Kar2014}.

This study is unique because it combines all of the above characterizations into a single picture to evaluate how the additive manufacturing process impacts the expected mechanical performance of two unit cells of a Ti-5553 octet truss lattice structure. Based on the nomenclature of Latture, Begley, and Zok \cite{Latture2019} the sample geometry is $\{2\text{FCC}\}^1$ The present investigation combines high energy X-ray diffraction, digital image correlation, finite element modeling, and fractography to evaluate the mechanical performance, texture, strain distribution, failure mechanisms, and residual stresses of the parts. 

The study herein takes a processing-structure-property approach to characterize the mechanical response of individual unit cells of additively manufactured Ti-5553 octet truss lattices. While previous studies have evaluated macroscopic behavior of lattices and compared them with continuum level predictions of mechanical properties, no study has yet investigated individual unit cells. The measurement and characterization of individual unit cells is important for verifying the many homogenization schemes which exist for these parts.

%~~~~~~~~~~~~~~~~~~~~~~~~~~~
\subsection{Characterization Techniques}

\begin{figure}[b]
	\includegraphics[width=1\linewidth]{/Users/njohnson/git/thesis/document/chapters/single_lattice/figures/build_direction}
	\caption{The orientation of the part on the build substrate relative to the build direction and recoater blade direction.}
	\label{build_direction}
\end{figure}

\subsubsection{Printing Parameters and Heat Treatment}
Samples were printed on an SLM Solutions SLM 280 printer. Two different sets of scan parameters were used for the bulk and strut portions of the part. For the strut portion, a scan speed of 800 mm/s was used with a laser power of 100W and a hatch spacing of 0.12 mm. For the bulk portion of the sample, a scan speed of 725 mm/s was used with a laser power of 175W and a hatch spacing of 0.12 mm. The bulk portion also had a skin layer applied with a laser scan speed of 525 mm/s and a laser power of 100W. 

The samples were subjected to ultrasonic cleaning after manufacture, then heat treated at 300$^\circ$C for 1 hour (within 16-17$^\circ$C per minute) in vacuum (10$^{-5}$ mbar) while on the build substrate. After 1hr samples were cooled in an Argon atmosphere to 90$^\circ$C.

The orientation of the part on the build substrate can be seen in Figure \ref{build_direction}. It is important to note the orientation of truss members relative to the build and heat flow directions. Some of the trusses are oriented flat on the substrate and should experience even heating and cooling cycles during building. Other members are extended at a 45$^\circ$ angle in the build direction. These extended members will experience many heating and cooling cycles during the build process. This behavior is known to cause residual stress buildup in the sample \cite{Ganeriwala2019}.

\begin{figure}[b]
	\includegraphics[width=1\linewidth]{/Users/njohnson/git/thesis/document/chapters/single_lattice/figures/crosshead_displacement}
	\caption{Load frame crosshead displacement and associated measured force. Each drop in force corresponds to a fracture or buckling in the sample. Diffraction data was collected through 380$\mu$m of displacement.}
	\label{loaddisp}
\end{figure}


%~
\subsubsection{High Energy X-ray Diffraction}
\paragraph{Experimental Setup}
\begin{figure*}[t]
	\includegraphics[width=1\linewidth]{/Users/njohnson/git/thesis/document/chapters/single_lattice/figures/exp_setup.png}
	\caption{The orientation of the part in the laboratory coordinate frame relative to the X-ray beam and the detector.}
	\label{expsetup}
\end{figure*}

%
\begin{figure}
	\includegraphics[width=1\linewidth]{/Users/njohnson/git/thesis/document/chapters/single_lattice/figures/grayscale_microstructure}
	\caption{SEM micrograph of the microstructure of a bulk piece of additively manufactured Ti-5553. Several small pores are highlighted by red circles.}
	\label{microstructure}
\end{figure}
%~

Samples were loaded in a custom load frame developed by the Advanced Photon Source and collaborators \cite{Loadframe}. Samples were loaded quasi-statically in increments of $25 \mu$m through elastic loading. Once buckling and fracture occurred samples were loaded quasi-statically until further buckling and fracture occurred. The load frame crosshead displacement and measured force can be seen in Figure \ref{loaddisp}.

High-energy x-ray diffraction experiments were conducted at the 1-ID beamline of the Advanced Photon Source, Argonne National Laboratory. A diagram of the experimental setup used can be seen in Figure \ref{expsetup}. A 71.6 keV beam was used for transmission X-ray diffraction. All data was collected on a Dexela detector with a 78$\mu$m resolution and pixel dimensions of 3888 $\times$ 3072 pixels. Samples were exposed for $0.1$s at each location. The beam was scanned along struts in steps of roughly 40 $\mu$m. The size of the beam relative to a strut can be seen in the upper right hand corner of Figure \ref{expsetup}.%Note: that's 75 images per strut (150 for the transverse struts) with a strut length of 3.5mm

The microstructure of the sample can be seen in Figure \ref{microstructure}, demonstrating a fairly homogenous distribution of grains whose sizes range from tens of microns to around $75\mu$m in diameter. This gives reason to suggest that good grain statistics were observed in the diffraction pattern as the beam was being scanned along a fairly large number of grains at once.

Due to sample deformation during loading the movement of the struts had to be tracked to ensure that scans were taken on the strut. Each node location was tracked and recorded using computed tomography after each load step. The coordinates of the nodes were then given to an algorithm that computed the straight-line path from node to node. This output of this algorithm was then used to perform fly scanning across the struts.

\paragraph{Explanation of Plots}
\begin{figure}
	\includegraphics[width=1\linewidth]{/Users/njohnson/git/thesis/document/chapters/single_lattice/figures/location_explanation}
	\caption{Locations on the sample that were measured and a corresponding 2D plot of measurements locations. This type of plot will be used throughout the paper.}
	\label{location_explanation}
\end{figure}
Due to the complicated geometry of the sample being studied, in combination with the 3-dimensional nature of the measurements made, it is necessary to explain how plots of data have been constructed. Figure \ref{location_explanation} shows the locations measured using high energy X-ray diffraction colored in red, yellow, and green. For the sake of display, the measurements at these locations were projected onto 2-dimensional plots. An example of one such plot is also shown in Figure \ref{location_explanation}b. It is important to note that the two horizontal lines shown in Figure \ref{location_explanation}b are not one continuous strut, but rather two struts that sit perpendicular to one another.

Certain locations on the sample contained multiple struts that overlapped each other in the beam direction. If the beam hit multiple struts at a time then it became impossible to differentiate the signals of each, causing significant beam broadening and uncertainty in diffraction peak location. To overcome this, data was discarded for locations that featured multiple diffracting struts at once. The locations where data was discarded are shown as black sections on Figure \ref{location_explanation}.

%~
\paragraph{Stress and Strain Calculations}
Each diffraction image was binned into 24 bins consisting of 15$^\circ$ of the full diffraction ring around the coordinate $\eta$, which is the angle between $y$ and $x$ in Figure \ref{expsetup}. After binning, the bins were integrated to produce a diffraction histogram for each angle.

The individual $\eta$ direction histograms were integrated and refined with Rietveld Refinement using GSAS \cite{GSAS} and the SMARTSware routine \cite{Smartsware2004}. This refinement produced a lattice parameter for each eta direction and load step $a_{\eta}^{\ell}$ where $\eta = 0^\circ, 15^\circ,...345^\circ$ is the integration direction and $\ell = 0 \mu\text{m}, 25\mu\text{m}, 50 \mu\text{m}, ... , 380 \mu\text{m}$ is the crosshead displacement in compression at each load step.

The choice of datum used was the averaged lattice parameter at zero load, or
\begin{equation}
	\bar a^0 = \frac{1}{24}\sum_\eta a_{\eta}^{0}.
	\label{datum}
\end{equation}
The choice of datum is important and has several effects on the results. Most importantly, if there is a residual stress on the sample then taking the averaged lattice parameter at zero load will remove distortion of the lattice parameter in certain $\eta$ orientations that would otherwise exist. 

Once the datum was found then a strain at each load step $\ell$ and orientation $\eta$ was computed as
\begin{equation}
	\epsilon_\eta^\ell = \frac{\bar a^0 - a_\eta^\ell}{\bar a^0}.
	\label{strain}
\end{equation}

One important aspect of this investigation was to find the value and orientation of \textit{principal strains} in the sample. The maximum tensile $\epsilon_{11}$ and compressive $\epsilon_{22}$ strains give information about where the maximum strains are building up on the sample while the orientation of the principal strain coordinate system $\psi$ give information about whether the strains are along the struts or at an off-axis angle. This can reveal if the AM microstructure of the sample is having an impact on the loading behavior of the sample. The principal strains were fit using the model of I.C. Noyan and and J.B. Cohen \cite{Noyan1987}, which has been used in similar investigations \cite{Brown2019}, given by
\begin{equation}
	\epsilon_{\eta}^{\ell} = \epsilon_{11}^{\ell}\sin^2{\left( \eta + \psi\right)} + \epsilon_{22}^{\ell}\cos^2{\left(\eta+\psi\right)}
	\label{strainmodel}
\end{equation}
where $\psi$ is the orientation, in degrees, of the principal coordinate system relative to $\eta = 0^\circ$ in the test coordinate system.

\begin{figure*}[t]
	\includegraphics[width=01\linewidth]{/Users/njohnson/git/thesis/document/chapters/single_lattice/figures/voids.png}
	\caption{Voids in the sample after heat treatment and after compression through multiple strut failures.}
	\label{voids}
\end{figure*}

The full model of Noyan and Cohen includes shear terms and a third strain component $\epsilon_{33}$ but in the current experiment only two strain components could be measured at a single time. In order to calculate a stress, however, assumptions about the third strain component had to be made. Calculation of stress is desirable because it allows measurement of mechanical values like yield strength, ultimate tensile strength, shear modulus, and more. These measured values can then be compared to the values predicted by the DFA model.

The third strain component was set as equal to either the $\epsilon_{11}$ or $\epsilon_{22}$ strain depending on the location of the strain on the sample. In the horizontal struts it is assumed that the primary tensile direction $\epsilon_{11}$ is along the strut direction; for horizontal members this is along the $x$ direction in the laboratory coordinate system; for other members, this is $45^\circ$ away from the $x$ axis toward the $y$ axis. Therefore, by Poisson's ratio, the other two strains should be compressive. The opposite was assumed for all remaining struts: the third strain component $\epsilon_{33}$ was set equal to the primary tensile strains.

In order to compute stress in the sample it was first necessary to compute certain material constants including the elastic modulus $E$ and Poisson's ratio $\nu$. A sample of known dimensions was cut from bulk Ti-5553 that had been manufactured using the same print parameters as the octet truss sample. The sample was loaded through the elastic region in compression and the material constants were found. A Young's modulus of $E_s = 107.02$ GPa and Poisson's ratio of $\nu = 0.34$ were measured. This is slightly higher than the values found by other investigations, which normally fell in the range of $\sim$ 90-100 GPa \cite{Clement2010}, although these measurements were found for alloys which underwent slightly different heat treatments and were measured using more traditional mechanical characterization.

The effective modulus of the lattice $E_L$ is computed as
\begin{equation}
	E_L = E_s \frac{\rho}{9}
	\label{effmod}
\end{equation}
where $E_s$ is the Young's modulus of the base material and $\rho$ is the density of the solid as calculated from the DFA model. In this case samples of $20\%$ density were manufactured making the effective modulus of the sample $E_L = 2.38$ GPa.

Once the material constants were computed from the bulk sample, the stress was calculated using
\begin{equation}
	\sigma_{ii}^\ell = \frac{E_s}{1+\nu}\epsilon_{ii}^\ell + \frac{\nu E_s}{(1+\nu)(1-2\nu)}\text{tr}\left(\mathbf{\epsilon^\ell}\right)
	\label{sigma}
\end{equation}
where $\text{tr}\left(\mathbf{\epsilon}\right)$ is the trace of the strain tensor computed from Eqn. \ref{strainmodel}. 

%~
\subsection{Finite Element Modeling}
Ansys Mechanical 2020 was used to conduct a FEA simulation to predict the residual stress within the unit cell resulting from the SLM process. The model used was a 1-way coupled non-linear thermal-structural analysis, using a layer-wise approximation of the SLM process. The mesh for this simulation used 0.15mm size quadratic hexahedral elements constrained to a cartesian grid aligned with the build direction of the part (commonly known as a voxel mesh). Each layer of elements (representing multiple physical powder layers) were activated at the melting temperature of Ti-5553 and allowed to cool, thereby inducing residual stress within the component. The layer-wise approximation of the SLM build process sacrifices the detail of local anisotropy due to the laser scan path, but allows for full temperature-dependent non-linear material models to be used whilst running on a multi-core workstation computer.

Two compression simulations were constructed using the same geometry, but utilising a finer, conformal hexahedral mesh. Whilst the geometry was shared between the SLM and compression simulations, measurement paths were defined within the body in order to extract strain results in locations and coordinate systems identical to those measured on the physical test specimen.

The elemental stress results from the SLM simulation were imported and interpolated to initialise one of the compression simulations, in order to validate the hypothesis that residual stress plays a role in the anisotropy of the load distribution within the sample. Strain results from both compression simulations were exported and visualised within MATLAB to perform a qualitative comparison between simulations with and without residual stress applied.


\begin{figure}
		\includegraphics[width=1\linewidth]{/Users/njohnson/git/thesis/document/chapters/single_lattice/figures/001PF_Ti5553}
		\caption{Pole figure of the (001) plane normal in a bulk (node) part of the specimen.}
		\label{P3}
\end{figure}
	%
\begin{figure}
		\includegraphics[width=1\linewidth]{/Users/njohnson/git/thesis/document/chapters/single_lattice/figures/P3_001PF_Ti5553}
		\caption{Pole figure of the (001) plane normal along the strut direction for one of the horizontal (transverse) struts passing through the center node of the sample.}
		\label{strut}
	\label{texture}
\end{figure}



\begin{figure*}
	\includegraphics[width=1\linewidth]{/Users/njohnson/git/thesis/document/chapters/single_lattice/figures/DIC_e11}
	\caption{The $\epsilon_{11}$ principal strains on the surface of the sample calculated from digital image correlation. The load frame crosshead displacement is shown for each image.}
	\label{e11}
\end{figure*}
\begin{figure*}
	\includegraphics[width=1\linewidth]{/Users/njohnson/git/thesis/document/chapters/single_lattice/figures/DIC_e22}
	\caption{The $\epsilon_{22}$ principal strains on the surface of the sample calculated from digital image correlation. The load frame crosshead displacement is shown for each image.}
	\label{e22}
\end{figure*}

%~
\subsubsection{Ex Situ Characterization}
Characterization of the sample was performed both pre- and post-mortem to identify the impact of additive manufacturing on the microstructure and to observe changes in that microstructure, respectively. A wide suite of ex situ characterization techniques was used.


\paragraph{X-ray Computed Tomography}


X-ray computed tomography was performed on a Zeiss Xradia 510. The isotropic voxel width was 7$\mu$m. The smallest detectable defect size was 16 voxels.

The pre-mortem and post-mortem samples can be seen in Figure \ref{voids}. Large voids existed in bulk regions of the sample such as in the nodes or in the build substrate. The void size decreased significantly in the struts. After compression, however, many more large voids nucleated within the sample. There are two possible reasons for the appearance of these voids. Either a) the voids were already present but under the voxel threshold for the detection algorithm; they grew during compression due to dislocation coalescence or b) compression of the sample caused delamination of the additive layers, leading to void nucleation.

An example microstructure from one sample can be seen in Figure \ref{microstructure}, featuring several pores that are smaller than the minimum detectable voxel size used by the XCT software. This gives good reason to suspect that the voids were already present in the material and grew in size during loading.


\paragraph{Texture Mapping}
The heat transfer through a bulk piece of Ti-5553 and a thin lattice strut, oriented at 45$^\circ$ relative to the build substrate, may be different. Due to the geometry of the samples it was not possible to rotate the struts to obtain multiple diffraction patterns at different orientations. Therefore texture maps were obtained using a single diffraction image. 

Texture was fit using spherical harmonic functions in GSAS II. No assumptions about texture symmetry were made. Once the spherical harmonic coefficients were fit, the data was exported from GSASII and imported into the MTEX software for Matlab. MTEX was used to generate an orientation distribution function, which can be seen in Figure \ref{texture}. Texture maps were taken for both a strut in the sample as well as through a bulk piece of the sample. The bulk exhibits a weak fiber texture in the build direction, shown in Figure \ref{P3}, which is in agreement with previously characterized textures for additively manufactured Ti-5553 \cite{Schwab2016}. Likewise, the strut also showed a weak fiber texture, Figure \ref{strut}, though the texture is concentrated into several poles instead of being dispersed like the texture of Figure \ref{P3}.


%~
\paragraph{Digital Image Correlation}
Digital image correlation was performed on samples using Correlated Solution's VIC 2D software. Samples were lit during compression using a white light source; no surface modifications, such as speckling, were used. The software computed the 2D deformation using a subcell size of 23 pixels. The crosshead displacement can be seen, in $\mu$m, listed at the top of each load step image in Figures \ref{e11} and \ref{e22}.

The magnitude of the $\epsilon_{22}$ strains are higher than the $\epsilon_{11}$ strains which is to be expected because the $\epsilon_{22}$ strains are in the compression direction. The majority of the $\epsilon_{11}$ strains are in the nodes of the sample. As the transverse struts begin to stretch they push the central nodes outwards causing a significant buildup of strain. Equal magnitude strains can also be seen in the nodes and horizontal members in Figure \ref{e11}. By comparison, the $\epsilon_{22}$ strains build up mostly in the vertical members near the central nodes. These members are in the direction of the load and therefore receive the highest magnitude strains.

The biggest takeaway from the DIC analysis is that the 45$^\circ$ members experience a complex mix of strains. Neither the $\epsilon_{11}$ nor the $\epsilon_{22}$ completely capture the behavior of these struts. Both tensile and compressive strains can be observed. This motivates the use of multiple coordinate systems for exploring the behavior of these complex stress states, as discussed in following sections.
%~~~~~~~~~~~
\section{Results}
\subsection{Stress and Strain Measurements with HEXRD}
The strain in the sample at 50$\mu$m of crosshead displacement can be seen in Figure \ref{50um_principal_strains}. The strains are shown in three different coordinate systems: the laboratory coordinate system, the strut coordinate system, and the principal coordinate system. The laboratory coordinate system is the $(x,y,z)$ coordinate system shown in Figure \ref{expsetup}. The strut coordinate system are the strains oriented along (parallel to) the strut direction. The principal coordinate system is the $\epsilon_{11}$ strains calculated from Equation \ref{strainmodel} or the principal tensile strains. Figure \ref{50um_principal_strains}c also has arrows showing the orientation of the principal strain direction $\psi$ relative to the $x$ axis in the laboratory coordinate frame. Some tensile strains can be seen building up in the transverse members in all coordinate frames, with a very slight compressive strain building up in the 45$^\circ$ members. The principal strain direction in the transverse struts is mostly down the strut direction with some slightly off-axis behavior. The principal strain direction in the $45^\circ$ members is more complex. The principal strain direction is typically either down the strut direction or $90^\circ$ to it. 

By comparison, the distribution of stress and strain across the sample at peak elastic load, before fracture occurred, can be seen in Figure \ref{200um_principal_strains}. 

In every reference frame the horizontal members are dominated by tensile strains which is to be expected from struts that are transverse to the load and therefore should be in pure stretch. However, the struts sitting at 45$^\circ$ to the loading axis demonstrate a noticeable anisotropy in the distribution of load, particularly in Figure \ref{200um_principal_strains}a. In particular, the central node shows a fairly high compressive strain in the strut to its left, while the same strain is missing from the struts to the right. In the struts to the right there is a buildup of tensile strains near the center of the strut, which is likewise missing from the mirroring strut on the left side of the sample.

Figure \ref{200um_principal_strains}c shows the $\epsilon_{11}$ strain as well as the orientation of $\epsilon_{11}$ relative to $0^\circ$ in the laboratory coordinate system. For the transverse struts the orientation of the strain is entirely down the strut direction, with a few locations slightly off-axis. This is to be expected for a member in stretch. 

For the $45^\circ$ members, however, the principal tensile direction changes from location to location. On the right half of the sample the tensile strains are primarily in the strut direction, while for the left side of the sample they vary from being in the strut direction, to being $90^\circ$ rotated off the strut, to being another orientation entirely.


The two horizontal struts demonstrated near identical behavior during loading. The samples gain a positive tensile stress as they stretch and then, eventually, both return to a near-zero stress state, followed by a negative stress and a reduction in strain.

The stress in the sample, shown in Figure \ref{200um_stress}, presented unexpected results. The stress was calculated using Equation \ref{sigma} from the principal stresses in Figure \ref{200um_principal_strains}c. Figure \ref{200um_stress}a and \ref{200um_stress}b show $\sigma_{11}$ and $\sigma_{22}$ respectively. In many locations in the sample the character of the stress is the same for both $\sigma_{11}$ and $\sigma_{22}$ which was not expected. Further analysis of the fit revealed that the tr($\epsilon^\ell$) term of Equation \ref{sigma} dominated over the $\epsilon_{ii}$ terms, causing the larger magnitude strain to dominate the stress behavior. 


\begin{figure*}
	\includegraphics[width=1\linewidth]{figures/50um_principal_strains}
	\caption{The principal strains and orientation of the principal coordinate system at a macroscopic displacement of 50um on the sample.}
	\label{50um_principal_strains}
\end{figure*}

\begin{figure*}
	\includegraphics[width=1\linewidth]{figures/200um_principal_strains}
	\caption{The principal strains and orientation of the principal coordinate system at a macroscopic displacement of 200um on the sample.}
	\label{200um_principal_strains}
\end{figure*}

\begin{figure*}
	\includegraphics[width=1\linewidth]{figures/200um_stress}
	\caption{Principal stress in the sample with a) showing the $\sigma_{11}$ stress and b) showing the $\sigma_{22}$ stress.}
	\label{200um_stress}
\end{figure*}


\subsection{FEA Model Without and With Residual Stresses}
\begin{figure*}
	\includegraphics[width=1\linewidth]{figures/fea_no_stress}
	\caption{Results of the FEA simulation tested using only the material geometry and standard material properties for Ti-5553.}
	\label{fea_no_stress}
\end{figure*}

\begin{figure*}
	\includegraphics[width=1\linewidth]{figures/fea_with_stress}
	\caption{Results of the FEA simulation using the material geometry, the standard material properties for Ti-5553, and a gradient of residual stress applied across the sample in the build direction.}
	\label{fea_with_stress}
\end{figure*}

A finite element model of the sample geometry was run to compare the distribution of stresses and strains with the diffraction information. First, the model was run on just the sample geometry as shown in Figure \ref{fea_no_stress}. The plots in Figure \ref{fea_no_stress} are in the $(x,y,z)$ coordinate system as with Figures \ref{50um_principal_strains}a and \ref{200um_principal_strains}a. The behavior of the sample is as predicted relative to the loading direction. The transverse struts exhibit a high tensile stress because their loading direction is directly down the strut, outwards.

The 45$^\circ$ struts exhibit more complex strain states. In the $x$-direction they have a small compressive strain as the nodes are bulging outward in the $x$-direction and pulling the struts with them. In the $y$-direction, the loading direction, a larger compressive strain can be observed. The topmost and bottommost nodes are experiencing a small tensile strain however.

Of note, the FEA model on only the geometry did not reproduce the same anisotropy in distribution of strains as that observed for the diffraction data. 

Because residual stresses were observed in the diffraction data it was hypothesized that they played a role in the anisotropy of the load distribution. As such, the FEA model was re-run with a gradient of residual stress applied in the build direction.

The FEA model with residual stress qualitatively agrees with the diffraction results as shown in Figure \ref{fea_with_stress}. The transverse struts are still in a pure stretch mode and exhibit a high tensile strain. The 45$^\circ$ struts, however, now exhibit a more complex strain distribution. Struts on the left side of the figure show a small tensile strain and a relatively high compressive strain near the central node. The left side of this figure corresponds to build locations nearest the build substrate. On the right side, however, the behavior is not mirrored. Instead, a mix of compressive and tensile strains build up. It is important to notice that the location of the compressive and tensile strains on the left and right sides of the sample match the same locations on the diffraction results.

%%%%%%%%%%%%%%%%%%%%%%%%%%%%%%%%%
\subsection{The Presence of Secondary Phases}
\begin{figure*}
	\includegraphics[width=1\linewidth]{figures/secondary_phases}
	\caption{A secondary phase was observed present in the samples. Diffraction spots indicative of the secondary phase can be observed in a) and b). A logarithmic intensity plot of an integrated diffraction pattern demonstrates the presence of unexpected peaks, highlighted with red arrows in c). The same plot can be seen in linear intensity in d).}
	\label{secondary_phases}
\end{figure*}

An unexpected secondary phase was observed at some locations in the samples. Examples of diffraction peaks from the secondary phase can be observed in Figure \ref{secondary_phases}a and b. The presence of these peaks were primarily observed in the bulk regions of the samples such as at the nodes or in the top and bottom solid sections that mated with the compression platens of the load frame. It is possible that the phase was also present in the strut sections of the sample but there were too few grains for observable peaks to be found. Integrated patterns of the diffraction images in Figure \ref{secondary_phases}.c) reveal low-intensity, broad peaks that are indicative of a small grain size.

Several hypotheses were put forward for the structure of the secondary phase including HCP $\alpha$ titanium, hexagonal martensitic $\omega$ titanium, as well as precipitate phases of Titanium and one of the alloying elements. However, a literature review revealed that a face-centered orthorhombic titanium phase has been observed for Ti-5553 when it undergoes the same heat treatment as used in this study. Zheng et al. performed a TEM study on wrought Ti-5553 and discovered numerous secondary phases, including the secondary phase dubbed $O''$ \cite{Zheng2016}. Zheng solved the structure and found it to be face-centered orthorhombic.

Several phases were fit to fully integrated diffraction patterns to determine the structure of the secondary phase. The phases fit included the $\alpha$, $\omega$ and $O''$ phases. The best fit obtained was using only the $O''$ with the starting lattice parameters reported by Zheng et al., with a final weighted residual of 9\%. The lattice parameters were refined and found to be $a = 3.277\AA$, $b = 4.564 \AA$, and $c = 13.903 \AA$. 
%~~~~~~~~~~~~~
\section{Discussion}

\begin{figure*}
	\includegraphics[width=1\linewidth]{figures/fractography.png}
	\caption{Fractography of the first failure site on a horizontal strut. The strut shown is transverse to the loading direction of the sample. Analysis of the fracture site shows a typical cup-and-cone failure. It does not appear that fracture initiated on a failure site such as an unmelted particle particle or a void.}
	\label{fractography}
\end{figure*}

\begin{figure*}
	\includegraphics[width=1\linewidth]{figures/annotated_fractography}
	\caption{Higher magnitude image of the first failure site, a fracture which occurred in pure stretch. Both macroscopic dimpling can be observed, as well as fracture sites where microvoid coalescence led to cleavage.}
	\label{annofrac}
\end{figure*}

\subsection{Anisotropy in Stress Distribution Due to Residual Stress}
The anisotropy in mechanical response across the sample indicates that the build process impacts the mechanical properties of the material, as well as the geometry. In an ideal scenario, strain would be symmetrically distributed across members of like orientation and distance from the loading surface. The left-right anisotropy in the sample matches the build direction of the printing process. The (001) texture in the build direction, shown in Figure \ref{texture} also indicates build anisotropy. Orientation dependence of grain structures relative to the build direction has been widely reported for laser powder bed fusion additively manufactured materials \cite{Hayes2017,Keist2020, Todaro2020}.

The anisotropy in load distribution is also evident in Figure \ref{200um_principal_strains}c. The transverse struts both show load being concentrated down the strut direction. On the top left hand side of the figure eight, the strains are oriented either down axis or at 90$^\circ$ relative to strut. On the upper right hand side, the struts are almost entirely oriented down the strut. The bottom left hand side and right hand side show even more complicated loading directions; in some cases the load is distributed neither down the strut nor perpendicular to it. 

This anisotropy in the loading directions and distributions of loads likely played a role in the failure locations and order of failure of the sample. The transverse struts build up the most strain the fastest and thus failed first, in plastic yielding as explained in a previous section. Further testing is required to associate the build direction -- as shown in Figure \ref{build_direction} -- with the magnitude and direction of anisotropy in strain distribution. 

Because the samples were loaded quasi-statically and because diffraction characterization can only measure elastic behavior, it was not possible to measure the yield strength from diffraction. However, the DFA provides a prediction of the plastic yield load as
\begin{equation}
	P_Y = \pi a^2\sigma_Y
	\label{yield}
\end{equation}
where $a$ is the radius of the strut (250$\mu$m in this case) and $\sigma_Y$ is the yield strength of the material. Schwab et al. found a yield strength of 0.8 GPa, therefore the plastic yield load of the octet truss should be around 50 kN, a full two orders of magnitude higher than the load at which the sample fractured. 

The elastic buckling stress can be calculated as
\begin{equation}
	\sigma_Y = \frac{k^2\pi^2E_s^2}{12}\left(\frac{t}{\ell}\right)^2
	\label{bucklingstress}
\end{equation}
where $t$ is the strut thickness, $\ell$ is the strut length, $E_s$ is the Young's modulus of the solid, and $k$ is a constant based on if the strut ends are pin-jointed or fixed. In this case the struts are fixed and $k=2$ \cite{Dong2015}. The elastic buckling stress for these parts is 2.69 GPa, which is considerably higher than any stress measured for the samples, as shown in Figure \ref{200um_stress}. Therefore the samples are most likely to fail in a pure stretch mode, which was the case for this sample.

\subsection{Failure Occurring in Pure Stretch}
Digital image correlation partially reveals the deformation behavior of the structure at the surface level. The displacements shown in Figure \ref{e22} indicate that the majority of the surface displacements were concentrated in the $y$ direction at first. Around $180um$s displacement of the middle nodes begins to occur in the $\epsilon_{11}$ direction. 

It is clear from comparing magnitudes in Figures \ref{e11} and \ref{e22} that the majority of the load was concentrated in the loading direction, despite microstructural texture and the anisotropy of residual stresses in the sample. This gives good reason to believe that the horizontal struts through the middle node would be experiencing a purely stretch dominated behavior, as predicted by the DFA model. The outward bending of the horizontal nodes shown at $180\mu$m, $230 \mu$m, and $330\mu$m further demonstrates this because if the nodes are moving outward they are stretching the horizontal members in addition. 

Fracture occurred before buckling in the sample. The Deshpande-Fleck-Ashby model \cite{Deshpande2001} predicts for a perfect material that plastic failure in stretch should occur first, followed by elastic buckling. The first failure to occur in this sample was fracture in a stretch-dominated strut, followed by a second fracture in the other stretch-dominated strut, followed by elastic buckling at the uppermost nodes. The buckling behavior can be seen in the last frames of Figures \ref{e11} and \ref{e22}.

Fractography of the first fracture site, shown in Figure \ref{fractography} reveals that the first fracture occurred due to plastic yielding, as shown in Figure \ref{annofrac}. The fracture site exhibits several different features which are labeled in Figure \ref{annofrac}. The non-circular, reduced cross section shown in Figure \ref{annofrac} is a likely reason why this site failed first as opposed to other locations in the strut. The reduced cross section likely caused a buildup of stress; this is in agreement with the strain behavior shown for the horizontal struts in Figure \ref{200um_principal_strains}. Dimpling can be observed in the regions labeled `ductile' while striations on other facets indicate a different kind of failure mode. However, high magnification fractography revealed that even on flat surfaces with slip plane striations microvoids can be seen. 

The microvoid behavior is in agreement with the pre- and post-mortem CT scans that are shown in Figure \ref{voids}. It is likely that at the flat surfaces in Figure \ref{annofrac} voids existed in the material before compression and grew to larger sizes during loading. This coalescence of voids ultimately led to failure under plastic yield. 

\subsection{Microstructural Evolution}
Microstructure is perhaps a major component missing from homogenization schemes of lattice materials and the investigation herein reveals that the additive process has a pronounced effect on the microstructure. The void behavior in the samples is particularly concerning as the density of the sample changed from $99.95\%$ in the as heat-treated condition to $99.61\%$ in the deformed condition. Based on the size of pores seen in Figure \ref{microstructure} it is likely that pores existed below the voxel threshold of the XCT software and grew in size during compression. While it is unlikely that this had an effect on the mechanical properties in this studies is likely going to play a role in fatigue-limited applications like biomedical implants.


\section{Conclusions}
The mechanical response of additively manufactured Ti-5553 octet truss lattice structures was investigated. The material responded anisotropically to compressive load due to residual stresses that were present from the build process. This was apparent in both HEXRD testing and confirmed by an FEA model.  The primary direction of loading during compressive is down the strut direction, with some locations demonstrating off-axis loading. The failure mechanism of the sample was plastic yielding first, followed by elastic buckling, as predicted by the DFA model. Microvoid nucleation lead to ductile fracture in the sample, with some cases of dimpling occuring and in other cases microvoid coalescence causing cleavage. This is likely due to dislocation buildup which commonly occurs in AM materials.

\section*{Acknowledgements}
We would like to acknowledge a gracious graduate research fellowship from Los Alamos National Laboratory.

\chapter{High Energy X-ray Diffraction Characterization of Additively Manufactured Ti-5553 Octet Truss Lattices}
\begin{itemize}
	\item Characterization technique
	\item Sample manufacturing
	\item Loading conditions
	\item Ashby and Deshpande's theory of octet trusses
	\item Calculation of strain
	\item Calculation of stress
	\item Metallography/microscopy of sample microstructures
	\item fractography of samples
	\item tomography of samples before and after
	\item ex situ loading of samples
	\item Results from HEXRD calculations
	\item FEA models for comparison with HEXRD
	\item spatial statistical regression of strain/stress response in samples
\end{itemize}
\chapter{Summary and Future Work}


%% Parts of a Thesis - Back Matter
\backmatter

%%% Parts of a Thesis - Back Matter - References Cited (required)
% Use "Advanced" Bibliography Techniques
\bibliography{bib.bib}

%%% Appendices

\appendix{Analysis of Diffraction Results using numpy and matplotlib}


\end{document}
