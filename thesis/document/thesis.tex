\documentclass[letterpaper,12pt]{article}


%%%%%%%%%%%%%%%%%%%%%%%%%%%%%%
%%%%%% LATEX PACKAGES %%%%%%%%%%%%%
% csm-thesis automatically includes the following packages:
%% float
%% setspace
%% geometry
%% graphics
%% textcase
%% subfig

% Note: Two package options exist for your convenience: ``insane'' and ``nolabel''.  To use these options together separate them by a comma, ie. \usepackage[insane,nolabel]{csm-thesis}
% * \usepackage[insane]{csm-thesis}
% Turn off all document sanity checks.  This option can be used to render a ``sub-document'' that is part of the root thesis document.  It is important to note that you should NEVER disable this check on your root thesis document, as important format errors and warnings will be disabled.
% * \usepackage[nolabel]{csm-thesis}
% Disables automatic reference ``labeling'' of figures and tables.  By default the thesis template prepends any reference to a figure or table with ``Figure~'' or ``Table~''.  This option is meant for disabling the labeling behavior when a document already has the appropriate labeling.  It is important to note that if your document DOES NOT have the appropriate labeling (the reference label must EXACTLY MATCH the caption label) then it will not pass the format review.
\usepackage{csm-thesis}
% For inserting large multi-page tables:
\usepackage{array}
\usepackage{longtable}
% For inserting sideways tables and figures
\usepackage{rotating}

% Since hyperref and cite don't completely get along, the template now recommends using natbib:
% For an explanation, see http://www.tex.ac.uk/cgi-bin/texfaq2html?label=citesort
\usepackage[numbers]{natbib}
% If you wish to use ``cite'' instead then your choices are:
% 1) Don't use the hyperref package
% 2) Put ``cite'' before hyperref, resulting in no citation hyperlinks
% 3) Put ``cite'' after hyperref, resulting in ugly looking citations

% If you choose not to use natbib then you can set the standard ``numeric'' style like so:
%\bibliographystyle{unsrt}

% Important Note: math-mode in sections, titles, and other bookmarks will generate warnings with hyperref.  You can work around this by either:
% 1) Not using the hyperref package
% 2) Using \texorpdfstring{TeX Code}{PDF Replacement} to display an alternative bookmark (for example, \texorpdfstring{H$_2$O}{Water}).
% The thesis template will automatically import your document information into hyperref, so if you go to ``File | Properties'' in Adobe Acrobat it will display the title and author.  If you would like to over-ride this option then just change the line below to ``\usepackage[]{hyperref}''.
\usepackage{hyperref}
% For inserting programming code:
\usepackage{listings}
% For inserting landscape-mode objects:
\usepackage{pdflscape} % use ``lscape'' if you are not creating a PDF output
% For matrices:
\usepackage{amsmath}
% For differently colored fonts
\usepackage{xcolor}

%%%%%%%%%%%%%%%%%%%%%%%%%%%%%%%%%%%%%%%%%
%%%%%%%%%%%%%%%%%%%%%%%%%%%%%%%%%%%%%%%%%
%%%%%%%%%%%%%%%%%%%%%%%%%%%%%%%%%%%%%%%%%


%%%%%%%%%%%%%%%%%%%%%%%%%%%%%%%%%%%%%%%%%
%%%%%%%% DOCUMENT START %%%%%%%%%%%%%%%%%%%%%%

% <<AUTOMATIC PYRAMID:>>
% Do not put any carriage returns (\\), all normal LaTeX should function properly.
	\title{Ph.D Thesis Working Title}
% Please note: If you are generating a title containing math mode then it is best to use \texorpdfstring to provide an alternative text for the PDF Title.  If you do not do this then you will see a ''Token not allowed in a PDFDocEncoded string`` warning when rendering your document.


\degreetitle{Doctor of Philosophy}
\discipline{Materials Science and Engineering}
\department{Mechanical Engineering}

\author{Nathan S. Johnson}
\advisor{Dr. Craig A. Brice}
% Comment out the following line if you do not have a co-advisor:
\coadvisor{Dr. Aaron P. Stebner}
\dpthead{Dr. John Berger}{Professor and Department Head}

\begin{document}

% Parts of a Thesis
%% Parts of a Thesis - Front Matter

\frontmatter

%%% Parts of a Thesis - Front Matter - Title Page (required)
\maketitle
\newpage

%%% Parts of a Thesis - Front Matter - Copyright Page (optional)
% If the copyright for your document spans multiple years, or does not match the current year, then replace ''\the\year`` below with the appropriate text.
\makecopyright{\the\year}
\newpage

%%% Parts of a Thesis - Front Matter - Signature Page (required)
\makesubmittal
\newpage

%%% Parts of a Thesis - Front Matter - Abstract (required)
\begin{abstract}

This thesis focused on the characterization of additively manufactured titanium alloys using high energy X-ray diffraction.

\end{abstract}
\newpage

%%% Parts of a Thesis - Front Matter - Table of Contents (required)
\tableofcontents
\newpage

%%% Parts of a Thesis - Front Matter - List of Figures (if applicable)
%%% Parts of a Thesis - Front Matter - List of Tables (if applicable)
% NOTE: If you have more than 2 items in either list they must be separate.
% This case is generally handled automatically, but if you are told to separate the lists then comment or remove the two lines below:
\listoffiguresandtables
\newpage

% ... and then uncomment these four lines to force separate lists:
%\listoffigures
%\newpage
%\listoftables
%\newpage


%% NOTE: If included in the front matter, a glossary, a list of abbreviations, or a list of symbols is placed as the last list. If these lists are included in the back matter, they are placed immediately before the REFERENCES CITED.

%%% Parts of a Thesis - Front Matter - Glossary (if applicable)
%\glossary

%%% Parts of a Thesis - Front Matter - List of Symbols (if applicable)

% Place this call before ''\listofsymbols`` to make the symbols appear on the left instead of the right:
%\ShowSymbolFirst
% To call the “List of Symbols” “Nomenclature” instead use:
%\listofsymbols[Nomenclature]
% To autosort the list use a star after the command (ie. \listofsymbols*[Nomenclature] or \listofsymbols*)
\listofsymbols
% With very large symbol lists it is sometimes good to split the list into multiple sub-lists.  To output the lists just use the extended \listofsymbols command (below) and to add an element to the list use the optional parameter to ''\addsymbol``.
%\listofsymbols{General Nomenclature}
%\listofsymbols{Greek Letters}
\newpage

% Note that you may define symbols anywhere in the document, when you re-run LaTeX they
% will be added to the list (just like all other lists)
\addsymbol{alpha}{$\alpha$}

% Example for sub-list symbols (optional parameter specifies which list to use):
%\addsymbol[General Nomenclature]{absorption coefficient}{$\alpha_c$}
%\addsymbol[General Nomenclature]{absorption cross section}{$\alpha_{\sigma}$}
%\addsymbol[Greek Letters]{average radius of cylindrical shell}{$c$}
%\addsymbol[Greek Letters]{activation energy of oxidation reaction of a-C in excited state}{$E^{\ast}_{act}$}

%%% Parts of a Thesis - Front Matter - List of Abbreviations (if applicable)
% To autosort the list use a star after the command (ie. \listofabbreviations*)
\listofabbreviations
\newpage

% Note that you may define abbreviations anywhere in the document, when you re-run LaTeX they
% will be added to the list (just like all other lists)
\addabbreviation{high energy X-ray diffraction}{HEXRD}


%%% Parts of a Thesis - Front Matter - Acknowledgments (optional)
\begin{acknowledgments}
Mom \& Dad \& Noelle \& Zack \\
Thomas \& Marcella \\

\end{acknowledgments}
\newpage

%%% Parts of a Thesis - Front Matter - Dedication (optional)
\begin{dedication}
To be determined.
\end{dedication}
\newpage

%~~~~~~~~~~~~~~~~~~~~~~~~~~~~~~~~~~~~~~~~~~~
%~~~~~~~~~~~~~~~~~~Main Text Body~~~~~~~~~~~~~
%% Parts of a Thesis - Body
\bodymatter

% Outline
\chapter{Outline}
This is an outline that I made with Dr. Stebner on February 13th of 2020. It is a rough outline of the chapters of my Ph.D thesis, as well as the potential publications that might result from each chapter.

\begin{enumerate}
	\item Intro 
		\begin{enumerate}
		\item Motivation
		\item Background 
			\begin{enumerate}
			\item Diffraction Techniques
			\item Materials science of metals AM
			\end{enumerate}
		\end{enumerate}
	\item Vision for Machine Learning Applications in Additive Manufacturing
	\item Austenititic Stainless Steel Weld Beads
	\item Ti-6Al-4V Weld Beads 
	\item Mechanical Behavior and Phase Transformations of Additively Manufacture NiTi Elastocaloric Cooling Materials 
	\item Single Unit Cell Lattices
	\item Multiple Unit Cell Lattices
	\item Summary and Future Work
	\item Appendices
		\begin{enumerate}
		\item Matrix Completion
		\item Lockheed Martin Ti-6Al-4V Antler Build Characterization
		\item Scripting with GSASII Scriptable
		\end{enumerate}
\end{enumerate} 
\chapter{Scientific Approaches to Researching, Modeling, and Engineering Additively Manufactured Materials}
This section probably needs to come first because (at least at the moment) it contains a lot of literature review that is relevant to the later chapters.

\chapter{In Situ High Energy X-ray Diffraction of Cold Metal Transfer Welded SS308L and Ti-6Al-4V}
Highlights of this chapter:
\begin{itemize}
	\item Characterization technique
	\item Equilibrium phase diagrams, continuous cooling curves, TTT diagrams for each alloy
	\item Results obtained
	\item SS308L -- measurement of temperature, cooling rate, difficulty in calculating temperature gradient
	\item Ti-6Al-4V -- measurement of phases, attempts at fitting temperature
	\item Ti-6Al-4V -- observation of $\alpha$ unit cell shift during $\beta \to \alpha$ transition
	\item Ti-6Al-4V -- differences in cooling rate, phase fraction observed between point depositions and line welds
\end{itemize}

\subsection{Characterization Technique}
There is a lot of information to cover here and a lot of it is going to overlap/relate to the lattices section as well. I want to get the in situ stuff figured out first, then the lattices characterization, and then find a way to make them into one section.

\subsubsection{High Energy X-ray Diffraction}

\subsubsection{Scattering Through Amorphous Materials}

\subsubsection{Scattering Through a Rapidly Cooling Material}
Debye-Waller Factor, Preferred Orientation of Samples, Detector Acquisition Speed, Detector Coverage

\subsubsection{Diffraction Information Specific to $\alpha + \beta$ Ti alloys}


\subsubsection{The various sources of broadening and intensity drops}
Residual stress, dislocation density, alloy partitioning, the Lorentz factor

%%%%%%
\subsection{Material Properties of Ti-6Al-4V}
This is where we include important background information on Ti-6Al-4V including things like:
\begin{itemize}
	\item Composition
	\item Phase diagram
	\item Thermomechanical properties including CTEs and moduli
	\item Microstructural characteristics, both the possible microstructures and the ones we observe
	\item Deviations in material properties between traditional manufacturing and wire feed DED
\end{itemize}

%%%%%%
\subsection{Calculation of Temperature}
Write up Don and Adrian's idea of calculating temperature using lattice thermal strains. Show their applicability to SS but inapplicability to Ti-6Al-4V. Discuss possible reasons why Ti-6Al-4V did not work out.


%%%%%% 
\subsection{Phase transformations during solidification/cooling of the material}
Discuss what you observed in your diffraction results. Talk about whether or not the observed results can be interpreted to be `true' or if something else (like preferred orientation) is skewing the results. Talk about what useful information you got out of it.

%%%%%%
\subsection{The martensitic phase transformation in Ti-6Al-4V}
This is where the letter on unit cell shifts observed will go. Of course, you need to tweak it to flow with the rest of the chapter. 


%%%%%
\subsection{Differences between point depositions and line welds}
Talk about the observed differences in results between the two deposition geometries. What does this indicate about differences in the processes? Why are they different? What implications does this have for the modeling/research of AM Ti-6Al-4V?


%%%%%
\subsection{CHESS samples}
This research is not quite yet resolved -- how do the results of your experiment at CHESS, coupled with Behnam's TEM of the same wall, relate to the results obtained in situ?
\chapter{High Energy X-ray Diffraction Characterization of Additively Manufactured Ti-5553 Octet Truss Lattices}
\begin{itemize}
	\item Characterization technique
	\item Sample manufacturing
	\item Loading conditions
	\item Ashby and Deshpande's theory of octet trusses
	\item Calculation of strain
	\item Calculation of stress
	\item Metallography/microscopy of sample microstructures
	\item fractography of samples
	\item tomography of samples before and after
	\item ex situ loading of samples
	\item Results from HEXRD calculations
	\item FEA models for comparison with HEXRD
	\item spatial statistical regression of strain/stress response in samples
\end{itemize}

%% Parts of a Thesis - Back Matter
\backmatter

%%% Parts of a Thesis - Back Matter - References Cited (required)
% Use "Advanced" Bibliography Techniques
\bibliography{thesis}

%%% Appendices

\appendix{Analysis of Diffraction Results using numpy and matplotlib}


\end{document}
