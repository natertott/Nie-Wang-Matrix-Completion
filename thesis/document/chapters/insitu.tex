\chapter{In Situ High Energy X-ray Diffraction of Cold Metal Transfer Welded SS308L and Ti-6Al-4V}
Highlights of this chapter:
\begin{itemize}
	\item Characterization technique
	\item Equilibrium phase diagrams, continuous cooling curves, TTT diagrams for each alloy
	\item Results obtained
	\item SS308L -- measurement of temperature, cooling rate, difficulty in calculating temperature gradient
	\item Ti-6Al-4V -- measurement of phases, attempts at fitting temperature
	\item Ti-6Al-4V -- observation of $\alpha$ unit cell shift during $\beta \to \alpha$ transition
	\item Ti-6Al-4V -- differences in cooling rate, phase fraction observed between point depositions and line welds
\end{itemize}

\subsection{Characterization Technique}
There is a lot of information to cover here and a lot of it is going to overlap/relate to the lattices section as well. I want to get the in situ stuff figured out first, then the lattices characterization, and then find a way to make them into one section.

\subsubsection{High Energy X-ray Diffraction}

\subsubsection{Scattering Through Amorphous Materials}

\subsubsection{Scattering Through a Rapidly Cooling Material}
Debye-Waller Factor, Preferred Orientation of Samples, Detector Acquisition Speed, Detector Coverage

\subsubsection{Diffraction Information Specific to $\alpha + \beta$ Ti alloys}


\subsubsection{The various sources of broadening and intensity drops}
Residual stress, dislocation density, alloy partitioning, the Lorentz factor

%%%%%%
\subsection{Material Properties of Ti-6Al-4V}
This is where we include important background information on Ti-6Al-4V including things like:
\begin{itemize}
	\item Composition
	\item Phase diagram
	\item Thermomechanical properties including CTEs and moduli
	\item Microstructural characteristics, both the possible microstructures and the ones we observe
	\item Deviations in material properties between traditional manufacturing and wire feed DED
\end{itemize}

%%%%%%
\subsection{Calculation of Temperature}
Write up Don and Adrian's idea of calculating temperature using lattice thermal strains. Show their applicability to SS but inapplicability to Ti-6Al-4V. Discuss possible reasons why Ti-6Al-4V did not work out.


%%%%%% 
\subsection{Phase transformations during solidification/cooling of the material}
Discuss what you observed in your diffraction results. Talk about whether or not the observed results can be interpreted to be `true' or if something else (like preferred orientation) is skewing the results. Talk about what useful information you got out of it.

%%%%%%
\subsection{The martensitic phase transformation in Ti-6Al-4V}
This is where the letter on unit cell shifts observed will go. Of course, you need to tweak it to flow with the rest of the chapter. 


%%%%%
\subsection{Differences between point depositions and line welds}
Talk about the observed differences in results between the two deposition geometries. What does this indicate about differences in the processes? Why are they different? What implications does this have for the modeling/research of AM Ti-6Al-4V?


%%%%%
\subsection{CHESS samples}
This research is not quite yet resolved -- how do the results of your experiment at CHESS, coupled with Behnam's TEM of the same wall, relate to the results obtained in situ?