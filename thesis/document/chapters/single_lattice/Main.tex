\documentclass[aps,twocolumn,nofootinbib]{revtex4-1}
\usepackage{amsmath}
\usepackage{amssymb}
\usepackage{graphicx}
\usepackage{floatrow}
\usepackage{subcaption}
\usepackage{xcolor}


\begin{document}
\title{Integrated Computational Materials Engineering Investigation of Additively Manufactured Ti-5553 Octet Truss Lattice Unit Cells}
\author{Nathan S. Johnson}
\affiliation{Colorado School of Mines}
\affiliation{Los Alamos National Laboratory}

\author{Maria Strantza, Jenny Wang, Manyalibo Matthews}
\affiliation{Lawrence Livermore National Laboratory}

\author{Dave Macknelly}
\affiliation{Atomic Weapons Establishment}

\author{Jun-Sang Park, Peter Kenesei}
\affiliation{Argonne National Laboratory}

\author{Bj\o rn Clausen, Cheng Liu, Donald W. Brown, John S. Carpenter}
\affiliation{Los Alamos National Laboratory}

\author{Craig A. Brice, Aaron P. Stebner}
\affiliation{Colorado School of Mines}

\begin{abstract}
Additive manufacturing has made the fabrication of complex geometries with small feature sizes possible in recent years. One technology AM is uniquely suited to benefit are octet truss lattice structures. The octet truss lattice structure for structural applications was first described in the theory of Ashby, Deshpande, and Fleck in the late 1990s and has seen an increase in use recently. Among the many applications of octet trusses are for lightweighting structural applications in defense, aerospace, and biomedical applications. However, the impact of the additive manufacturing process has not yet been elucidated for these materials. Octet truss lattice structures were manufactured using selective laser melting in a 2x1x1 unit geometry and subjected to uniaxial compression. A suite of characterization techniques was used to study the samples during compression including high energy X-ray diffraction and digital image correlation. A second suite of characterization techniques was used to study the samples ex situ including X-ray computed tomography and texture mapping. All of this information was compared against a finite element model of the same geometry. It was discovered that anisotropy in distribution of mechanical strains existed due to residual stresses caused by the build process. Furthermore, the samples failed in a pure stretch mode as predicted by the DFA model. This result has implications for how the AM build process will impact mechanical properties of octet trusses manufactured on a larger scale.
\end{abstract}
\date{\today}
\maketitle

%~~~~~~~~~~~~~~
\subsection{Introduction}
Continuous metal micro lattice structures have been theorized and described since the 1990s. These structures are able to provide specifically engineered mechanical properties, like stiffness and strength, at a fraction of the density of bulk metals. Early adoptions of the lattice structure include the related open and closed cellular foam materials \cite{Bonatti2017} as well as the snap-fit and weld joined lattices \cite{Dong2015}. 

In 2001, Deshpande, Fleck, and Ashby (DFA) presented a continuum theory for one type of lattices structure: the octet truss \cite{Deshpande2001}. Octet truss lattice structures are analogous to face centered cubic crystal structures, where the nodes of the lattice correspond to atom sites. The DFA model provides predictions for macroscopic stress and strain in the trusses, as well as calculations of important mechanical properties like yield strength, buckling strength, and failure modes. In particular, the DFA model predicts that the mechanical deformation in octet trusses should be stretch-dominated first, with buckling occurring as a secondary mechanism. While the DFA model is robust, the technology at the time of its introduction was limited to characterize such complex structures. Advancements in imaging analysis and computing have made characterization of these structures possible. Furthermore, advancements in additive manufacturing (AM) have allowed for these complex structures to be manufactured with little machining and for arbitrarily large and complex geometries to be built.

The combination of a unique geometry with the octet truss and the additive manufacturing process requires multiple approaches for characterization. The geometry of the octet truss means that models like the DFA model can be used to assess mechanical behavior, specifically failure of the part. These continuum models can be compared against computational models using finite element analysis or against macroscopic stress-strain response observed. However, the additive manufacturing process adds another complication into the mix. The additive process is known to create unique microstructures. Additive manufacturing impacts grain morphologies \cite{Tan2015, Zhu2018}, dislocation density \cite{Zhang2015, Wang2017, Gallmeyer2020} crystallographic texture \cite{Wang2019}, phase fractions \cite{Gallmeyer2020}, and defect structures \cite{Yang2017,Matthews2016}, all of which impact the material structure-property relationship and therefore the mechanical performance. 

Several studies have performed mechanical testing or finite element modeling of octet trusses and, in some cases, both. Dong et al, examined the macroscopic response of snap-fit octet truss lattices and compared the results to the DFA model, finding that snap-fit octet truss lattices moduli are well predicted by the DFA model \cite{Dong2015}. Latture, Begley, and Zok evaluated the mechanical performance of ideal lattice structures with octet trusses being one amongst many studied \cite{Latture2017}. Bonatti and Mohr looked at macroscopic mechanical measurements of large (5x5x5 unit cell) octet truss structures and characterized the failure mechanisms \cite{Bonatti2017}. Tancogne-Dejean did a comprehensive study that compared macroscopic stress-strain results for an octet truss to finite element modeling \cite{Tancogne-Dejean2016}. Their approach examined different unit cell densities, strut thicknesses, and strut shapes and the impact on mechanical response. Tancogne-Dejean was able to achieve decent agreement in failure modes for different densities between a finite element model and additively manufactured samples. 

Other studies have examined how additive manufacturing impacts lattice materials. Yan et al. investigated additively manufactured gyroid structures, a different kind of period lattice material \cite{Yan2012}. Yan was primarily concerned with the manufacturability of the structure and its impact on surface roughness, shape, and deformation. They found that some cracking occurred in the samples during printing due to residual stress buildup. Tancogne-Dejean used selective laser melting to build and assess the macroscopic response of octet truss samples \cite{Tancogne-Dejean2016}. They found that changes in morphology and texture of the sample can result in mechanical property differences up to 20\%. Liu et al. studied the role of geometric imperfections of the struts due to the SLM process on octet truss lattices \cite{Liu2017}. They counted a statistical distribution of strut deformities based on how far the strut deviated from a perfect circle. These distributions of imperfections were then introduced into a finite element model which varied the beam width and thickness in order to capture imperfections. They found a decrease in Young's modulus in all directions along the struts as a result of the imperfections. This was linked back to the manufacturing process by the realization that struts in the orientation of the build plane were oversized.

These studies are all concerned with comparing continuum level models of octet trusses to macroscopic responses in the parts. What is often unclear is how the behavior of an individual unit cell impacts the macroscopic behavior of the samples. To this end, several authors have implemented homogenization theories which connect unit-cell-level mechanical properties to macroscopic mechanical properties. Park et al. developed a homogenization scheme which takes into account deformations at the unit cell scale due to build processes in fused deposition modeling and predicts mechanical performance at the macroscale \cite{Park2014}. Vigliotti et al. used the representative volume element approach to develop a homogenization scheme which linked discrete elements (unit cells) to the macroscopic behavior of arbitrarily shaped lattice structures \cite{Vigliotti2014}. Mohr used a similar approach which treated the macroscopic part as a homogenous medium and calculated mechanical properties based on the mechanics of unit cells \cite{Mohr2005}. In all cases, however, the unit cell is treated as a  a perfect material.

The success of homogenization schemes is built upon having accurate properties for the representative volume elements. Consideration of the additive manufacturing process including deformation, defects, microstructure, texture, and more must be made for models to accurately predict macroscopic deformation behavior of lattice materials. To this end, it is beneficial to examine how additive manufacturing impacts an individual unit cell.

In this case, samples were manufactured out of Ti-5V-5Mo-5Hf-3Cr (Ti-5553). Some studies have looked at the impact of additive manufacturing on Ti-5553 specifically, such as Schwab et al. \cite{Schwab2016}. Characterization of the phases resulting from the AM process will be especially important because Ti-5553 is known to have unexpected secondary phases form \cite{Dehghan-Manshadi2011, Zheng2016}. Post-processing heat treatments will be equally important to characterize, as the heat treatment schedule can significantly vary mechanical properties like Young's modulus and yield strength \cite{Kar2014}.

This study is unique because it combines all of the above characterizations into a single picture to evaluate how the additive manufacturing process impacts the expected mechanical performance of two unit cells of a Ti-5553 octet truss lattice structure. Based on the nomenclature of Latture, Begley, and Zok \cite{Latture2019} the sample geometry is $\{2\text{FCC}\}^1$ The present investigation combines high energy X-ray diffraction, digital image correlation, finite element modeling, and fractography to evaluate the mechanical performance, texture, strain distribution, failure mechanisms, and residual stresses of the parts. 

The study herein takes a processing-structure-property approach to characterize the mechanical response of individual unit cells of additively manufactured Ti-5553 octet truss lattices. While previous studies have evaluated macroscopic behavior of lattices and compared them with continuum level predictions of mechanical properties, no study has yet investigated individual unit cells. The measurement and characterization of individual unit cells is important for verifying the many homogenization schemes which exist for these parts.

%~~~~~~~~~~~~~~~~~~~~~~~~~~~
\subsection{Characterization Techniques}

\begin{figure}[b]
	\includegraphics[width=1\linewidth]{/Users/njohnson/git/thesis/document/chapters/single_lattice/figures/build_direction}
	\caption{The orientation of the part on the build substrate relative to the build direction and recoater blade direction.}
	\label{build_direction}
\end{figure}

\subsubsection{Printing Parameters and Heat Treatment}
Samples were printed on an SLM Solutions SLM 280 printer. Two different sets of scan parameters were used for the bulk and strut portions of the part. For the strut portion, a scan speed of 800 mm/s was used with a laser power of 100W and a hatch spacing of 0.12 mm. For the bulk portion of the sample, a scan speed of 725 mm/s was used with a laser power of 175W and a hatch spacing of 0.12 mm. The bulk portion also had a skin layer applied with a laser scan speed of 525 mm/s and a laser power of 100W. 

The samples were subjected to ultrasonic cleaning after manufacture, then heat treated at 300$^\circ$C for 1 hour (within 16-17$^\circ$C per minute) in vacuum (10$^{-5}$ mbar) while on the build substrate. After 1hr samples were cooled in an Argon atmosphere to 90$^\circ$C.

The orientation of the part on the build substrate can be seen in Figure \ref{build_direction}. It is important to note the orientation of truss members relative to the build and heat flow directions. Some of the trusses are oriented flat on the substrate and should experience even heating and cooling cycles during building. Other members are extended at a 45$^\circ$ angle in the build direction. These extended members will experience many heating and cooling cycles during the build process. This behavior is known to cause residual stress buildup in the sample \cite{Ganeriwala2019}.

\begin{figure}[b]
	\includegraphics[width=1\linewidth]{/Users/njohnson/git/thesis/document/chapters/single_lattice/figures/crosshead_displacement}
	\caption{Load frame crosshead displacement and associated measured force. Each drop in force corresponds to a fracture or buckling in the sample. Diffraction data was collected through 380$\mu$m of displacement.}
	\label{loaddisp}
\end{figure}


%~
\subsubsection{High Energy X-ray Diffraction}
\paragraph{Experimental Setup}
\begin{figure*}[t]
	\includegraphics[width=1\linewidth]{/Users/njohnson/git/thesis/document/chapters/single_lattice/figures/exp_setup.png}
	\caption{The orientation of the part in the laboratory coordinate frame relative to the X-ray beam and the detector.}
	\label{expsetup}
\end{figure*}

%
\begin{figure}
	\includegraphics[width=1\linewidth]{/Users/njohnson/git/thesis/document/chapters/single_lattice/figures/grayscale_microstructure}
	\caption{SEM micrograph of the microstructure of a bulk piece of additively manufactured Ti-5553. Several small pores are highlighted by red circles.}
	\label{microstructure}
\end{figure}
%~

Samples were loaded in a custom load frame developed by the Advanced Photon Source and collaborators \cite{Loadframe}. Samples were loaded quasi-statically in increments of $25 \mu$m through elastic loading. Once buckling and fracture occurred samples were loaded quasi-statically until further buckling and fracture occurred. The load frame crosshead displacement and measured force can be seen in Figure \ref{loaddisp}.

High-energy x-ray diffraction experiments were conducted at the 1-ID beamline of the Advanced Photon Source, Argonne National Laboratory. A diagram of the experimental setup used can be seen in Figure \ref{expsetup}. A 71.6 keV beam was used for transmission X-ray diffraction. All data was collected on a Dexela detector with a 78$\mu$m resolution and pixel dimensions of 3888 $\times$ 3072 pixels. Samples were exposed for $0.1$s at each location. The beam was scanned along struts in steps of roughly 40 $\mu$m. The size of the beam relative to a strut can be seen in the upper right hand corner of Figure \ref{expsetup}.%Note: that's 75 images per strut (150 for the transverse struts) with a strut length of 3.5mm

The microstructure of the sample can be seen in Figure \ref{microstructure}, demonstrating a fairly homogenous distribution of grains whose sizes range from tens of microns to around $75\mu$m in diameter. This gives reason to suggest that good grain statistics were observed in the diffraction pattern as the beam was being scanned along a fairly large number of grains at once.

Due to sample deformation during loading the movement of the struts had to be tracked to ensure that scans were taken on the strut. Each node location was tracked and recorded using computed tomography after each load step. The coordinates of the nodes were then given to an algorithm that computed the straight-line path from node to node. This output of this algorithm was then used to perform fly scanning across the struts.

\paragraph{Explanation of Plots}
\begin{figure}
	\includegraphics[width=1\linewidth]{/Users/njohnson/git/thesis/document/chapters/single_lattice/figures/location_explanation}
	\caption{Locations on the sample that were measured and a corresponding 2D plot of measurements locations. This type of plot will be used throughout the paper.}
	\label{location_explanation}
\end{figure}
Due to the complicated geometry of the sample being studied, in combination with the 3-dimensional nature of the measurements made, it is necessary to explain how plots of data have been constructed. Figure \ref{location_explanation} shows the locations measured using high energy X-ray diffraction colored in red, yellow, and green. For the sake of display, the measurements at these locations were projected onto 2-dimensional plots. An example of one such plot is also shown in Figure \ref{location_explanation}b. It is important to note that the two horizontal lines shown in Figure \ref{location_explanation}b are not one continuous strut, but rather two struts that sit perpendicular to one another.

Certain locations on the sample contained multiple struts that overlapped each other in the beam direction. If the beam hit multiple struts at a time then it became impossible to differentiate the signals of each, causing significant beam broadening and uncertainty in diffraction peak location. To overcome this, data was discarded for locations that featured multiple diffracting struts at once. The locations where data was discarded are shown as black sections on Figure \ref{location_explanation}.

%~
\paragraph{Stress and Strain Calculations}
Each diffraction image was binned into 24 bins consisting of 15$^\circ$ of the full diffraction ring around the coordinate $\eta$, which is the angle between $y$ and $x$ in Figure \ref{expsetup}. After binning, the bins were integrated to produce a diffraction histogram for each angle.

The individual $\eta$ direction histograms were integrated and refined with Rietveld Refinement using GSAS \cite{GSAS} and the SMARTSware routine \cite{Smartsware2004}. This refinement produced a lattice parameter for each eta direction and load step $a_{\eta}^{\ell}$ where $\eta = 0^\circ, 15^\circ,...345^\circ$ is the integration direction and $\ell = 0 \mu\text{m}, 25\mu\text{m}, 50 \mu\text{m}, ... , 380 \mu\text{m}$ is the crosshead displacement in compression at each load step.

The choice of datum used was the averaged lattice parameter at zero load, or
\begin{equation}
	\bar a^0 = \frac{1}{24}\sum_\eta a_{\eta}^{0}.
	\label{datum}
\end{equation}
The choice of datum is important and has several effects on the results. Most importantly, if there is a residual stress on the sample then taking the averaged lattice parameter at zero load will remove distortion of the lattice parameter in certain $\eta$ orientations that would otherwise exist. 

Once the datum was found then a strain at each load step $\ell$ and orientation $\eta$ was computed as
\begin{equation}
	\epsilon_\eta^\ell = \frac{\bar a^0 - a_\eta^\ell}{\bar a^0}.
	\label{strain}
\end{equation}

One important aspect of this investigation was to find the value and orientation of \textit{principal strains} in the sample. The maximum tensile $\epsilon_{11}$ and compressive $\epsilon_{22}$ strains give information about where the maximum strains are building up on the sample while the orientation of the principal strain coordinate system $\psi$ give information about whether the strains are along the struts or at an off-axis angle. This can reveal if the AM microstructure of the sample is having an impact on the loading behavior of the sample. The principal strains were fit using the model of I.C. Noyan and and J.B. Cohen \cite{Noyan1987}, which has been used in similar investigations \cite{Brown2019}, given by
\begin{equation}
	\epsilon_{\eta}^{\ell} = \epsilon_{11}^{\ell}\sin^2{\left( \eta + \psi\right)} + \epsilon_{22}^{\ell}\cos^2{\left(\eta+\psi\right)}
	\label{strainmodel}
\end{equation}
where $\psi$ is the orientation, in degrees, of the principal coordinate system relative to $\eta = 0^\circ$ in the test coordinate system.

\begin{figure*}[t]
	\includegraphics[width=01\linewidth]{/Users/njohnson/git/thesis/document/chapters/single_lattice/figures/voids.png}
	\caption{Voids in the sample after heat treatment and after compression through multiple strut failures.}
	\label{voids}
\end{figure*}

The full model of Noyan and Cohen includes shear terms and a third strain component $\epsilon_{33}$ but in the current experiment only two strain components could be measured at a single time. In order to calculate a stress, however, assumptions about the third strain component had to be made. Calculation of stress is desirable because it allows measurement of mechanical values like yield strength, ultimate tensile strength, shear modulus, and more. These measured values can then be compared to the values predicted by the DFA model.

The third strain component was set as equal to either the $\epsilon_{11}$ or $\epsilon_{22}$ strain depending on the location of the strain on the sample. In the horizontal struts it is assumed that the primary tensile direction $\epsilon_{11}$ is along the strut direction; for horizontal members this is along the $x$ direction in the laboratory coordinate system; for other members, this is $45^\circ$ away from the $x$ axis toward the $y$ axis. Therefore, by Poisson's ratio, the other two strains should be compressive. The opposite was assumed for all remaining struts: the third strain component $\epsilon_{33}$ was set equal to the primary tensile strains.

In order to compute stress in the sample it was first necessary to compute certain material constants including the elastic modulus $E$ and Poisson's ratio $\nu$. A sample of known dimensions was cut from bulk Ti-5553 that had been manufactured using the same print parameters as the octet truss sample. The sample was loaded through the elastic region in compression and the material constants were found. A Young's modulus of $E_s = 107.02$ GPa and Poisson's ratio of $\nu = 0.34$ were measured. This is slightly higher than the values found by other investigations, which normally fell in the range of $\sim$ 90-100 GPa \cite{Clement2010}, although these measurements were found for alloys which underwent slightly different heat treatments and were measured using more traditional mechanical characterization.

The effective modulus of the lattice $E_L$ is computed as
\begin{equation}
	E_L = E_s \frac{\rho}{9}
	\label{effmod}
\end{equation}
where $E_s$ is the Young's modulus of the base material and $\rho$ is the density of the solid as calculated from the DFA model. In this case samples of $20\%$ density were manufactured making the effective modulus of the sample $E_L = 2.38$ GPa.

Once the material constants were computed from the bulk sample, the stress was calculated using
\begin{equation}
	\sigma_{ii}^\ell = \frac{E_s}{1+\nu}\epsilon_{ii}^\ell + \frac{\nu E_s}{(1+\nu)(1-2\nu)}\text{tr}\left(\mathbf{\epsilon^\ell}\right)
	\label{sigma}
\end{equation}
where $\text{tr}\left(\mathbf{\epsilon}\right)$ is the trace of the strain tensor computed from Eqn. \ref{strainmodel}. 

%~
\subsection{Finite Element Modeling}
Ansys Mechanical 2020 was used to conduct a FEA simulation to predict the residual stress within the unit cell resulting from the SLM process. The model used was a 1-way coupled non-linear thermal-structural analysis, using a layer-wise approximation of the SLM process. The mesh for this simulation used 0.15mm size quadratic hexahedral elements constrained to a cartesian grid aligned with the build direction of the part (commonly known as a voxel mesh). Each layer of elements (representing multiple physical powder layers) were activated at the melting temperature of Ti-5553 and allowed to cool, thereby inducing residual stress within the component. The layer-wise approximation of the SLM build process sacrifices the detail of local anisotropy due to the laser scan path, but allows for full temperature-dependent non-linear material models to be used whilst running on a multi-core workstation computer.

Two compression simulations were constructed using the same geometry, but utilising a finer, conformal hexahedral mesh. Whilst the geometry was shared between the SLM and compression simulations, measurement paths were defined within the body in order to extract strain results in locations and coordinate systems identical to those measured on the physical test specimen.

The elemental stress results from the SLM simulation were imported and interpolated to initialise one of the compression simulations, in order to validate the hypothesis that residual stress plays a role in the anisotropy of the load distribution within the sample. Strain results from both compression simulations were exported and visualised within MATLAB to perform a qualitative comparison between simulations with and without residual stress applied.


\begin{figure}
		\includegraphics[width=1\linewidth]{/Users/njohnson/git/thesis/document/chapters/single_lattice/figures/001PF_Ti5553}
		\caption{Pole figure of the (001) plane normal in a bulk (node) part of the specimen.}
		\label{P3}
\end{figure}
	%
\begin{figure}
		\includegraphics[width=1\linewidth]{/Users/njohnson/git/thesis/document/chapters/single_lattice/figures/P3_001PF_Ti5553}
		\caption{Pole figure of the (001) plane normal along the strut direction for one of the horizontal (transverse) struts passing through the center node of the sample.}
		\label{strut}
	\label{texture}
\end{figure}



\begin{figure*}
	\includegraphics[width=1\linewidth]{/Users/njohnson/git/thesis/document/chapters/single_lattice/figures/DIC_e11}
	\caption{The $\epsilon_{11}$ principal strains on the surface of the sample calculated from digital image correlation. The load frame crosshead displacement is shown for each image.}
	\label{e11}
\end{figure*}
\begin{figure*}
	\includegraphics[width=1\linewidth]{/Users/njohnson/git/thesis/document/chapters/single_lattice/figures/DIC_e22}
	\caption{The $\epsilon_{22}$ principal strains on the surface of the sample calculated from digital image correlation. The load frame crosshead displacement is shown for each image.}
	\label{e22}
\end{figure*}

%~
\subsubsection{Ex Situ Characterization}
Characterization of the sample was performed both pre- and post-mortem to identify the impact of additive manufacturing on the microstructure and to observe changes in that microstructure, respectively. A wide suite of ex situ characterization techniques was used.


\paragraph{X-ray Computed Tomography}


X-ray computed tomography was performed on a Zeiss Xradia 510. The isotropic voxel width was 7$\mu$m. The smallest detectable defect size was 16 voxels.

The pre-mortem and post-mortem samples can be seen in Figure \ref{voids}. Large voids existed in bulk regions of the sample such as in the nodes or in the build substrate. The void size decreased significantly in the struts. After compression, however, many more large voids nucleated within the sample. There are two possible reasons for the appearance of these voids. Either a) the voids were already present but under the voxel threshold for the detection algorithm; they grew during compression due to dislocation coalescence or b) compression of the sample caused delamination of the additive layers, leading to void nucleation.

An example microstructure from one sample can be seen in Figure \ref{microstructure}, featuring several pores that are smaller than the minimum detectable voxel size used by the XCT software. This gives good reason to suspect that the voids were already present in the material and grew in size during loading.


\paragraph{Texture Mapping}
The heat transfer through a bulk piece of Ti-5553 and a thin lattice strut, oriented at 45$^\circ$ relative to the build substrate, may be different. Due to the geometry of the samples it was not possible to rotate the struts to obtain multiple diffraction patterns at different orientations. Therefore texture maps were obtained using a single diffraction image. 

Texture was fit using spherical harmonic functions in GSAS II. No assumptions about texture symmetry were made. Once the spherical harmonic coefficients were fit, the data was exported from GSASII and imported into the MTEX software for Matlab. MTEX was used to generate an orientation distribution function, which can be seen in Figure \ref{texture}. Texture maps were taken for both a strut in the sample as well as through a bulk piece of the sample. The bulk exhibits a weak fiber texture in the build direction, shown in Figure \ref{P3}, which is in agreement with previously characterized textures for additively manufactured Ti-5553 \cite{Schwab2016}. Likewise, the strut also showed a weak fiber texture, Figure \ref{strut}, though the texture is concentrated into several poles instead of being dispersed like the texture of Figure \ref{P3}.


%~
\paragraph{Digital Image Correlation}
Digital image correlation was performed on samples using Correlated Solution's VIC 2D software. Samples were lit during compression using a white light source; no surface modifications, such as speckling, were used. The software computed the 2D deformation using a subcell size of 23 pixels. The crosshead displacement can be seen, in $\mu$m, listed at the top of each load step image in Figures \ref{e11} and \ref{e22}.

The magnitude of the $\epsilon_{22}$ strains are higher than the $\epsilon_{11}$ strains which is to be expected because the $\epsilon_{22}$ strains are in the compression direction. The majority of the $\epsilon_{11}$ strains are in the nodes of the sample. As the transverse struts begin to stretch they push the central nodes outwards causing a significant buildup of strain. Equal magnitude strains can also be seen in the nodes and horizontal members in Figure \ref{e11}. By comparison, the $\epsilon_{22}$ strains build up mostly in the vertical members near the central nodes. These members are in the direction of the load and therefore receive the highest magnitude strains.

The biggest takeaway from the DIC analysis is that the 45$^\circ$ members experience a complex mix of strains. Neither the $\epsilon_{11}$ nor the $\epsilon_{22}$ completely capture the behavior of these struts. Both tensile and compressive strains can be observed. This motivates the use of multiple coordinate systems for exploring the behavior of these complex stress states, as discussed in following sections.
%~~~~~~~~~~~
\subsection{Results}
\subsubsection{Stress and Strain Measurements with HEXRD}
The strain in the sample at 50$\mu$m of crosshead displacement can be seen in Figure \ref{50um_principal_strains}. The strains are shown in three different coordinate systems: the laboratory coordinate system, the strut coordinate system, and the principal coordinate system. The laboratory coordinate system is the $(x,y,z)$ coordinate system shown in Figure \ref{expsetup}. The strut coordinate system are the strains oriented along (parallel to) the strut direction. The principal coordinate system is the $\epsilon_{11}$ strains calculated from Equation \ref{strainmodel} or the principal tensile strains. Figure \ref{50um_principal_strains}c also has arrows showing the orientation of the principal strain direction $\psi$ relative to the $x$ axis in the laboratory coordinate frame. Some tensile strains can be seen building up in the transverse members in all coordinate frames, with a very slight compressive strain building up in the 45$^\circ$ members. The principal strain direction in the transverse struts is mostly down the strut direction with some slightly off-axis behavior. The principal strain direction in the $45^\circ$ members is more complex. The principal strain direction is typically either down the strut direction or $90^\circ$ to it. 

By comparison, the distribution of stress and strain across the sample at peak elastic load, before fracture occurred, can be seen in Figure \ref{200um_principal_strains}. 

In every reference frame the horizontal members are dominated by tensile strains which is to be expected from struts that are transverse to the load and therefore should be in pure stretch. However, the struts sitting at 45$^\circ$ to the loading axis demonstrate a noticeable anisotropy in the distribution of load, particularly in Figure \ref{200um_principal_strains}a. In particular, the central node shows a fairly high compressive strain in the strut to its left, while the same strain is missing from the struts to the right. In the struts to the right there is a buildup of tensile strains near the center of the strut, which is likewise missing from the mirroring strut on the left side of the sample.

Figure \ref{200um_principal_strains}c shows the $\epsilon_{11}$ strain as well as the orientation of $\epsilon_{11}$ relative to $0^\circ$ in the laboratory coordinate system. For the transverse struts the orientation of the strain is entirely down the strut direction, with a few locations slightly off-axis. This is to be expected for a member in stretch. 

For the $45^\circ$ members, however, the principal tensile direction changes from location to location. On the right half of the sample the tensile strains are primarily in the strut direction, while for the left side of the sample they vary from being in the strut direction, to being $90^\circ$ rotated off the strut, to being another orientation entirely.


The two horizontal struts demonstrated near identical behavior during loading. The samples gain a positive tensile stress as they stretch and then, eventually, both return to a near-zero stress state, followed by a negative stress and a reduction in strain.

The stress in the sample, shown in Figure \ref{200um_stress}, presented unexpected results. The stress was calculated using Equation \ref{sigma} from the principal stresses in Figure \ref{200um_principal_strains}c. Figure \ref{200um_stress}a and \ref{200um_stress}b show $\sigma_{11}$ and $\sigma_{22}$ respectively. In many locations in the sample the character of the stress is the same for both $\sigma_{11}$ and $\sigma_{22}$ which was not expected. Further analysis of the fit revealed that the tr($\epsilon^\ell$) term of Equation \ref{sigma} dominated over the $\epsilon_{ii}$ terms, causing the larger magnitude strain to dominate the stress behavior. 


\begin{figure*}
	\includegraphics[width=1\linewidth]{/Users/njohnson/git/thesis/document/chapters/single_lattice/figures/50um_principal_strains}
	\caption{The principal strains and orientation of the principal coordinate system at a macroscopic displacement of 50um on the sample.}
	\label{50um_principal_strains}
\end{figure*}

\begin{figure*}
	\includegraphics[width=1\linewidth]{/Users/njohnson/git/thesis/document/chapters/single_lattice/figures/200um_principal_strains}
	\caption{The principal strains and orientation of the principal coordinate system at a macroscopic displacement of 200um on the sample.}
	\label{200um_principal_strains}
\end{figure*}

\begin{figure*}
	\includegraphics[width=1\linewidth]{/Users/njohnson/git/thesis/document/chapters/single_lattice/figures/200um_stress}
	\caption{Principal stress in the sample with a) showing the $\sigma_{11}$ stress and b) showing the $\sigma_{22}$ stress.}
	\label{200um_stress}
\end{figure*}


\subsubsection{FEA Model Without and With Residual Stresses}
\begin{figure*}
	\includegraphics[width=1\linewidth]{/Users/njohnson/git/thesis/document/chapters/single_lattice/figures/fea_no_stress}
	\caption{Results of the FEA simulation tested using only the material geometry and standard material properties for Ti-5553.}
	\label{fea_no_stress}
\end{figure*}

\begin{figure*}
	\includegraphics[width=1\linewidth]{/Users/njohnson/git/thesis/document/chapters/single_lattice/figures/fea_with_stress}
	\caption{Results of the FEA simulation using the material geometry, the standard material properties for Ti-5553, and a gradient of residual stress applied across the sample in the build direction.}
	\label{fea_with_stress}
\end{figure*}

A finite element model of the sample geometry was run to compare the distribution of stresses and strains with the diffraction information. First, the model was run on just the sample geometry as shown in Figure \ref{fea_no_stress}. The plots in Figure \ref{fea_no_stress} are in the $(x,y,z)$ coordinate system as with Figures \ref{50um_principal_strains}a and \ref{200um_principal_strains}a. The behavior of the sample is as predicted relative to the loading direction. The transverse struts exhibit a high tensile stress because their loading direction is directly down the strut, outwards.

The 45$^\circ$ struts exhibit more complex strain states. In the $x$-direction they have a small compressive strain as the nodes are bulging outward in the $x$-direction and pulling the struts with them. In the $y$-direction, the loading direction, a larger compressive strain can be observed. The topmost and bottommost nodes are experiencing a small tensile strain however.

Of note, the FEA model on only the geometry did not reproduce the same anisotropy in distribution of strains as that observed for the diffraction data. 

Because residual stresses were observed in the diffraction data it was hypothesized that they played a role in the anisotropy of the load distribution. As such, the FEA model was re-run with a gradient of residual stress applied in the build direction.

The FEA model with residual stress qualitatively agrees with the diffraction results as shown in Figure \ref{fea_with_stress}. The transverse struts are still in a pure stretch mode and exhibit a high tensile strain. The 45$^\circ$ struts, however, now exhibit a more complex strain distribution. Struts on the left side of the figure show a small tensile strain and a relatively high compressive strain near the central node. The left side of this figure corresponds to build locations nearest the build substrate. On the right side, however, the behavior is not mirrored. Instead, a mix of compressive and tensile strains build up. It is important to notice that the location of the compressive and tensile strains on the left and right sides of the sample match the same locations on the diffraction results.

%%%%%%%%%%%%%%%%%%%%%%%%%%%%%%%%%
\subsubsection{The Presence of Secondary Phases}
\begin{figure*}
	\includegraphics[width=1\linewidth]{/Users/njohnson/git/thesis/document/chapters/single_lattice/figures/secondary_phases}
	\caption{A secondary phase was observed present in the samples. Diffraction spots indicative of the secondary phase can be observed in a) and b). A logarithmic intensity plot of an integrated diffraction pattern demonstrates the presence of unexpected peaks, highlighted with red arrows in c). The same plot can be seen in linear intensity in d).}
	\label{secondary_phases}
\end{figure*}

An unexpected secondary phase was observed at some locations in the samples. Examples of diffraction peaks from the secondary phase can be observed in Figure \ref{secondary_phases}a and b. The presence of these peaks were primarily observed in the bulk regions of the samples such as at the nodes or in the top and bottom solid sections that mated with the compression platens of the load frame. It is possible that the phase was also present in the strut sections of the sample but there were too few grains for observable peaks to be found. Integrated patterns of the diffraction images in Figure \ref{secondary_phases}.c) reveal low-intensity, broad peaks that are indicative of a small grain size.

Several hypotheses were put forward for the structure of the secondary phase including HCP $\alpha$ titanium, hexagonal martensitic $\omega$ titanium, as well as precipitate phases of Titanium and one of the alloying elements. However, a literature review revealed that a face-centered orthorhombic titanium phase has been observed for Ti-5553 when it undergoes the same heat treatment as used in this study. Zheng et al. performed a TEM study on wrought Ti-5553 and discovered numerous secondary phases, including the secondary phase dubbed $O''$ \cite{Zheng2016}. Zheng solved the structure and found it to be face-centered orthorhombic.

Several phases were fit to fully integrated diffraction patterns to determine the structure of the secondary phase. The phases fit included the $\alpha$, $\omega$ and $O''$ phases. The best fit obtained was using only the $O''$ with the starting lattice parameters reported by Zheng et al., with a final weighted residual of 9\%. The lattice parameters were refined and found to be $a = 3.277\AA$, $b = 4.564 \AA$, and $c = 13.903 \AA$. 
%~~~~~~~~~~~~~
\subsection{Discussion}

\begin{figure*}
	\includegraphics[width=1\linewidth]{/Users/njohnson/git/thesis/document/chapters/single_lattice/figures/fractography.png}
	\caption{Fractography of the first failure site on a horizontal strut. The strut shown is transverse to the loading direction of the sample. Analysis of the fracture site shows a typical cup-and-cone failure. It does not appear that fracture initiated on a failure site such as an unmelted particle particle or a void.}
	\label{fractography}
\end{figure*}

\begin{figure*}
	\includegraphics[width=1\linewidth]{/Users/njohnson/git/thesis/document/chapters/single_lattice/figures/annotated_fractography}
	\caption{Higher magnitude image of the first failure site, a fracture which occurred in pure stretch. Both macroscopic dimpling can be observed, as well as fracture sites where microvoid coalescence led to cleavage.}
	\label{annofrac}
\end{figure*}

\subsubsection{Anisotropy in Stress Distribution Due to Residual Stress}
The anisotropy in mechanical response across the sample indicates that the build process impacts the mechanical properties of the material, as well as the geometry. In an ideal scenario, strain would be symmetrically distributed across members of like orientation and distance from the loading surface. The left-right anisotropy in the sample matches the build direction of the printing process. The (001) texture in the build direction, shown in Figure \ref{texture} also indicates build anisotropy. Orientation dependence of grain structures relative to the build direction has been widely reported for laser powder bed fusion additively manufactured materials \cite{Hayes2017,Keist2020, Todaro2020}.

The anisotropy in load distribution is also evident in Figure \ref{200um_principal_strains}c. The transverse struts both show load being concentrated down the strut direction. On the top left hand side of the figure eight, the strains are oriented either down axis or at 90$^\circ$ relative to strut. On the upper right hand side, the struts are almost entirely oriented down the strut. The bottom left hand side and right hand side show even more complicated loading directions; in some cases the load is distributed neither down the strut nor perpendicular to it. 

This anisotropy in the loading directions and distributions of loads likely played a role in the failure locations and order of failure of the sample. The transverse struts build up the most strain the fastest and thus failed first, in plastic yielding as explained in a previous section. Further testing is required to associate the build direction -- as shown in Figure \ref{build_direction} -- with the magnitude and direction of anisotropy in strain distribution. 

Because the samples were loaded quasi-statically and because diffraction characterization can only measure elastic behavior, it was not possible to measure the yield strength from diffraction. However, the DFA provides a prediction of the plastic yield load as
\begin{equation}
	P_Y = \pi a^2\sigma_Y
	\label{yield}
\end{equation}
where $a$ is the radius of the strut (250$\mu$m in this case) and $\sigma_Y$ is the yield strength of the material. Schwab et al. found a yield strength of 0.8 GPa, therefore the plastic yield load of the octet truss should be around 50 kN, a full two orders of magnitude higher than the load at which the sample fractured. 

The elastic buckling stress can be calculated as
\begin{equation}
	\sigma_Y = \frac{k^2\pi^2E_s^2}{12}\left(\frac{t}{\ell}\right)^2
	\label{bucklingstress}
\end{equation}
where $t$ is the strut thickness, $\ell$ is the strut length, $E_s$ is the Young's modulus of the solid, and $k$ is a constant based on if the strut ends are pin-jointed or fixed. In this case the struts are fixed and $k=2$ \cite{Dong2015}. The elastic buckling stress for these parts is 2.69 GPa, which is considerably higher than any stress measured for the samples, as shown in Figure \ref{200um_stress}. Therefore the samples are most likely to fail in a pure stretch mode, which was the case for this sample.

\subsubsection{Failure Occurring in Pure Stretch}
Digital image correlation partially reveals the deformation behavior of the structure at the surface level. The displacements shown in Figure \ref{e22} indicate that the majority of the surface displacements were concentrated in the $y$ direction at first. Around $180um$s displacement of the middle nodes begins to occur in the $\epsilon_{11}$ direction. 

It is clear from comparing magnitudes in Figures \ref{e11} and \ref{e22} that the majority of the load was concentrated in the loading direction, despite microstructural texture and the anisotropy of residual stresses in the sample. This gives good reason to believe that the horizontal struts through the middle node would be experiencing a purely stretch dominated behavior, as predicted by the DFA model. The outward bending of the horizontal nodes shown at $180\mu$m, $230 \mu$m, and $330\mu$m further demonstrates this because if the nodes are moving outward they are stretching the horizontal members in addition. 

Fracture occurred before buckling in the sample. The Deshpande-Fleck-Ashby model \cite{Deshpande2001} predicts for a perfect material that plastic failure in stretch should occur first, followed by elastic buckling. The first failure to occur in this sample was fracture in a stretch-dominated strut, followed by a second fracture in the other stretch-dominated strut, followed by elastic buckling at the uppermost nodes. The buckling behavior can be seen in the last frames of Figures \ref{e11} and \ref{e22}.

Fractography of the first fracture site, shown in Figure \ref{fractography} reveals that the first fracture occurred due to plastic yielding, as shown in Figure \ref{annofrac}. The fracture site exhibits several different features which are labeled in Figure \ref{annofrac}. The non-circular, reduced cross section shown in Figure \ref{annofrac} is a likely reason why this site failed first as opposed to other locations in the strut. The reduced cross section likely caused a buildup of stress; this is in agreement with the strain behavior shown for the horizontal struts in Figure \ref{200um_principal_strains}. Dimpling can be observed in the regions labeled `ductile' while striations on other facets indicate a different kind of failure mode. However, high magnification fractography revealed that even on flat surfaces with slip plane striations microvoids can be seen. 

The microvoid behavior is in agreement with the pre- and post-mortem CT scans that are shown in Figure \ref{voids}. It is likely that at the flat surfaces in Figure \ref{annofrac} voids existed in the material before compression and grew to larger sizes during loading. This coalescence of voids ultimately led to failure under plastic yield. 

\subsubsection{Microstructural Evolution}
Microstructure is perhaps a major component missing from homogenization schemes of lattice materials and the investigation herein reveals that the additive process has a pronounced effect on the microstructure. The void behavior in the samples is particularly concerning as the density of the sample changed from $99.95\%$ in the as heat-treated condition to $99.61\%$ in the deformed condition. Based on the size of pores seen in Figure \ref{microstructure} it is likely that pores existed below the voxel threshold of the XCT software and grew in size during compression. While it is unlikely that this had an effect on the mechanical properties in this studies is likely going to play a role in fatigue-limited applications like biomedical implants.


\section{Conclusions}
The mechanical response of additively manufactured Ti-5553 octet truss lattice structures was investigated. The material responded anisotropically to compressive load due to residual stresses that were present from the build process. This was apparent in both HEXRD testing and confirmed by an FEA model.  The primary direction of loading during compressive is down the strut direction, with some locations demonstrating off-axis loading. The failure mechanism of the sample was plastic yielding first, followed by elastic buckling, as predicted by the DFA model. Microvoid nucleation lead to ductile fracture in the sample, with some cases of dimpling occuring and in other cases microvoid coalescence causing cleavage. This is likely due to dislocation buildup which commonly occurs in AM materials.

\section*{Acknowledgements}
We would like to acknowledge a gracious graduate research fellowship from Los Alamos National Laboratory.


\bibliography{bib.bib}
\end{document}


