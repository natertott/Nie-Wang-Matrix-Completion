\vspace{-2em}
\paragraph{Motivation}
\begin{figure*}[t]
	\includegraphics[width=1\linewidth]{/Users/njohnson/git/thesis/document/chapters/review/Images/Fig1_AMgene.png}
	\caption{The design space of metals additive manufacturing spans many engineering disciplines since the material and part are made at the same time. As shown in this schematic, alloys, parts, and manufacturing process designs are concurrently considered in the "pre-build" phase. The physics of the process itself may then be modeled, including feedstock dynamics, thermodynamics and kinetics of melting, solidification, and thermal histories, which dictate the final microstructure. Today, post processing treatments are typically performed as secondary processes, though the future points to ``hybrid manufacturing" processes where they are also incorporated at the point of fabrication.}
	\label{AMgene}
\end{figure*}

Metals additive manufacturing (AM) has created a paradigm shift in they way metal components are manufactured; materials and parts are fabricated simultaneously using a single machine, highly complex geometries are possible, and local variations of microstructure-property relationships may be realized through local process variations. Although decades of scientific and engineering work in industry, academia, and government have resulted in the commercialization of metals AM technologies, the consistency and quality of parts and materials are still open challenges for many applications. In recent decades, Integrated Computational Materials Engineering (ICME) approaches have proven to accelerate the development and adoption of materials technologies \cite{Panchal2013}. Traditionally, ICME approaches incorporate physics-based experimental data with simulations that span different length and time scales. However, for metals AM, much of the physics are still being discovered; hence, the development of comprehensive, computationally feasible physics-first approaches to ICME are still an open challenge. The diverse array of promises and problems in AM has resulted in a field of study that is rich with data -- so much so that our ability to store and analyze the data is challenged. At the same time, this wealth of data is motivating a paradigm shift to incorporate machine learning into ICME approaches. 

\subsection{Background}
The 20th century saw the maturation of materials science and engineering as a field of study, enabling targeted materials discoveries and innovations for specific applications. Over the past several decades, materials development cycles have greatly accelerated by formulating materials problems through the process-structure-property-performance paradigm \cite{Olson2000, Panchal2013}. 

The process-structure-property-performance (PSPP) paradigm is a core philosophy in materials science and engineering that governs how the manufacturing of a material determines its ability to be used in different engineering applications. The PSPP relationships break down materials development into four key areas of scientific and engineering interest \cite{Olson2000}.

In AM, the \textbf{processing} of a material is dictated by the thermal, mechanical, and chemical changes experienced during its manufacture. Controllable machine parameters like energy density of the heat source, the path in which material is deposited or fused, the order in which part layers are manufactured, or the location of parts on the build plate are determining factors of the material process history. Table \ref{table:design_space} shows many of the controllable parameters common to laser-based additive manufacturing systems. The choice of these parameters largely impacts the processing history. The true processing history, however, is better described by the thermal history of the build volume, both during manufacture and post-processing, the mechanical forces it experiences, and any chemical reactions that occur in or on the part. Processing \textit{routes} are often discussed in AM and typically refer to beneficial or detrimental processing histories that impact the part's structure.

The \textbf{structure} of a material is a wide-ranging concept that spans many length scales. Structure can refer to the crystallographic structure at the atomic scale, to the morphology and orientation of grains at the mesoscale, to the geometry being manufactured at the macroscale. Microstructure is a term often used in materials science referring to a specific subset of the material structure. Microstructure for metals most commonly refers to grain and sub-grain level information like material phases, grain morphologies, texture, and any defects like pores or dislocations that might be present. Microstructures are often considered in analysis of material structures because they fundamentally dictate a material's properties. 

The \textbf{properties} of a material are characteristics that determine its qualities. Properties of metals AM parts that have been of interest are wide-ranging and they vary depending on the desired engineering application of the part. Mechanical properties are some of the most studied for AM metals since the majority of metals applications are structural. Other properties of interest include thermal conductivity, which determines the heat transfer through an AM part, chemical properties, like corrosion resistance, and optical properties, like reflectivity. 

The \textbf{performance} of a part is its ability to be successfully implemented in an engineering application. Performance can be viewed through the lifetime of an AM part when subjected to the mechanical, thermal, chemical, etc., forces it will experience. Early additively manufactured alloys showed degraded-to-comparable static properties compared to traditionally manufactured alloys \cite{Spierings2013}. Further research and development improved the static properties of AM materials, yet high microstructure variability and defect density can still cause AM material to fail unexpectedly in fatigue limited applications \cite{Wycisk2014, Edwards2014}. Some recent AM developments have resulted in material properties that exceed those of traditionally manufactured materials\cite{Probstle2016,Gallmeyer2019, Martin2017, Wang2017a, Liu2017a, Zhu2018}. Ultimately AM processes are unique relative to other metal fabrication techniques and it is difficult to make fair comparisons regarding performance across various manufacturing methods. When properly designed, AM parts can meet the intended performance needs in a wide variety of end-use applications. The large combinatorial space of manufacturing options in AM often obfuscates how proper design choices can be made.

The materials scientist interacts with the process-structure-property paradigm in traditionally manufactured materials. Traditional material manufacturing can be phrased in a cause-and-effect relationship between process, structure, and property. Once the material has been developed and characterized by the materials scientist or engineer, another engineer then considers the property-performance linkage. Since material is made separately from an engineered part in traditional manufacturing, the PSPP paradigm can be broken up into these two separate sets of relationships. In AM, the material and the part are made simultaneously. Simultaneous material-part manufacturing motivates consideration of linkages across the entire PSPP paradigm. The ICME approach to materials science is focused on modeling, bridging, and predicting relationships throughout the PSPP paradigm.

Computational materials science and engineering has enabled the prediction of microstructure from processing and of properties from microstructure, reducing the need for costly and time consuming experimentation in discovering or developing a new material and/or its manufacturing. Today, ICME approaches tightly integrate physics-based computational models into the industrial design process, allowing the desired performance requirements of a part to guide the design of a material. Alloy specific examples include low-RE Ni superalloys for better turbine performance \cite{Pollock2016} and lower cost and radioactive element free Ferrium S53 alloy designed for corrosion-resistant landing gears \cite{Olson2014}. Both cases reduced materials innovation timelines from decades to years, demonstrating the practical capability of designing and qualifying new materials within an industrial product development cycle. Generalizing and accelerating this capability across different industries and materials is a primary goal of the Materials Genome Initiative (MGI) \cite{MGI}.

Predicting PSPP linkages in metals AM is difficult with existing physics-based ICME approaches. The physics of AM processes are more complex than traditional fabrication methods, like casting, as they involve rapid solidification, vaporization and ingestion of volatile elements, and complex thermal history that consists of dozens of heating and cooling cycles, each one different. Furthermore, all of these additional complexities vary from one location to another within a part, and from part to part within a build volume. For AM, physics-based ICME tools have been mostly developed through attempts to adopt legacy manufacturing models to AM data, with some success. However, today's relatively low cost and time for performing AM processing experiments has led to metals AM development being largely combinatorial, with a chief strategy of adopting AM processing to legacy alloys that were developed for other types of manufacturing using extensive design of experiments.

It is with awareness of the large amounts of data being generated in AM through these combinatorial development cycles that machine learning (ML) has been targeted to accelerate AM innovations and their commercialization. Machine learning as a technology development accelerator has shown wide application in recent years across fields including finance \cite{Bose2001}, molecule design for genomics, chemistry and pharmacology \cite{Gomez-Bombarelli2018}, social networking \cite{Brusilovsky2007} and, most relevant to this review, materials science and engineering \cite{Wagner2016, Ramprasad2017, Butler2018}. Still, the use of ML in materials science was relatively limited for a variety of reasons, especially the lack of large curated datasets amenable to existing ML methodologies. Through the work done under the MGI, this data limitation was identified as a primary impediment to future materials innovations \cite{MGI}. In response, there has been significant recent investment in materials database developments to better enable materials data informatics innovations. It is now recognized and accepted that ML frameworks can couple legacy physics-based ICME tools with experimental data to produce more accurate process-structure-property models and to automate the iteration of designed experiments for model improvement and optimized materials \cite{Rajan2005, Agrawal2016, Butler2018, Ball2019, Druzgalski2020}.

We proceed to review how the paradigm shift from purely physics-based to coupled physics-based/data-driven ICME approaches can be made through solving metals AM challenges. We begin by phrasing terms and ideas from AM in ways that are compatible with machine learning. We provide a basic review of machine learning algorithms and how they can be applied to additive manufacturing. Following this introduction to using ML for AM problems, we review other uses of machine learning in materials science and engineering and state the uses of such approaches for solving AM challenges. 
