\subsubsection{Featurization of Qualitative Image Data}
The same processing algorithms that are used for featurization and modeling of in situ signals can also be applied to automate part of the scientific process of studying additive manufacturing. Specifically, computer vision can be used to automate classification of microstructures during parametric analysis.

Parametric analysis in additive manufacturing requires the characterization and measurement of material properties that result from a specific coordinate in the design space. Often, material properties like mechanical strength, surface roughness, microstructure, or defect density have to be measured, analyzed, and quantified or classified as part of the parametric analysis process. This experimental process can be tedious. More often than not, images are relied upon heavily in classifying material properties, especially microstructures. Fortunately, machine learning algorithms can be applied to automate the analysis of images during the parametric analysis process.

It is worthwhile to mention up front that these algorithms have been \textit{tested} on microstructure and, in some cases, additive-specific images. There are few algorithms that can process AM microstructure data `out-of-the-box.' Rather, these algorithms will need to be tailored in order to quantify AM images specifically. However, the algorithms discussed here have been proven on non-AM microstructure datasets, thus they should be extensible to AM datasets. The computer vision approaches that work for microstructure data are often the same approaches that will be discussed again later for in situ monitoring and feedback.

One AM-related application of image characterization is measuring particle size distributions in AM powder feedstock. DeCost and Holm used SIFT with a dictionary classifier to measure the particle size distribution for a dataset of synthetic powder particles \cite{DeCost2017a}. Particle size distribution plays in several steps across the additive process including energy absorption and part metrology \cite{Zhou2009, Boley2015, Boley2016}. DeCost created datasets with six different particle size distributions. Image features were identified and classified using $k$-means clustering on the features found by SIFT. Then, a classification algorithm known as a support vector machine (SVM) was trained to classify image features into particle sizes. DeCost was able to achieve $89$\% overall classification accuracy in measuring particle size distribution this way. DeCost later improved upon this powder classification method and were able to achieve higher classification accuracies for real powder images \cite{DeCost2017}. Machine learning algorithms have also been trained for the automatic classification/identification of EBSD texture maps \cite{Shen2019, Kaufmann2020}.

\begin{figure*}
	\includegraphics[width=1\linewidth]{/Users/njohnson/git/thesis/document/chapters/review/Images/Fig14_MiyazakiSegmentation}
	\caption{The image segmentation approach implemented by Miyazaki et al. to automatically segment, classify, and characterize SEM images of Ti-6Al-4V microstructures. a) First, the SEM images are obtained. b) A random forest algorithm is used to classify regions of the image and c) build a database of classified images. d) Image segmentation proceeds to separate out the $\alpha$ and $\beta$ phases. e) An ellipse approximation is overlaid on the segmented image to characterize grain morphology and size. f) The nearest neighbor distance can be calculated from the ellipse locations to provide a measure of grain distribution. Microstructures can be very complex for additively manufactured alloys and performing this characterization by hand becomes burdensome. Image recognition algorithms can automate the process and significantly speed up characterization, development, and qualification times. Image reproduced from Miyazaki et al\cite{Miyazaki2019}.}
	\label{MiyazakiSegementation}
\end{figure*}

Strides have been made in automatically identifying and quantifying information from metallographs \cite{DeCost2015, DeCost2017b, Ling2017a, Bulgarevich2018}. A good portion of quality control in materials science as a whole, not just AM, involves classifying materials based on metallographs or micrographs of microstructure. Work is being done across materials science to apply machine learning based computer vision to classifying and quantifying information in these microstructural images. Doing so will speed up the process of materials characterization and qualification, while also providing methods of quantifying information that otherwise would have stayed in a qualitative form. Examples include classification of grain structures, measurements of grain size, pore size calculations, and more. 

An additive-specific image segmentation algorithm was used by Miyazaki et al. \cite{Miyazaki2019}. Five image filters were convolved with microstructure images of selective laser melted Ti-6Al-4V. The features identified by these filters were used in a random forest algorithm to segment the image into regions of $\alpha$ phase grains and $\beta$ phase grains. The algorithm was able to automatically calculate area fraction of primary and secondary $\alpha$ phases that form during cooling. It was also able to calculate the nearest-neighbor distance between grains. Nearest neighbor distance of grains is indicative of grain characteristics like size, morphology, and distribution. 

Chowdhury et al. took a more expansive approach to performing feature identification in microstructures. In particular, they were looking to classify microstructures as either dendritic or non-dendritic. Chowdhury employed 8 different feature identification methods for a dataset of images. Classification was performed using an ensemble of ML techniques including support vector machines (SVM), Na\"ive Bayes, nearest neighbor, and a committee of the three previous classification methods \cite{Chowdhury2016}. Chowdhury's wide approach to image classification achieved classification accuracies above 90\%. 

Efforts are underway across materials science to implement computer vision for the automation of materials classification. Rather, the authors would like to refer the reader to the computer vision libraries listed in Table \ref{data_tools}. 

