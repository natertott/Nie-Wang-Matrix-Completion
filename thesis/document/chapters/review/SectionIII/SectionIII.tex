\paragraph{Current ICME Tools are Well Equipped to Integrate with an ML Framework}

The following section discusses how machine learning approaches can be used in current R\&D efforts in AM. This discussion includes how physics-based analyses, characterizations, and simulation methods may connect with different machine learning algorithms. Overall, the discussion is aimed at conveying how ML can be used to automate the generation of AM PSPP knowledge. Still, this article stops short of providing an exhaustive review of either machine learning algorithms or additive manufacturing. Instead, the intent is to introduce how ML approaches can be connected to AM research. The algorithms that are discussed were chosen because they were previously demonstrated in a materials science and engineering application \textbf{or} because the possible application of an algorithm to AM was clear and immediate. Similarly, the additive manufacturing problems addressed are not all-encompassing; they are merely a few that may be immediately addressable with machine learning approaches. 

%~%~%~%~%~%~%~%~%~%~%~%~%~%~%~%~%~%~%~%~%~%~%~%~%~%~%~%~%~%~%~%~%~%~%~%~%~%~%~%~%~
\subsection{Experimental Methods and Manufacturing Design}
\subsubsection{Alloy Design and Feedstock Selection}
\begin{figure*}
	\includegraphics[width=1\linewidth]{/Users/njohnson/git/thesis/document/chapters/review/Images/Fig6_AlloyDesign}
	\caption{An illustration of the alloy design process using a genetic algorithm. First, a target property $P_\text{target}$ and an evaluation method $f(\mathbf{X})$ for the alloy $\mathbf{X}$ are chosen. The evaluation method is most often a material modeling approach that can predict material properties based on composition. Then, a population of starting compositions are made. The model is run for each composition and an associated material property is measured. The predicted values are compared against the target value. If no material matches the target, then the genetic algorithm begins. The closest-matching compositions are selected to create a child generation. Crossover and mutation occurs for those compositions that were selected. In this way, a new population of compositions are created that are similar to the best-performing compositions from the previous generation. Model assessment and the genetic algorithm are then run again until a composition is found that meets the target property value.}
	\label{fig:GA}
\end{figure*}
Choice of alloy impacts the physics of AM from start to finish, ranging from the interactions of energy sources with material feedstocks to the performances of the final parts. For example: the reflected vs. absorbed intensity of lasers on powder beds is determined by the powder's composition \cite{Boley2016, Trapp2017}; the density of feedstock, both intra- and inter-granular density, plays a role in final part density \cite{Bi2013}; conduction modes in the melt are partially determined by the thermal properties of the alloy \cite{Martin2017}; and different alloys exhibit different solidification kinetics, which can lead to drastically different microstructures after manufacture \cite{Collins2016}. Problems in the additive process can also be linked to composition such as vaporization of constituent elements due to rapid thermal fluxes, impacting the stoichiometry of melt pools and, ultimately, quality \cite{Brice2018}. These can be different for different feedstock types (e.g., wire vs. powder), even for the same alloy choice. Wysocki et al. discuss the differences between different additive manufacturing processes for titanium alloys: electron beam, laser based, powder, wire, etc\cite{Wysocki2017}. Some studies have also investigated the impact of feedstock properties like particle size distribution and morphology on process quality \cite{Slotwinski2014, Strondl2015, Trapp2017}, although the direct impacts have not been fully resolved.


As such, alloys developed for traditional metals manufacturing techniques such as casting, rolling, extrusion, etc. sometimes need to be altered to improve AM processing. In the best cases, alloys developed for AM may outperform traditionally manufactured alloys. For example, unique strengthening mechanisms can result from AM processing \cite{Brice2018, Wang2017, Martin2017, Gallmeyer2019}. Designing alloys for AM -- either altering the chemistries of known alloys or discovering new alloys -- requires considering the implications of the physical properties of alloys with AM processing. An understanding of what trend in a physical property is ``better" or ``worse" for AM processing is still an open area of research. Hence, while information about the physical properties of different alloys has been collated into databases that are compatible with design for AM, models and optimization targets for mining those databases to extract candidate alloys for AM are still being developed and verified. 

Existing databases contain alloy properties ranging from the reflectivity to the mechanical properties. The International Crystal Structure Database (ICSD) contains the crystal structures of millions of compositions. The Linus Pauling files contains a range of material information, from atomic properties like radius and electron valency to crystallographic level information \cite{Villars1998}. More modern databases such as AFLOWLib \cite{Curtarolo2012a} and the Materials Project \cite{Jain2013} allow users to interactively search across different types of alloy information. Searching through large databases of information to find optimal compositions for manufacturing is actually one of the earliest materials informatics problems ever addressed. Methods exist to perform these searches in a fast, automated way. These methods are referred to as data mining, a data-driven materials design approach.

Data mining has been demonstrated to be useful for AM alloy development. Martin et al. used such an approach to modify the chemistry of aluminum alloys to make them process better during LPBF\cite{Martin2017}. The first step in a data-driven design process is to identify which alloy properties are important to the desired application. Laser powder bed fusion of Al alloys had been plagued by sparse nucleation of grains. The result was that large grains formed during AM together with large intergranular stresses, the combination of which resulted in hot-cracking. To overcome this problem, Martin searched for candidate grain inoculant compounds that could form through chemical reactions during LPBF. Searching for grain-refining nanoparticles has improved solidification properties\cite{Neuchterlein2016}. For example, silicon and carbon could react to form SiC particles that would force more homogeneously, densely packed grain nucleation throughout the material. However, if such compounds had lattices that were dissimilar to those of the aluminum alloy, large stresses could form at the interface of the inoculants and the alloy matrix, still leading to cracking. Hence, they searched not only for potential inoculants, but more specifically for inoculants with crystallographic lattice parameters that closely matched those of the base aluminum alloy. Martin's study employed a search algorithm to search through 4,500 different possible nucleants and identify those with the closest-matching parameters. Ultimately, hydrogen-stabilized Zr was found to be the best candidate.

The same database mining process employed by Martin -- identify the target properties, then search for the closest match -- can be extended to many AM problems as well. Database mining was first introduced in materials science to predict stable compositions, or estimate material properties from composition. Database mining has been successfully implemented to predict stable crystal structures \cite{Franceschetti1999, Fischer2006, Oganov2006} and predict material properties as a function of composition \cite{Ikeda1997, Gopakumar2018, Wu2018, Kirklin2013, Setyawan2011}. Some specially designed search algorithms have also been designed for improved speed in automated searches \cite{Wolf2000}. Successes have been found in designing Heusler compounds using high throughput search methods \cite{Roy2012}. Several reviews exist detailing early high-throughput searches for compositions with ideal properties\cite{Gilmer1998, Koinuma2004}. The same search algorithms employed in these studies can be extended to AM cases.

A limitation of database mining is that searches are limited to previously measured and/or calculated properties. Generally, information about the vast space of \textit{all possible} materials is unknown. Traditional materials science and engineering approaches would turn to explicitly calculating or measuring the unknown points of interest, one at a time. Searching through compositions may be accessible for manufacturing processes like thin-film deposition where the composition can be adjusted continuously and with several species at once using well established methods. A combinatorial study of compositional changes for AM feedstock is hindered by the difficulty and expense of producing feedstock.

For example, consider the cost of combinatorially alloying Ti with alloying elements $\{\text{Al}, \text{V}, \text{Zr}, \text{Cr}, \text{Hf}\}$ and then testing printability. Explicitly creating all possible combinations of $\{\text{Ti},\text{Al}, \text{V}, \text{Zr}, \text{Cr}, \text{Hf}\}$ is feasible if using a coarse set of level choices for additions of alloying elements, but undesirable. There are 15,503 alloy combinations if alloying in steps of $1$ wt. \% up to $15$\% total alloying elements from the choices above.

However, using machine learning methods, the process of combinatorial exploration to find an optimal composition can be achieved without explicitly modeling each combination. For example, \textit{genetic algorithms} (GA) can be used to augment many physics-based models. Genetic algorithms have been one of the most-used data driven approaches in materials science over the past few decades \cite{Morris1996, Ho1998, Wolf2000, Johannesson2002, Stucke2003, Hart2005, Oganov2006}. The principle of genetic algorithms is to evaluate the \textit{fitness} of a population of candidate alloys against a \textit{fitness function}. The fitness function $f(\cdot)$ is a method of evaluating how well a candidate alloy meets a criteria. Often in materials science the fitness function is evaluated by running models that can measure a material property based on composition. Examples include identifying stable crystal structure of a composition using DFT\cite{Franceschetti1999, Oganov2006} and evaluating thermomechanical properties of an alloy using ThermoCalc \cite{Xu2008}. Some additive-specific models include the model of Tan, which predicts dendrite arm spacing from composition \cite{Tan2011}. The calculation of thermodynamic properties relevant to AM -- such as vaporization temperature, coefficient of thermal expansion, solidus and liquidus temperatures -- using the CALculation of PHAse Diagrams (CALPHAD) method \cite{Andersson2002} can also be a fitness function.  For the sake of alloy design a model must be able to predict a materials properties based on composition. In reality, however, models must also consider additional physics related to the composition, such as crystal structure, thermodynamic properties, interatomic potentials, and more.

In using a GA for alloy design, a desired target property value must be identified. This value $P_\text{target}$ is then formulated as a function of composition and process variables. Additionally, a method of measuring the property value as a function of composition and process variables $\mathbf{X}$ is needed; the models proposed previously (ThermoCalc, DFT, etc.) can serve as the evaluation step $f(\mathbf{X})$. The goal is to find a material whose measured property closest matches the desired target property, or

\begin{equation}
	\text{min} || f(\mathbf{X}) - P_\text{target} ||.
	\label{GAopt}
\end{equation}
As a thought experiment, consider various amounts of $\{\text{Al}, \text{V}, \text{Zr}, \text{Cr}, \text{Hf}\}$ alloyed into Ti. These are the \textit{genes} of the genetic algorithm. This is similar to a study completed by Li et al\cite{Li2017}. Once a fitness function has been identified, the next step in a genetic algorithm is to represent candidate alloys as a \textit{chromosome}. 

We can represent a chromosome as \\

% You must have an empty line between text and the start of a table for some reason -- this is definitely going to be a problem later
\begin{table}[h!]
\begin{tabular}{cccccc}
	$\mathbf{X}$ & $=$ & [$\chi_1$, & $\chi_2$, & $\ldots$, & $\chi_n$] \\
\end{tabular}
\end{table}
\noindent 
where $\chi_1$ is the species and weight percent of the first element (titanium, in this example), $\chi_2$ is the species and weight percent of the second element, up to $n$ elements. For example, Ti-6Al-4V would be represented as \\

\begin{table}[h!]
\begin{tabular}{ccc}
	 [0.9 Ti,  & 0.06 Al, & 0.04 V ] \\
\end{tabular}
\end{table}
\noindent
The goal is to find the alloy with optimal dendrite arm spacing. First, a population of candidate chromosomes needs to be generated, either randomly or by design. Two examples from a starting population may be \\

\begin{table}[h!]
\begin{tabular}{ccccc}
	Alloy 1 & $=$ & [0.9 Ti, & 0.05 Al, & 0.05 V ] \\
	Alloy 2 & $=$ & [0.9 Ti, & 0.1 Zr] & \\
\end{tabular}
\end{table}
\noindent
The chromosomes produced from this initial population will serve as inputs to the fitness function. 

Genetic algorithms select chromosomes out of the current population -- called the parent generation --  to proceed to another generation of model assessment -- called the child generation. Selection consists of keeping the best performing compositions, say the top $10\%$, and discarding the rest, as determined by Eqn. \ref{GAopt}. Genetic algorithms find optimal locations in the design space by relying on the similarity hypothesis. If one alloy is in the top $10\%$ of chromosomes then it is possible that a similar alloy will also be high performing -- it may even perform better. Once selection is done, the next step is to search the space near the best performing alloys from the parent generation.

Genetic algorithms generate similar compositions from those selected in the parent generation by making alterations to genes. One operation is \textit{mutation}, whereby genes are changed. For example, we could mutate alloy 1 by changing the composition:\\

\begin{table}[h!]
\begin{center}
\begin{tabular}{c|ccccc}
	\textbf{Parent Generation:} & Alloy 1 & $=$ & [0.9 Ti, & {\color{red}0.05} Al, & {\color{red}0.05} V ] \\ \hline
	\textbf{Child Generation:} & Alloy 1 & $=$ & [0.9 Ti, & {\color{blue}0.02} Al, & {\color{blue}0.08} V  ]  \\ 
\end{tabular}
\end{center}
\end{table}
\noindent where in the child generation the amount of V was increased, while the amount of Al was decreased. Another operation that may be performed is \textit{crossover} where genes are added or interchanged. For example, one crossover operation may look like

\begin{table}[h!]
\begin{center}
\begin{tabular}{c|ccccc}
	\textbf{Parent Generation:} & Alloy 1 & $=$ & [0.9 Ti, & 0.05 Al, & 0.05 {\color{red} V} ]  \\
						 & Alloy 2 & $=$ & [0.9 Ti, & 0.1 {\color{blue} Zr}] &              \\ \hline					 
	 \textbf{Child Generation:} & Alloy 1 & $=$ & [0.9 Ti, & 0.05 Al, & 0.05 {\color{blue} Zr} ]  \\
						& Alloy 2 & $=$ & [0.9 Ti, & 0.1 {\color{red} V}] &              \\ 
\end{tabular}
\end{center}
\end{table}
\noindent where in the second generation V and Zr have been interchanged.

Selection, mutation, and crossover followed by model assessment and further selection, mutation, and crossover continues until the design criteria is met. A schematic of the GA process can be seen in Figure \ref{fig:GA}. 

Genetic algorithms have been applied to alloy design for low and high temperature structural materials \cite{Ikeda1997, Kulkarni2004}, ultra high strength steels \cite{Xu2008}, specific electronic band gaps \cite{Dudiy2006}, minimum defect structures \cite{Anijdan2006}, exploring stable ternary or higher alloys alloys \cite{Hautier2010, Johannesson2002}, and more. Chakraborti et al. wrote a review on the application of GA's to alloy design through the early 2000s\cite{Chakraborti2004}.

In addition to genetic algorithms, other machine learning algorithms have also been applied to classify and optimize alloy compositions. Anijdan used a combined genetic algorithm--neural network method to find Al-Si compositions of minimum porosity \cite{Anijdan2006}. Liu et al. applied partial least squares to data mining of structure-property relationships across compositions \cite{Liu2006}. Decision trees, which are discussed in the next section, have been implemented for a number of different alloy optimizations, such as predicting ferromagnetism \cite{Landrum2003} and the stability of Heusler compounds \cite{Oliynyk2016}. In the search for new alloys, a wide range of machine learning algorithms can be implemented to guide the entire experimental design process so that an optimized property is found as quickly as possible. In the next section, we focus on using ML in design of experiments.
\subsubsection{Design of Experiments}
Design of Experiments (DOX) is the design of task(s) aimed at performing parametric analysis \cite{Dox2014}. Parametric analysis, broadly defined, is a method of mapping independent variables to corresponding dependent parameters. In materials science and engineering, process-property relationships are typically assessed using parametric analysis. Machine learning can reduce the number of experiments (i.e., tasks) needed to perform parametric analyses sufficient to characterize process-property relationships. Approaches such as \textit{sequential learning} model relationships in parametric studies to discover regions of the parameter space that produce the most information about process-property relationships. 

 In additive manufacturing research, process parameters such as laser energy, speed, build direction, composition, and layer height are varied to study their impact on material properties. Examples include relating build geometry to microstructure or surface roughness \cite{Antonysamy2013, Strano2013}, temperature history to microstructure \cite{Bontha2009, Nie2014}, substrate temperature to residual stress development \cite{Chen2016, Brice2018}, or even entire manufacturing processes to microstructure \cite{Baufeld2011}. Other commonly performed parametric analyses in AM relate heat source parameters to part temperature history \cite{Bontha2006, Li2014}, microstructure \cite{Cherry2015, Jia2014}, mechanical properties \cite{Delgado2012, Khorasani2018}, and residual stresses \cite{Wu2014, Denlinger2015}.

\textit{Information} in AM research is any observation of process-structure-property relationships. For example, observing that a set of laser parameters results in an equiaxed microstructure can be considered information because the researcher has gained an idea of the structure to expect from set processing conditions. Therefore, \textit{information gain} is any experiment that reveals a previously unobserved process-structure-property relationship. Rigorous mathematical definitions of information and information gain have been defined, typically referencing back to Shannon's original formulation of information theory \cite{Shannon1948}.

Both engineering and scientific investigations of AM utilize parametric analysis. In science, tasks designed for information gain are performed until parametric analysis results in a theory or model for a process-structure-property relationship. In engineering, tasks designed for information gain are performed until an optimality criterion is met, such as maximum strength or minimum porosity. Both disciplines vary independent parameters and measure dependent responses until enough information about the underlying phenomenon is known to complete the parametric analysis with some predetermined level of certainty, variance, and/or precision. 

Traditional DOX approaches maximize information gain from performing tasks by subdividing the design space \textit{a priori} to maximize the likelihood of information gain from task to task. In these approaches, all pre-determined tasks are performed before parametric analysis is attempted. In machine learning DOX approaches, parametric analysis is performed after each individual task, and the next task to perform is determined based upon a statistical metric of the parametric analysis - as such, the likelihood of information gain incrementally improves as each task is carried out, and usually only a fraction (20 - 60\%) of the number of tasks need to be performed to reach the established success criterion for the parametric analysis relative to the traditional DOX approaches \cite{Wigley2016, Ling2017a}.

For ML-based DOX, the first step is still to identify process-structure-property parameters of interest and to classify them as either inputs or outputs relative to the desired relationship that is to be determined, as is done in traditional DOX. As more parameters are added, the size of the design space grows. Once the scope of the design space has been defined, the next step is to generate an initial dataset (i.e., initial information). The first tasks can be designed with traditional DOX methods -- often, an approach as simple as selecting an initial uniform sample from the design space. In addition to generating an initial dataset, a \textit{response function} must be defined to interpret the relationship between the inputs and outputs. One example is a regression model of the process parameters (inputs) and material properties (outputs). A \textit{random forest} algorithm trains many regression algorithms, each on a subset of the experimental data. 

Random forest algorithms are ensembles of a type of simple regression algorithm called a classification and regression tree or a decision trees. Decision trees can be used for both classification and regression. Consider a design space that an engineer wishes to explore represented as a matrix, such as \newline

\begin{center}
\begin{tabular}{c|c|c|c} 
	Feature 1 & Feature 2 & Feature 3 & Property 1 \\ \hline
	$x_{1,1}$ & $x_{1,2}$ & $x_{1,3}$ & $y_{1}$ \\
	$\vdots$ & $\vdots$ & $\vdots$ & $\vdots$ \\
	$x_{n,1}$ & $x_{n,2}$ & $x_{n,3}$ & $y_{n}$ \\
\end{tabular}
\end{center}
where $x_{1,1}$ is the first parameter setting for feature 1, $x_{1,2}$ is the first parameter setting for feature 2, and $y_1$ is the first property measurement for the associated position in the design space, out of $n$ total measurements. This design space could be represented as a matrix by
\begin{equation}
	\mathbf{B} = \begin{bmatrix}
		x_{1,1} & x_{1,2} & x_{1,3} & y_{1} \\
		\vdots & \vdots & \vdots & \vdots \\
		x_{n,1} & x_{n,2} & x_{n,3} & y_{n} \\
	\label{Bmatrix}
	\end{bmatrix}
\end{equation}
The rows of $\mathbf{B}$ represent different observations in the design space and the columns of $\mathbf{B}$ are different parameters or properties. The goal of parametric analysis is to map different values of $x$ to a property $y$. 

Decision trees begin by taking a samples from the design space -- rows in $\mathbf{B}$ -- and computing a split in one of the features (columns) that best classifies the data point. Consider a set of three experiments that has feature-property $\left(x,y\right)$ pairings of $\left(0.1, 0\right)$, $\left(0.2,0\right)$ and $\left(0.3,1\right)$. The decision tree computes every possible partitioning of $x$ and computes a misclassification error called the Gini impurity, defined as 
\begin{equation}
	I_G(x) = \sum_i^J p_i\left(1-p_i\right)
\label{gini}
\end{equation}
where $p_i$ is the percentage of samples classified into class $i$ for each split out of $J$ classes. For our fictional example, $J = 2$.  Consider a split along the value $x = 0.1$ where values less than or equal to $0.1$ are predicted to have $y=0$ and values of $x$ greater than $0.1$ are predicted to have $y=1$. The Gini impurity can be calculated for each side of the split. For the case of $x\leq0.1$, all the samples provided (only one sample, in this case) have an associated $y$ value of $0$. Therefore, the Gini impurity would be 
\begin{equation}
	\begin{split}
	I_G\left(x \leq 0.1\right) & = \frac{1}{1}\left(\frac{1}{1} - 1\right) + \frac{0}{1}\left(\frac{0}{1} - 1\right) \\
		& = 0.
	\end{split}
\end{equation}
The value $0$ is the lower bound for the Gini impurity, thus this split produces perfect classification for values sorted into $x \leq 0.1$. However, the Gini impurity for the remaining values becomes 
\begin{equation}
	\begin{split}
		I_G\left(x > 0.1\right) & = \frac{1}{2}\left(\frac{1}{2} - 1\right) + \frac{1}{2}\left(\frac{1}{2} - 1\right) \\
		& = 0.5.
	\end{split}
\end{equation}
This higher value of the Gini impurity indicates that splitting feature $x$ along the value $0.1$ produces an imperfect classification. If the split was chosen along $0.2$ instead, the Gini impurity for both sides would be $0$, a perfect classification. The Gini impurity can be extended to an arbitrary number of classes, allowing decisions trees to behave as regression algorithms as well as classification tools.

Decision trees compute every possible partition for each feature in the dataset such that the misclassification error, as defined by the Gini impurity, is minimized. However, decision trees are highly prone to overfitting. Random forests overcome this overfitting problem by training many different decision trees, each on a subset of the total dataset. A random sampling, with replacement, of design space coordinates (rows of $\mathbf{B}$) are chosen, known as bootstrap aggregating, or bagging, and a decision tree is trained. Alternatively, or in addition to bagging, jackknifing selects a subset of features (columns of B) to prevent overfitting to specific features..

Training many different decision trees in this way allows a user to calculate uncertainty metrics for each prediction. The method of calculating uncertainty depends on how the random forest is being applied \cite{Ling2017a}. Once the random forest has been trained on the initial dataset, new points in the design space are given to the algorithm and the expected result is predicted. 

The predictions made for new points in the design space can be characterized by several different response functions. A study by Ling et al. employed three response functions: the maximum likelihood of improvement (MLI), maximum expected improvement (MEI), and maximum uncertainty. Each response function has its own benefits. The MEI selects the best experiment for maximizing (or minimizing) a target value. The MU, as the name implies, selects the experiment with the highest uncertainty in predicted result. The MLI chooses the experiment most likely to have a higher (or lower) target value compared to the best previously observed value.

Often, parametric analysis is concerned with either exploring relationships in the design space or optimizing on a property (either minimizing or maximizing the property). The random forest can be trained on $m$ many subsets of the $n$ rows of $\mathbf{B}$. Then, new points in the design space are chosen and their associated property $y_{n+1}$ is predicted. If the goal is to maximize a property, then the next experiment to run can be chosen by the MEI or MLI. If the goal is to explore the design space then it is useful to choose the design space coordinate based on the MU.

Ling et al. trained a random forest to maximize the fatigue life of steel as a function of composition (among other test cases presented in the article) \cite{Ling2017}. The features used in Ling's study included composition as a function of nine different alloying elements $\left(\text{C, Si, Mn, P, S, Ni, Cr, Cu, Mo}\right)$ as well as thirteen different processing steps such as heat treatment temperature. The total dataset used had 437 tests of steel fatigue as a function of the features. The random forest algorithm was used to choose experiments to run balancing maximum predicted fatigue life with uncertainty in the prediction. Ling's random forest approach found the composition and processing combination with the best fatigue life in fewer than 50 experiments out of the 437 possible options when using the MLI. The sequential learning workflow used by Ling, as well as the performance of differently trained random forest algorithms and response functions is shown in Fig. \ref{RFopt}.


%\begin{figure}
%	\centering
%	\subfloat[\label{LingAlgorithm}]\includegraphics[width=1\linewidth]{Images/Fig7a_LingAlgorithm}		
%	\subfloat[\label{RandomForestOpt}]\includegraphics[width=1\linewidth]{Images/Fig7b_RandomForestOpt}
%	\caption{Application of a random forest algorithm to find optimal material candidates for four different datasets: magnetocaloric materials, superconducting materials, thermoelectrics, and steels. A random forest algorithm was used with four different response functions: maximum likelihood of improvement (MLI), maximum expected improvement (MEI), maximum uncertainty (MU), and the COMBO Bayesian optimization approach \cite{Ling2017, Ueno2016}. The algorithm in \ref{LingAlgorithm} was used to speed up the experimental design process. In every case, the optimal material for the application in the dataset was found more quickly through sequential learning than through random guessing. The random forest approach was compared against COMBO, another sequential learning tool. The figure in \ref{RandomForestOpt} demonstrates how much more quickly the random forest algorithm was able to find an optimized state than random sampling of experiments to perform.}
%	\label{RFopt}
%\end{figure}

Random forests have been applied successfully to a range of applications in materials science. They have been used to discover new thermoelectric materials~\cite{Gaultois2016}. They have also been used to model material properties such as thermal conductivity in half-Heusler semiconductors ~\cite{Carrete2014} and to break down fields for dielectrics~\cite{Kim2016}.  A review article detailing many optimization algorithms for design of experiments can be found in Shan et al\cite{Shan2010}. Adoption of machine-learning assisted design of experiments algorithms can rapidly increase the rate at which the relationship between AM process parameters and material properties are understood. 


\subsubsection{Topology Optimization} \label{sec:topology optimization}
\begin{figure*}
	\centering
	\begin{subfigure}{1\textwidth}
		\includegraphics[width=5in]{Images/topopt1}
		\label{topopt1}
	\end{subfigure}
	%
	~
	\centering
	\begin{subfigure}[b]{1\textwidth}
		\includegraphics[width=3in]{Images/topopt2}
		\label{topopt2}
	\end{subfigure}
	\caption{Topology optimization algorithms to optimally design shapes and support structures for additively manufactured parts. While the major benefit of AM is in manufacturing unique, complex geometries the desired shapes sometimes require specific supports. a) Design of support structures for different slopes and shapes that may be difficult to print as-is. b) Topology optimization of a part to reduce the amount of sharp slopes which may cause print failure or residual stress build up. Images taken from the review on topology optimization in \cite{Liu2018}; images in b) taken from \cite{Huang2009, Vanek2014, Dumas2014}}
	\label{topopt}
\end{figure*}

Alloy design and experimental design focus around combinatorial screening of inputs to either search for \textit{new} properties or optimize on current properties. These optimizations reduce manufacturing cost, monetary or otherwise, and maximize performance capability. The same optimization can be applied to mechanical properties of parts. For structural materials, the goal is to optimize load bearing capacity or lifetime while minimizing the amount of material used. For aerospace, the goal is to minimize weight. Unique manufacturing geometries was one of the first intended applications of AM. Topology optimization (TO) focuses around exactly this task -- finding optimized topological structures for a given mechanical application. 

TO algorithms change the topology of a part by applying filters to different regions of its CAD model. A filter is a mathematical operation which reveals information about a region of pixels/voxels in a mesh. Filters are most often represented as a product of a filter matrix with a matrix of mesh pixel values. Topology optimization proceeds by generating a CAD model of an AM part and modeling its performance, such as testing performance under mechanical load through an FEA simulation. Filters are applied to the CAD mesh which selectively removes material from the part. Then, the mechanical performance of the new part is modeled, followed by further material removal. This process proceeds until either a minimum weight/volume condition is met or the mechanical performance of the part is degraded.

In additive, topology optimization serves an additional purpose: TO algorithms can find un-printable regions of a part. Unsupported structures, low angle slopes, and certain part orientations during building are prohibited in AM because they will cause part deformation. An unsupported slope at acute angles can lead to part deformation and warpage \cite{Gaynor2016}. Sacrificial support structures also need to be considered during topological optimization, along with the number of free-hanging features and the orientation of the part during manufacture. Several examples of parts with topology optimized support structures/geometries can be seen in Figure \ref{topopt}. Langelaar et al. developed an AM-specific TO algorithm which searches for regions of parts that have too little support for manufacture \cite{Langelaar2016, Langelaar2017}. Other additive specific algorithms have been designed for optimizing density of parts \cite{Zegard2016}. These algorithms augment the AM process, both by taking advantage of the ability to optimize unique geometries, and also by identifying regions of parts which are incompatible with AM.



%~%~%~%~%~%~%~%~%~%~%~%~%~%~%~%~%~%~%~%~%~%~%~%~%~%~%~%~%~%~%~%~%~%~%~%~%~%~%~%~%~

\subsection{Machine Learning Assisted Modeling of Additive Manufacturing}
As previously discussed, the design space of AM experiments is often vast (e.g., Fig. \ref{AMgene}). While the design of process parameters is often integral to the material design methods reviewed in the previous sections, there are some additional process-centric engineering objectives where machine learning methods may also be beneficial. This section reviews the use of machine learning algorithms to aid in computational design of additive manufacturing process developments. Martukanitz et al. published a full ICME investigation of AM \cite{Martukanitz2014}. There are two modeling scenarios that plague the advancement of AM: the case where a model exists but current numerical methods are too expensive to simulate the model; or the case where a model does not exist. Put slightly differently, in either case $\mathbf{y} = f(\mathbf{x})$ exists but cannot be computed in a reasonable amount of time; or $\mathbf{y} = f(\mathbf{x})$ does not exist. 

Machine learning algorithms have addressed both these cases. In the first case, ML algorithms provide an alternative numerical method for calculating $\mathbf{y} = f(\mathbf{x})$ based on experimental measurements of $\mathbf{y}$ and $\mathbf{x}$, or based on the results of previously run simulations. Machine learning algorithms have also been developed to help visualize trends in high dimensional spaces, allowing researchers to study complex relationships and ask deeper questions. For the second case, ML algorithms provide a form of $\mathbf{y} = f(\mathbf{x})$ from observations (measurements) of the relationship between $\mathbf{y}$ and $\mathbf{x}$.

\subsubsection{Machine Learning as Numerical Methods for Modeling}
There is a suite of numerical methods that have been adopted by the materials science community for computing models of material phenomena. Finite element methods are some of the most common methods. Feedstock, heat source, and melt pool dynamics have been modeled by finite element methods \cite{Toyserkani2004, Khairallah2016, Manvatkar2014} or finite  volume methods \cite{Dai2014}. Some AM-specific tweaks to the finite element method have been developed, such as the quiet/inactive method of Michaeleris et al \cite{Michaleris2014}. They have also been applied to microstructure development \cite{Nie2014}. A review of finite element methods for AM can be found at \cite{Gouge2018}. Models of grain growth in AM have been solved using both phase field numerical methods \cite{Chen2002, Gong2015, Kundin2015, Sahoo2016}, and cellular automata \cite{Tan2011}. Francois et al. provide a review of ICME approaches across spatiotemporal scales\cite{Francois2017}. 

The success of these numerical methods have been in solving complex thermomechanical problems for engineering application. In AM, the number of models that need to be simultaneously considered/computed and the scale of the manufacturing process causes the computational complexity of these methods to grow quickly. Machine learning can make the modeling process more efficient through three primary applications:
\begin{itemize}
	\item Determine a priori which models \textbf{not} to run.
	\item Reduce dimensionality by discarding inputs or physics that are not relevant or that do not have an appreciable impact. 
	\item Compute the same relationship using ML as the numerical method, instead of using explicit methods like FEA, cellular automata, etc.
\end{itemize}

In the case that models \textit{can} be run, but are time-intensive, it behooves the researcher to run as few models as necessary to understand the material response. ML can identify which models will produce the most useful information, informing model choice to the researcher. In another case, ML can be used to identify physics that do not significantly impact the outcome of a model. Reduced-physics models can then be created with a reduced computational burden. In some cases ML can compute the same result as the explicit model with significantly less computational cost (under certain conditions and assumptions).

Dimensionality reduction algorithms identify which parameters are relevant to model in an ICME approach and which are not, enabling future ICME investigations to achieve the same result faster. Materials science has long had a need for dimensionally reduced, computationally accurate models. Some of the first applications of machine learning in materials science was for dimensionality reduction . Dimensionality reduction has been applied to find process-structure-property relations across multiple material length scales \cite{Fischer2006, Flores-Livas2017, Rupp2011, Snyder2012}. Homer applied dimensionality reduction to relate the impact of local atomic environments on mesoscale properties like atomic mobility at grain boundaries, demonstrating the benefit of the technique for advancing ICME \cite{Homer2019}. 

Statistically driven approaches can focus on the parameters in $\mathbf{x}$ that strongly impact AM model outputs, leading to a dynamically guided design of experiments. In design of experiments, a random forest is trained on previously completed experiments. These are the rows of the matrix $\mathbf{B}$ in Eqn. \ref{Bmatrix}. Then, new points in the design space that have not been observed are given to the algorithm and predictions about the output are made. Dimensionality reduction using random forests proceeds differently. 

The random forest is trained on subsets of the data in $\mathbf{B}$. The importance of a feature in the dataset is tested by randomly shuffling the values of one of the columns of $\mathbf{B}$. If randomly shuffling the values of a given feature does not significantly impact the prediction accuracy of the random forest then it is likely that the feature is not important. Towards Data Science provides a more in-depth tutorial on using random forests for feature importance determination \cite{FeatureImportance}. This approach can be applied in computational models that consider many different physics. Several models are run under different initial conditions in the design space. The entires of the matrix $\mathbf{B}$ are the inputs and predicted outcomes of the model. If the exclusion of a model input does not significantly impact the random forest's prediction of the model output, then that input can likely be ignored in future simulations, saving computational time and reducing the number of models that need to be run.

Kamath used a random forest algorithm to screen out irrelevant modeling parameters for predicting maximum density of additively manufactured parts \cite{Kamath2016}. Kamath started with an experimental dataset of manufacturing parameters and multiple modeling methods. An Eagar-Tsai simulation of a Gaussian laser beam on a powder bed was used to model thermal conduction during manufacture, as well as the computationally more expensive Verhaeghe model. The Eagar-Tsai model originally began with four inputs (laser power, speed, beam size, and powder absorptivity) and a design space of 462 possible input combinations. Kamath used random forests to determine which input was most important for achieving fully dense parts. If simulations are time-intensive to run then 462 different simulations may be out of the question. The computational dataset was complemented with an experimental dataset of measured melt pool widths at various printing conditions. Identifying which parameters do not impact the final result reduces the size of input combinations, therefore reducing the number of computations or experiments to be performed. 

Kamath identified that laser speed and power were the most important inputs out of the four to determine melt pool depth and shape. Now that the important physics have been identified, the researchers can proceed to the more expensive Verhaeghe model with knowledge of what parameters to vary.  After determining the most important inputs, the same regression tree was applied to find optimized manufacturing conditions for fully dense parts. However, instead of identifying which features impacted the model standard deviation, the machine settings that maximized $y$ were found. 

A final technique for reducing the burden of computational models requires expressing model data in a matrix and performing matrix factorization. As before, model inputs can be formed into a matrix, $\mathbf{X}$, whose rows are coordinates in the design space. Matrix factorization techniques represent correlations in large datasets in a simplified way. The matrix $\mathbf{X}^T\mathbf{X}$ is a measure of covariance within $\mathbf{X}$. The matrix $\mathbf{X}^T\mathbf{X}$ can be very large due to the design space of additive manufacturing. One type of matrix factorization, called Principal Component Analysis (PCA) represents the data matrix $\mathbf{X}$ as
\eqn
	\mathbf{X} = \mathbf{U} \mathbf{\Sigma}\mathbf{V}^T
	\label{PCA}
\equ
where the columns of $\mathbf{U}$ are the eigenvectors of $\mathbf{X}\mathbf{X}^T$ and are called the \textit{principal components of $\mathbf{X}$}. Similarly, the columns of $\mathbf{V}$ are the eigenvectors of $\mathbf{X}^T\mathbf{X}$. The matrix $\mathbf{\Sigma}$ is a diagonal matrix whose entries are the singular values of $\mathbf{X}$. The first singular value, which corresponds to the first column vector in $\mathbf{V}$, has the highest variance (most information); the second singular value, the second most; and so on. Therefore, regression can be performed on one, a few, or many of the principal components to predict new model results using considerably less data than present in $\mathbf{X}$, but with a minimal loss of information and a minimal reduction in model accuracy.  Materials science studies have used PCA previously to re-represent large datasets in simpler forms, such as predicting the formation energies of crystal structures from a lower dimensional space \cite{Curtarolo2003}. A review of applications of PCA in materials science can be found at \cite{Rajan2009}.

In additive manufacturing, PCA can serve to reduce the number of features in the design space. The new vectors can then be used as regression model inputs for prediction of material properties based on trends observed in dataset $\mathbf{X}$.



\subsubsection{Machine Learning for Visualizing Trends in the Design Space}\label{viz}

Visualizing relationships across high dimensional spaces helps researchers develop an intuitive understanding of data relationships that exist, an intuition that helps guide data preprocessing, feature engineering, model selection, and model training. However, visualizing an $n$-dimensional distribution is difficult. Process maps are commonly employed in AM to visualize 2D slices of the $n$-dimensional AM design space \cite{Beuth2001}. The Ashby plot is a well known generalization of process maps in materials science. Ashby plots show material properties as functions of two design coordinates, such as plotting mechanical strength of various alloys as a function of density and cost to produce. The process maps in welding and AM are more specific versions of Ashby plots. Process maps chart the possible values of machine inputs and identify regions of the design space with similar properties. A commonly employed process map in AM of Ti-6Al-4V describes grain morphology as a function of solidification velocity $R$ and temperature gradient $G$ \cite{DeHoff2015}. Extending process maps to $n$ many process variables would require ${n \choose 2}$ plots. Defining and examining metrics of similarity in an $n$ dimensional space can reveal trends in a human interpretable way without relying on multiple 2D process maps. 

%\begin{figure*}
%	\centering
%	\begin{subfigure}{0.49\textwidth}
%		\includegraphics[width=1\linewidth]{Images/Fig10a_ProcessMap}
%		\caption{}
%		\label{ProcessMap}
%	\end{subfigure}
%	%
%	\begin{subfigure}{0.49\textwidth}
%		\includegraphics[width=0.75\linewidth]{Images/Fig10b_tSNE}
%		\caption{}
%		\label{tSNE}
%	\end{subfigure}
%	\caption{Comparison of a traditional process map and tSNE plot. \ref{ProcessMap}) A process map for predicting microstructure characteristics based on absorbed power and deposition velocity in electron beam wire feed additive manufacturing. Image reproduced from Gocket et al\cite{Gockel2014}. \ref{tSNE}) A tSNE plot from Ling's study showing clusters of samples with similar fatigue strengths\cite{Ling2017a}. While process maps can be useful for predicting manufacturing outcomes they are limited by only showing the behavior of two process parameters at a time. The tSNE algorithm can cluster data based on many manufacturing inputs simultaneously and then display that information in a 2D plot, allowing engineers to study how processing parameters lead to good or bad material properties.}
%	\label{ProcessMaptSNE}
%\end{figure*}

$t$-distributed Stochastic Neighborhood Embedding (tSNE) is a visualization technique that measures distances in a high dimensional space and then projects data points onto a two dimensional plot. The similarity of all data points in the design space with each other is used to fit a distribution of similarities. The tSNE algorithm begins by fitting a probability distribution to all $\mathbf{x}$'s contained in a dataset. Relationships in $n$ dimensional space are assessed through a \textit{kernel function} $\kappa(\mathbf{x},\mathbf{x'})$ that measures similarity between points in the design space. A commonly employed kernel is the Gaussian kernel
\eqn
	\kappa(\mathbf{x},\mathbf{x'}) = \frac{1}{\sqrt{2\pi\sigma^2}}\exp{\left[- \frac{\mathbf{x} - \mathbf{x'}}{2\sigma^2}\right]}
	\label{gausskernel}
\equ 
where $\sigma$ is a user-specified or fit standard deviation in the distribution of points in the design space. This kernel function assesses distance in the $n$ dimensional space and assigns a similarity value between $\left[0, 1/\sqrt{2\pi\sigma^2}\right]$. 

 After the $n$ dimensional dataset is fit, then a 2 dimensional coordinate $\mathbf{x}^*$ is assigned to each $\mathbf{x}$. The reason for choosing a 2 dimensional coordinate is so that the final result can be visualized on a 2D plot. The tSNE algorithm fits a probability distribution to the $n$ dimensional data set first, then assigns values to each $\mathbf{x}^*$ such that they have the same probability as the associated high-dimensional $\mathbf{x}$. Once the probability distributions have been assigned, the $\mathbf{x}^*$ values can be visualized on a 2D plot to investigate trends.

The benefit of tSNE is that points that are close together in the $n$ dimensional space appear close together on the 2 dimensional plot. This gives AM modelers an idea of how machine inputs and material behavior are distributed in the $n$ dimensional space through a 2 dimensional visualization. Traditional process maps provide similar input/output relationships but are limited in the amount of processing parameters they can interpret at once. A comparison of process maps and tSNE is shown in Figure \ref{ProcessMaptSNE}.

\subsubsection{Machine Learning to Create Models of Additive Manufacturing Processes}
Another problem, equally important to solving models, is the creation of models for additive manufacturing problems. Scientists cannot engineer the additive manufacturing process without an understanding of how process parameters (inputs) impact material properties and performance (outputs). The generation of models in AM is a difficult task due to the large amount of physics that can be incorporated.

Many traditional material models from science and engineering have been applied to additive manufacturing, including thermal history models of heat transfer through the part \cite{Michaleris2014}, residual stress build up during manufacturing \cite{Pal2014, Ding2011}, and thermal signatures such as cooling rate and temperature gradient \cite{Li2014, Raghavan2016}. King et al. provide a review of the physics of AM modeling \cite{King2015a}. 

Phenomenon that are difficult to study experimentally, such as flow within the melt pool, are best studied through modeling approaches. Though expensive, full-physics modeling is often necessary to understand how physics at different scales interact to impact the AM process. If the computational expense of a simulation is too high then performing simulations at all relevant manufacturing conditions can be infeasible. While it is useful for optimization and visualization, reduced order models are unlikely to capture the full dynamics of solidification in AM. 

Within the context of the literature reviewed herein, a \textit{surrogate model} is a regression model that estimates the results of high-cost simulations. Surrogate models are regressed on the inputs and results of previously run simulations. Then, the surrogate model interpolates simulation results at new coordinates in the design space. Surrogate models preclude the need for running computationally expensive simulations for every possible manufacturing condition. Formulating surrogates can be as simple as performing linear regression between simulation inputs and results, but are often more complex. The accuracy of a surrogate model is dependent upon how many previous simulations have been run and at how many different points in the design space.

Tapia et al. built a surrogate model for laser powder bed fusion of 316L stainless steel. They were concerned with predicting the melt pool depth of single-track prints solely from the laser power, velocity, and spot size \cite{Tapia2017}. The dataset used to build the surrogate was computationally derived, based on previous simulation methods used by the same research team \cite{King2014}. In particular, they used the results from a computationally expensive but high-accuracy melt pool flow model of Khairallah et al. \cite{Khairallah2016}. They ran powder bed simulations at various laser powers, velocities, and spot sizes, and the model told them the depth of the melt pool, amongst other information. The datasets provided enough information for a surrogate model to be trained to predict simulation results.

%\begin{figure*}
%	\centering
%	\begin{subfigure}{1\textwidth}
%		\includegraphics[width=1\linewidth]{Images/Fig11ab_TapiaInputData}
%		\label{InputData}
%	\end{subfigure}
%	%
%	\begin{subfigure}{1\textwidth}
%		\includegraphics[width=1\linewidth]{Images/Fig11cd_TapiaResponseSurface}
%		\label{ResponseSurface}
%	\end{subfigure}
%	\caption{Input data, response surface $y(\mathbf{x})$, and error $\epsilon(\mathbf{x})$ of Tapia's study in predicting melt pool depth for laser powder bed fusion \cite{Tapia2017}. a-b) The input data used to train a Gaussian Process Regression model for predicting melt pool shape and depth. They started with a sparse sampling of the design space to build the model. c-d) The Gaussian process model predictions for melt pool depth as a function of two input parameters $y(\mathbf{x})$ (left) and the associated prediction errors $\epsilon(\mathbf{x})$.}
%	\label{GPMs}
%\end{figure*}

To build their surrogate model, Tapia used a machine learning algorithm known as a Gaussian process model (GPM). A common model assumption in Gaussian process modeling is
\begin{equation}
	z(\mathbf{x}) = y(\mathbf{x}) + \epsilon(\mathbf{x})
	\label{model}
\end{equation}
where $y(\mathbf{x})$ is the approximation (surrogate) of the simulation process, $\epsilon(\mathbf{x})$ is a stochastic, randomly distributed noise in measurement, and $z(\mathbf{x})$ is the value given by a simulation. The primary goal in GPMs is to find model parameters for the mean process $y(\mathbf{x})$ and a covariance function $\kappa(\mathbf{x},\mathbf{x}')$, which is a function of similar form to Eqn. \ref{gausskernel}. Fitting a Gaussian process model often begins with assuming a model function for covariance, fitting the model parameters such as $\sigma$ to the observed values $z(\mathbf{x})$, then using those model parameters to predict simulation results $y(\mathbf{x})$ at other locations in the design space. 

Tapia used Bayesian statistics to develop a probabilistic model that predicted melt pool depth from simulation inputs. They were able to successfully predict the outcomes of both high-fidelity simulations and experimental measurements solely by analyzing trends in previously obtained results. In particular, they were able to accurately predict the melt pool depth at a value that had never been observed before, either computationally or experimentally. For future investigations, predictions by the surrogate model can be relied upon instead of running a simulation or experiment. Regression models such as this provide engineers with faster routes toward optimized manufacturing states by predicting manufacturing at a wide range in the design space based on only a few initial experiments.

Gaussian Process Models provide robust uncertainty metrics on the predictions they make. Uncertainty estimation is important in materials informatics because it enables scientists to know how confident their models are in predictions in various regions of parameter space. Some machine learning models do not have straightforward ways of assessing model error \cite{Bessa2017}. 

Another benefit of GPM is that it aids in inverse design and design space visualization. GPMs can explicitly identify regions of the design space that will maximize or minimize a value. In the case of Tapia et al. response surfaces were created from the GPM that visualized the depth of melt pools as a function of laser power and speed. Doing so allows engineers to identify regions of the design space that provide specific material responses, an important tool in optimization for additive.

%Another approach to full scale models in AM is building high-fidelity models from an ensemble of low-fidelity models. Current integrated computational models link phenomena across spatiotemporal scales by running many single-physics models and passing the results from model to model, then comparing with experimental results. An example is the model of Martukanitz et al. that considered the thermal, mechanical, and material response of Ti-6Al-4V alloys manufactured with powder bed processes \cite{Martukanitz2014}. Martukanitz's model uses the finite element method to model the thermal and mechanical response based on classical continuum heat transfer equations. At the same time, the Calculation of Phase Diagrams (CALPHAD) method modeled the thermodynamics and mass transfer for each chemical species, while Diffusion-Controlled phase Transformations (DICTRA) is used to model phase formation. As noted by Martukanitz, this approach becomes computationally infeasible as the number of deposition layers increases.

%{\color{red} Consider cutting this entire next section, or seriously re-working it.} 
%
%Even a \textit{single} modeling method -- just FEA, phase field, CALPHAD, etc -- becomes computationally expensive for thousands of layers of material deposition. Based on this scaling, accurately modeling the physics of a full build seems prohibitive, at least without a major leap in computing power. A machine learning method known as \textit{committee voting} or \textit{ensemble modeling} may provide a workaround. These methods rely on sampling many individual models to predict the behavior of a larger class of physics.
%
%In traditional ensemble modeling approaches, many different regression models are trained to model a relationship $y = f(\mathbf{x})$, as in regression trees. For a given input $\mathbf{x_1}$ all regression models are assessed and various outputs predicted, $f_1(\mathbf{x}), f_2(\mathbf{x}), ..., f_n(\mathbf{x})$, etc. Each model $f_i(\mathbf{x})$ is a different type of machine learning regression model. Different types of regression approaches are robust for different trends in data sets. The individual regression models could be a random forest, support vector machine, neural network, or any of the type of regression models discussed herein. A linear combination of these models can have a higher prediction accuracy than any individual model.

%In AM, the same concept can be extended to datasets sampled from different physical models, instead of an ensemble of different regression models. Each model $f_i(\mathbf{x}_j)$ can be trained on the inputs and outputs of the $j$th AM simulation. These data sets for additive may be the inputs and solutions of a solidification or heat transfer model, for example, or experimentally obtained relationships.
%
%Ensemble modeling can be used in AM to solve multiobjective optimization problems, such as optimizing the heat transfer through a build and the grain growth simultaneously. The simulations to include could be thermomechanical models of heat transfer, phase field models of grain growth, finite element models of laser absorption, and so forth. The final ensemble model would take the form
%\begin{equation}
%	\mathbf{y}(\mathbf{x}) = \sum_{i=1}^N w_i f_i(\mathbf{x})
%	\label{ensemble}
%\end{equation}
%where $\mathbf{y}$ is a response vector of properties to optimize, $f_i(\mathbf{x})$ is a single simulation of the process out of $N$, and $w_i$ are weights associated with each simulation. The vector $\mathbf{y}$ would contain many parameters such as temperature $T(t)$, temperature gradient $\nabla T$, solidification velocity $\nu(t)$, and many more phenomenon to monitor from simulations. 
%
%In the AM case, where information from models may overlap, this type of approach can screen out noise from simulations or experiments to gauge a more fundamental relationship. The uncertainties associated with a model can be weighted by the experimentally observed data points. Furthermore, predictions can be made across many different physical models, resulting in a predictive method for holistic analysis of the final properties. This method does not provide a picture of how the physics in each simulation interact or disagree. However, it can be used to simultaneously optimize the results of many different simulations and point toward desirable manufacturing conditions. A review of multiobjective optimization functions can be found at \cite{Jin2008}.
%
%Meredig et al. applied this method to the prediction of ABC ternary compounds, where A, B, and C each represent an element \cite{Meredig2014}. The space of all combinations of A, B, and C elements is large and would require many, many simulations. Instead of running high-throughput DFT for all possibilities they ran calculations for AB, AC, and BC compounds to generate a database of information. Then, regression trees were trained from the inputs of each binary alloy simulation to predict formation energy. Finally, an ensemble model was trained of the form
%\begin{equation}
%	E(\text{ABC}) = w_{\text{AB}} f(\mathbf{x}_{\text{AB}}) + w_{\text{AC}} f(\mathbf{x}_{\text{AC}}) + w_{\text{BC}} f(\mathbf{x}_{\text{BC}})
%	\label{dftensemble}
%\end{equation}
%where $E()$ is the formation energy, $w_\text{AB}$ is the weight for a regression model of AB alloys, $f(\mathbf{x}_{\text{AB}})$ is the regression predicting formation energies for AB alloys, and so on. In the end, Meredig's model was able to predict $4,500$ new, stable ternary materials.

Machine learning is not only limited to ex situ experimental investigations or modeling approaches. Ideally, machine learning can be used to solve multiobjective optimization functions where multiple aspects of the AM process are optimized at once -- energy density, melt pool shape, heat transfer, grain growth, and the list continues. Models can be created that solve this multiobjective optimization problem and present to the engineer what an optimal manufacturing process looks like. Actually \textit{creating} that optimal process will require tight control of the manufacturing process. Machine learning models trained on correlations between build parameters, the dynamic response of the system, and the final part properties can be combined with real-time computer vision to simultaneously observe, characterize, and control many different aspects of the manufacturing process.

%~%~%~%~%~%~%~%~%~%~%~%~%~%~%~%~%~%~%~%~%~%~%~%~%~%~%~%~%~%~%~%~%~%~%~%~%~
\subsection{Process Monitoring and Control}

The numerousness of signals to measure in situ for AM warrants use of quick, efficient, and robust signal processing methods for process monitoring, feedback, and control. These signal processing algorithms are closely related to machine learning. They serve as tools in their own right, and can also pre-process data for use in other machine learning applications, like clustering and regression. Computer vision is one class of image recognition algorithms that has been developed for automated feature identification in signals. Intelligent computer vision uses ML algorithms to identify objects and features in a wide variety of data types. We proceed to discuss the potential uses of data analytics and ML to advance our ability to directly study and control AM process technologies.

\subsubsection{In Situ Process Monitoring and Feedback}
Computer vision is a class of image recognition algorithms that have been developed for automated feature identification in images. Intelligent computer vision utilizes machine learning algorithms to identify objects and features in images and time-series data. 

Computer vision can be employed in additive manufacturing to monitor the printing process, such as measuring temperature profiles, observing melt pool morphologies, and automatically detecting defect formation. Doing so will require methods for in situ process monitoring and data collection. Thus far, in situ control in AM has been consistently ranked as one of the most-needed technologies for advancing the technology \cite{Berumen2010, Tapia2014, Mani2017}. The combination of rapid solidification and the small length scales of AM solidification can make traditional process monitoring approaches difficult. Machine learning can fill in gaps where human-specified process monitoring models are insufficient.

Process monitoring involves acquisition of realtime signals which can be processes to reveal information about manufacturing. McKeown et al. used dynamic transmission electron microscopy to measure solidification rates in powder bed AM \cite{McKeown2016}. Bertoli et al. also characterized cooling rates using high speed imaging \cite{Bertoli2017}. Raplee et al. used thermography to monitor the solidification and cooling rates of electron beam powder bed fusion, then related the temperature profiles to defect and microstructural characteristics \cite{Raplee2017}. Distortion of parts due to thermal cycling was investigated by Denlinger et al. by means of thermocouples in contact with the build substrate \cite{Denlinger2015}. All of these methods are amenable to aid by computer vision. 

There are two areas of need for in situ measurement: \begin{itemize}
	\item Processing of signals real time to identify features of the AM process
	\item Multi objective feedback and control
\end{itemize} \textbf{There are several review articles on AM (Tapia, for example) which could go here, or could go in the intro paragraph of Section III.c}

Many of the difficulties in real time signal processing is the complexity and number of signals being acquired. Identifying features of interest in a signal becomes difficult when the necessary signal features to identify are not known. Researchers are well aware that defects form during the additive process. Which signals to monitor and what about the signals are indicative of a defect is not known. 

Setup for the following information:
\begin{itemize}	
	\item Explain the concept of identifying features in images (filters, SIFT/SURF)
	\item Explain template matching using a dictionary of identified features
	\item Explain the concept of `learning' filters through a CNN
	\item Explain the use of CNNs in AM process monitoring
	\item Move Gobert to case study section
\end{itemize}

Convolutional neural networks (CNNs) have proven to be one of the best computer vision approaches for identifying complex features in images \textbf{citation?}. At the highest level, CNNs use large databases of labeled information to learn features in the image that indicate whether or not certain human-identified objects are present. The level of abstractness of the object is arbitrary, so long as the dataset contains enough images with each feature to be identified. This is an improvement over current methods whereby humans develop models specifically to identify features of the printing process. Human-specified models for defect detection -- \textbf{search around the literature you've cited for an example of a human-specified model that is very specific to the problem being address, perhaps Abdelrahman? \cite{Abdelrahman2017}.} -- can be limited in their ability to identify edge cases, as well as in their ability to detect multiple types of defect formation. Convolutional neural networks can identify multiple features or defects in an image simultaneously. A downside to CNNs is that they require \textbf{very} large datasets (thousands of images, at the least) of labeled data to be successful. Neural networks have already been implemented for in situ AM analysis by Scime et al. \cite{Scime2018}, Yuan et al \cite{Yuan2018}. 

Another computer vision approach is called \textit{template matching} and does not require the same dataset size as CNNs. Template matching is the process of comparing computer-vision identified `keys' in an image with a database of keys associated with a certain feature. For example, powder inclusion in a melt pool can often be characterized by a local change in reflectivity in the melt pool near the particle \textbf{cite???? where are you coming up with this, or are you asssuming?}. A feature identification algorithm such as the scale invariant feature transform (SIFT) \cite{Lowe2004} or speeded-up robust features (SURF) \cite{Bay2008} can identify common characteristics of a powder inclusion in an image. Template matching involves using an algorithm like SIFT and SURF to identify features in an image, then compare those features with a database of typical AM features, like porosity or powder inclusion. If keys of the feature in the image closely match a key in the template database then it is likely that feature exists in the image. 


\subsubsection{Featurization of Qualitative Image Data}
The intuition gained from human interpretation of images is not always encodable in a way that is compatible with models. A human might make a complicated interpretation of an image, such as ``the microstructure is mostly acicular, with some prior grains present, and a string of pores near the boundary." Surely, such a description conjures an image in the reader's head of what such a microstructure might look like. Encoding the same information into a finite element mesh of a microstructure isn't necessarily straightforward. It can be accomplished, for sure, but transforming an observed microstructure into quantitative information requires time-consuming methods. 

Image recognition algorithms are making strides at automating quantification of information from images, while also being able to produce qualitative descriptions of the images. Information such as the size of grains, orientation of grains, presence of cracks or pores, different phases, and more can be automatically identified through computer vision. DeCost et al. have made strides in turning metallographs into sets of quantitative information \cite{DeCost2015, DeCost2017, DeCost2017a}. 

In one particular example, DeCost et al. utilize scale invariance feature transform (SIFT) in order to identify possible features in images. The SIFT algorithm is a widely used algorithm for feature identification and is implemented in many open-source packages, such as OpenCV for C++ or Python \textbf{cite OpenCV}. SIFT relies on identifying regions of maximal and minimal intensity in successively blurred versions of an image. The gradient of pixels are regions of extremal intensity are then computed, resulting in a keypoint descriptor which encodes the size and orientation of a feature, as well as generates a descriptor for identifying that object. The idea behind SIFT is that similar features across images will have similar descriptors. This way, common features can be identified across images. 

DeCost et al. performed SIFT on a database of microstructure images across several alloys \cite{DeCost2015}. The ``features'' that were identified across \textit{all} images were then clustered by $k$-means clustering to identify a dictionary of features. Human investigation can assign qualitative labels to clusters, such as ``$\alpha$ grain" or ``pore" or ``grain boundary." Now, new images can be analyzed with SIFT and their features can be compared to the dictionary of human-interpretable feature descriptors. 

Not only does this process partially automate analysis of images it also automates translation of image-level information into quantitative information. Grain size can be measured by the number of pixels in an identified grain. Orientation of grains in the image can be determined from the values of the keypoint descriptor. The number of pores in an image can be counted as the number of features which match ``pore" in the dictionary. Of course, a human could perform all of this with a ruler and their eyes. However, a computer can perform the same process much faster and on many more images. 

Chowdhury et al. took a more expansive approach to performing feature identification in microstructures. In particular, they were looking to classify microstructures as either dendritic or non dendritic. Chowdhury employed 8 different feature identification methods for a dataset of images. Classification was performed using support vector machines (SVM), Na\"ive Bayes, nearest neighbor, and a committee of the three previous classification methods \cite{Chowdhury2016}. Chowdhury's wide approach to image classification compared the predictive ability of all combinations of feature identification and classification methods, achieving classification accuracies above 90\%. 