\paragraph{Phrasing Additive Manufacturing as a Machine Learning Problem}\label{phrasing}
While machine learning may seem abstract at first, it can be expressed and understood in plain terms. Many of the tenets and frameworks for machine learning are based in mathematical operations that are likely familiar to any scientist or engineer, but applied in new ways. In this section, we proceed to define the basic terminology and classes of machine learning and data. A list of machine learning algorithms used in the papers cited in this review can be found in Table \ref{ML}. The following section details general terminology and intuition for the application of AM. Specific ML algorithms are then introduced specific to contextual AM examples in Section III.

Machine learning algorithms are mathematical constructs that may be used as scientific and/or engineering tools when warranted. They are not appropriate for all science and engineering problems - just as finite element simulations should not be used to study the mechanics of discrete interfaces or single atomic bonds between two atoms or DFT should not be used to simulate mm-sized polycrystals, machine learning algorithms should not be used to model data that lack statistical correlations. Hence, the first questions every scientist and engineer considering the use of machine learning approaches should ask and answer is: "how are the data statistically distributed?", and "are there statistical correlations between the data features of interest?" Once this is complete, then a researcher can decide if ML is appropriate.

If the data lack clear statistical correlations using basic probability analyses, machine learning is not a "magic box" that can suddenly make such correlations evident. Similarly, if the statistical distributions of the data are featureless except for an occasional outlier, machine learning cannot meaningfully fit a model that is based on statistical distributions. 

Today, many scientists and engineers are embracing the approach that "we will machine learn it," without understanding how to evaluate if machine learning is an appropriate tool to apply to a problem or not. One unpublished example in AM of a problem that ML is not well suited for is building a model to predict the location of a maximum pore within a powder bed laser fusion build. A maximum pore is a statistical outlier - usually one of thousands-to-millions, depending on the size of the part being built. Even though the pore may occur in the exact same position of the build volume if the same part is built over-and-over again (i.e., it is highly repeatable), the fact that it is a statistical anomaly means that nearly all machine learning algorithms are built to ignore it. Once this understanding is at hand, then a researcher can decide if ML is appropriate.

Still, most data of interest in metals AM have strong statistical features, as we will proceed to discuss in more detail in the examples given in this review. Once some basic statistical analysis of the data of interest has been performed and it has been determined that there are quantifiable correlations between the inputs and outputs, or across different inputs, and that there are also statistical features that describe the distributions of the data, then a researcher can proceed to consider data featurization and processing, and then tune and evaluate the performances of machine learning models to find the best performers. We proceed to describe these techniques in more detail, after defining some basic terminology used in this article.



    \renewcommand{\arraystretch}{0.8}

    \setlength{\tabcolsep}{5pt}
\begin{table}
\begin{center}
 \caption{A possible design space for laser powder bed fusion additive manufacturing. There are over $10^4$ possible combinations of machine inputs, based on the listed ranges and step sizes. Any possible combination of these parameters is a point in the design space.}
         \label{table:design_space}        
         \begin{tabular}{c|c|c|c}       
            \toprule
            \hline
             Parameter & range & step size & levels \\ \midrule
            \hline
            Power & 100-200 W & 10 W & 10 \\
            Scan speed & 500-1000 mm/s & 100 mm/s & 5 \\
            Spot size & 50-100 $\mu$m & 10 $\mu$m & 5 \\
            Energy density & 1-5 J/mm$^2$ & 1 J/mm$^2$ & 5 \\
            Sample Build Direction & 0-180$^\circ$ & 90$^\circ$  & 3 \\
            Amount of recycled powder & 0-100\% & 10\% & 10 \\
            Hatch spacing & 0.1-0.50 mm & 0.1 mm  & 5 \\
            \hline
            \bottomrule
        \end{tabular}
\end{center}
\end{table}
 
 
\subsection{The Design Space of Additive Manufacturing}
The \textbf{design space} of metals AM is the set of all PSPP relationships. More specifically, the term `design space' will be used throughout this article in reference to the set of AM data that is used and calculated by machine learning algorithms. An example design space for laser powder bed fusion (LPBF) of metals, the most industrially prolific of current metals AM technologies, is graphically depicted in Figure \ref{AMgene}. A complementary example of a process design space of LPBF is given in Table \ref{table:design_space}. Observable process phenomena may link the manufacturing parameters to the resulting materials properties, hence they may also be used to augment the manufacturing parameters and material properties within the design space. Examples include melt pool morphology, temperature history, and cooling rates.

A single combination of process parameters, observed process phenomenon, measured material properties, and a part's performance can be considered as a \textit{coordinate}, or point, in the design space. Single coordinates, defined this way, can sometimes lead to a multitude of material properties due to latent variables, unforseen complications, and the stochasticity of the process. Explicit consideration of process phenomenon in the design space coordinate can be used to more accurately establish unique points within the design space. In summary, any part that is processed under a single set of conditions and is observed to have a set thermal history and set of material properties can be considered to be manufactured \textit{at that point} in the design space.

While the design space of AM is vast, data cannot always be given to machine learning algorithms `as is.' It is important to consider the sources of data in the design space and how they need to be changed or curated for use with ML.


%%%
\begin{landscape}
\begin{longtable}{p{2.5cm}p{3cm}p{4cm}p{5.5cm}p{5.5cm}}
\endfirsthead

\hline
\raggedright Class of Algorithm & Examples & Applications & Strengths & Constraints \\ \hline 
Weighted neighborhood clustering & \raggedright Decision trees, Random Forest, k-Nearest neighbor & \raggedright Regression, Classification, Clustering and similarity & \raggedright These algorithms are robust against uncertainty in data sets and can provide intuitive relationships between inputs and outputs. See Ref. \cite{Quinlan1986} for a primer on clustering. & They can be susceptible to classification bias toward descriptors with more data entries than others. \\ 

\endhead
\caption{Several of the most widely used machine learning algorithms in materials science.\label{ML} }\\


\dashedrule{.1em}{.1em}{325} & & & & \\

\raggedright Linear dimensionality reduction & \raggedright Principle component analysis (PCA), Support vector regression (SVR), Nonnegative Matrix Factorization (NMF) & \raggedright Experimental design, model dimensionality reduction, model or experimental input/output visualization, descriptor analysis, regression  & \raggedright This type of algorithm can produce orthogonal basis sets that reproduce the training data space. They can also provide quick and accurate regression analysis. For a primer on PCA specifically, see Ref. \cite{Bro2014}. & The relationships studied must be linear in nature, and these algorithms are susceptible to bias when descriptors are scaled differently. \\ 

\dashedrule{.1em}{.1em}{325} & & & & \\

\raggedright Nonlinear dimensionality reduction & \raggedright t-SNE, Kernel ridge regression, Multidimensional metric scaling &\raggedright  Experimental design, model dimensionality reduction, model or experimental input/output visualization, descriptor analysis, regression &\raggedright These algorithms are robust against nonlinear input/output relationships and can help visualize similarity in high dimensional relationships. For accessible examples, see Refs. \cite{Tenenbaum2000, Roweis2000}. & Interpretation of high dimensional similarity can be difficult; while these algorithms are useful for visualizing relationships interpreting the \textit{why} of the relationship found is difficult. Global relationships can also be lost when nonlinear dimensionality reduction results are projected onto lower-dimensional spaces.\\ 

\dashedrule{.1em}{.1em}{325} & & & & \\

Search algorithms & \raggedright Genetic algorithms (GA), Evolutionary algorithms & Alloy design (in conjunction with a material modeling approach), model optimization. topology optimization for AM &  \raggedright Search algorithms are intuitive for material properties that can be described geometrically, such as topology optimization for weight reduction. They are efficient at searching spaces with multiple local extrema, such as finding local maxima of quality in multidimensional design spaces. For a useful application of genetic algorithms to process characterization, see Ref \cite{Grefenstette1986}. &  These success of these algorithms are highly dependent upon selection and mutation criteria. \\ 

\dashedrule{.1em}{.1em}{325} & & & & \\

\raggedright Neural Networks \& Computer Vision & \raggedright Artifical Neural Networks, Convolutional Neural Networks (CNN), General Adversarial Networks (GAN) & \raggedright Classification, regression, feature identification and extraction in images, simulation of atomic potentials, transfer learning, in situ process monitoring, feedback and control & Neural networks have successfully modeled processing and image data; the research and development surrounding NNs is among the most mature of any type of machine learning algorithm. & Neural networks tend to require large training datasets, especially for image analysis applications; however, transfer learning approaches can adopt NNs to small datasets. \\ \hline
\end{longtable}
\end{landscape}
%%%


\subsection{Data Sources}


\renewcommand{\arraystretch}{1.25}
\begin{landscape}
\begin{longtable}{p{2.75cm}p{2.75cm}p{2.75cm}p{2.75cm}p{2.75cm}p{2.75cm}}
		\caption{Types and sources of data common in materials science and, specifically, additive manufacturing. The entries under each vary from a source of data -- like a characterization technique -- to the data itself -- like a single measured scalar value. \label{sources}} \\ 		
		%\endfirsthead  %\hline
		Scalar & Time Series & Spectral & Images & Categorical & Spatial \\ \hline 
		
		\endhead%
\hline

		\raggedright Ultimate tensile strength & Stress-strain curve  & X-ray diffraction & TEM & Composition & 3D Model and Slicing Path (e.g. STL file)\\
		Hardness & Temperature Gradient  & \raggedright X-ray Photospectroscopy  & SEM & Quality & Scan path \\
		
		Toughness & Pyrometry &  \raggedright X-ray Dispersive Spectroscopy & Optical Metallography  & Crystal structure &  Part Orientation in Build Chamber\\
		
		Fracture Strength &Thermography   & & \raggedright X-ray Computed Tomography & \raggedright Melt Pool Morphology & Crystallographic Texture\\
		Density & \raggedright Differential Thermogravimetric Analysis & & High Energy Diffraction Microscopy& & \\
		
		Solidification Velocity & Differential Scanning Calorimetry & & & &\\
		Cooling rate & Chemorheology& & & &\\
		Solidus/Liquidus Temperature & Magnetometry& & & &\\
		\raggedright Enthalpy of Formation/Melting & & & & &\\ 
		Pore size &  & & & & \\
		Fatigue Properties & & & & &\\ \hline
\end{longtable}
\end{landscape}
		
	


Data, as a materials scientist normally thinks about the term, encompasses a vast range of sources and formats. Some of the most common sources of data used by materials scientists for AM can be seen in Table \ref{sources}. 

The most obvious data that materials scientists interact with are scalar values like modulus, ultimate tensile strength, laser scan speed, laser energy, layer height, etc. Distributions of scalars are also used such as grain size distribution or particle size distribution of AM feedstock. Many materials scientists interact with series data that can be subdivided into several more categories. Times series data can include a temperature measurement from a thermocouple during an AM build. Other series data include X-ray diffraction histograms or X-ray fluorescence spectra. 

Data can also take non-quantitative forms, often referred to as categorical data. These can include crystallographic structure, grain morphology, or the shape of an AM part. In many cases, these categorizations can be converted into quantitative data by measuring a feature such as the major and minor axis length of a grain. More difficult to quantify categorical data in AM includes melt pool morphology and track solidification defects like ``balling" or ``lack of fusion/delamination."

Images are some of the most commonly obtained data sources in materials science and are taken from a wide range of techniques. Light optical microscopy, scanning electron microscopy, and transmission electron microscopy images are all collected to study material structure. Materials processing images may include computed tomography radiographs and/or 3D reconstruction of a melt pool and thermal measurements using two-color pyrometry. Images can be treated as a data point on their own, but they are often analyzed to extract other data such as measuring grain size from light optical microscopy or categorizing crystal structure from a transmission diffraction pattern. 

Data can also be esoteric, depending upon the problems within AM that are being addressed. For example, a vector field of particle flow from a computational fluid dynamics simulation can be considered data. The orientation distribution function of the material's texture can also be considered data. The 3D model and slicing path used to generate an AM part can be considered data. Limitations on what constitutes ``data'' in a materials science problem are not worth defining. Rather, it is more important to consider how data can be featurized for use with an ML algorithm, as this ability determines whether or not data are amenable to use for machine learning approaches.
\subsubsection{Featurization of Qualitative Image Data}
The intuition gained from human interpretation of images is not always encodable in a way that is compatible with models. A human might make a complicated interpretation of an image, such as ``the microstructure is mostly acicular, with some prior grains present, and a string of pores near the boundary." Surely, such a description conjures an image in the reader's head of what such a microstructure might look like. Encoding the same information into a finite element mesh of a microstructure isn't necessarily straightforward. It can be accomplished, for sure, but transforming an observed microstructure into quantitative information requires time-consuming methods. 

Image recognition algorithms are making strides at automating quantification of information from images, while also being able to produce qualitative descriptions of the images. Information such as the size of grains, orientation of grains, presence of cracks or pores, different phases, and more can be automatically identified through computer vision. DeCost et al. have made strides in turning metallographs into sets of quantitative information \cite{DeCost2015, DeCost2017, DeCost2017a}. 

In one particular example, DeCost et al. utilize scale invariance feature transform (SIFT) in order to identify possible features in images. The SIFT algorithm is a widely used algorithm for feature identification and is implemented in many open-source packages, such as OpenCV for C++ or Python \textbf{cite OpenCV}. SIFT relies on identifying regions of maximal and minimal intensity in successively blurred versions of an image. The gradient of pixels are regions of extremal intensity are then computed, resulting in a keypoint descriptor which encodes the size and orientation of a feature, as well as generates a descriptor for identifying that object. The idea behind SIFT is that similar features across images will have similar descriptors. This way, common features can be identified across images. 

DeCost et al. performed SIFT on a database of microstructure images across several alloys \cite{DeCost2015}. The ``features'' that were identified across \textit{all} images were then clustered by $k$-means clustering to identify a dictionary of features. Human investigation can assign qualitative labels to clusters, such as ``$\alpha$ grain" or ``pore" or ``grain boundary." Now, new images can be analyzed with SIFT and their features can be compared to the dictionary of human-interpretable feature descriptors. 

Not only does this process partially automate analysis of images it also automates translation of image-level information into quantitative information. Grain size can be measured by the number of pixels in an identified grain. Orientation of grains in the image can be determined from the values of the keypoint descriptor. The number of pores in an image can be counted as the number of features which match ``pore" in the dictionary. Of course, a human could perform all of this with a ruler and their eyes. However, a computer can perform the same process much faster and on many more images. 

Chowdhury et al. took a more expansive approach to performing feature identification in microstructures. In particular, they were looking to classify microstructures as either dendritic or non dendritic. Chowdhury employed 8 different feature identification methods for a dataset of images. Classification was performed using support vector machines (SVM), Na\"ive Bayes, nearest neighbor, and a committee of the three previous classification methods \cite{Chowdhury2016}. Chowdhury's wide approach to image classification compared the predictive ability of all combinations of feature identification and classification methods, achieving classification accuracies above 90\%. 
\subsection{The Assumptions Behind Machine Learning}
Two assumptions are necessary when using machine learning:
\begin{enumerate}
\item \textit{The Similarity Hypothesis}: Parts manufactured at similar points in the design space will have similar properties.
\item \textit{The Relational Hypothesis}: A correlative relationship exists between the data input to the model and the response of the system.
\end{enumerate}

The similarity hypothesis is used to compare data and datasets, as well as to search and optimize through regression and classification algorithms; it is dependent upon mathematical tools for comparing similarity in a sensible way. Certain data types and data relationships can make similarity interpretation difficult. This is an active area of research within the data sciences

The second hypothesis is required for finding regression and classification functions that are physically accurate.

There are two types of machine learning covered in this review: unsupervised and supervised. Unsupervised learning will find trends in a dataset that are indicative of the underlying behavior. Supervised learning will learn a function $f(\mathbf{x}) = y$ that encodes part of the PSPP relationship.




\subsection{Unsupervised Machine Learning}\label{unsupervised}

Unsupervised machine learning algorithms are used to identify similarities or draw conclusions from unlabeled data by relying on the similarity hypothesis. Unsupervised approaches are useful for visualizing or finding trends in high dimensional data sets, screening out irrelevant modeling inputs, or finding manufacturing conditions which produce similar material properties. 

Consider an experiment that varies three different manufacturing inputs $x_1, x_2, x_3$ and measures a single material property $y$.
In matrix form, the data are expressed as:

\eqn
\begin{split}
\mathbf{X} &= \begin{bmatrix}
	x_{1,1} & x_{2,1} & x_{3,1} \\
	x_{1,2} & x_{2,2} & x_{3,2} \\
	\vdots & \vdots & \vdots \\
	x_{1,m} & x_{2, m} & x_{3, m} \\
	\end{bmatrix} \\
\mathbf{Y} &= \begin{bmatrix}
	y_1 \\
	y_2 \\
	\vdots \\
	y_m \\
	\end{bmatrix} \\
\end{split}\label{initialmeasure}
\equ

where $x_{i,j}$ is the $j^{th}$ measurement of the $i^{th}$ manufacturing input. A distance metric can be defined between data points in the design space. For example, data can be collected at two points $\mathbf{a} = (x_{1}, x_{2}, x_{3})$ and $\mathbf{b} = (x_{1} + \delta, x_{2}, x_{3})$. The $\ell _2$ norm of $\mathbf{a}-\mathbf{b}$ yields

\eqn
|| \mathbf{a} - \mathbf{b}||_2 = \delta.
\equ

The value and magnitude of $\delta$ gives an inclination about how similar $\mathbf{a}$ and $\mathbf{b}$ are.
If $\delta$ is close to zero, then a researcher can say that they are similar, or even the same if $\delta$ is exactly zero.
As $\delta$ becomes larger a researcher can say $\mathbf{a}$ and $\mathbf{b}$ become more dissimilar.
The concept of `similar' manufacturing conditions may be easy to assess by an experimentalist when tuning only a few parameters at a time.
When taking into consideration tens or hundreds of design criteria, sometimes with correlated inputs, elucidating similar manufacturing conditions becomes difficult.
This vector distance approach is a simple, yet effective first glance at similarity in a design space and is generalizable to $n$ many design criteria.

Let us say that $\delta$ is small and that $\mathbf{a}$ and $\mathbf{b}$ are similar manufacturing conditions.
Now, consider a third point in the design space $\mathbf{c} = (x_{1} + \delta, x_2 + \delta, x_3)$ that has not yet been measured.
Since $\mathbf{c}$ was manufactured at similar conditions to $\mathbf{a}$, as measured by $||\mathbf{c} - \mathbf{a}||_2 = 2\delta$, then we may say that $\mathbf{a}$, $\mathbf{b}$, and $\mathbf{c}$ are all similar to each other. If the similarity hypothesis is correct then manufacturing with conditions $\mathbf{a}$, $\mathbf{b}$ and $\mathbf{c}$ should yield similar measurements of $y$.

At some point, a researcher will have a set of initial manufacturing inputs $\mathbf{a}$, $\mathbf{b}$, $\mathbf{c}$, $\mathbf{d}$, etc., and associated property measurements that have been tested.
Churning through the remainder of all possible manufacturing conditions becomes expensive and tedious quickly.
Instead, researchers can use similarity metrics to determine whether or not a future test is worth running.
Comparing the manufacturing inputs through vector distance gives a rough idea of the possible outcome before spending time and resources on running a test.
If the intent is exploring design spaces then manufacturing at conditions \textit{furthest away} from previously observed points may be the answer.
If looking for local maxima of quality, an operator would want to manufacture at conditions \textit{nearest to} the conditions currently known to have high quality.

Using vector distances as metrics of similarities can produce results that are analogous to creating process maps \cite{Beuth2001}.
Process maps are used to divide $2$ dimensional plots of manufacturing inputs into regions of quality, or regions of different material responses. A common application of unsupervised learning is finding clusters in data sets which produce useful partitions of material behavior. Similarly to how process maps define boundaries between material performance and response, clustering with unsupervised learning can identify manufacturing conditions which will result in similar material performance.


%The following demonstration is based on $k$-means clustering, a commonly used unsupervised machine learning clustering algorithm.

%A researcher has acquired the datasets in Eqn. \ref{initialmeasure} and wants to partition $\mathbf{Y}$ into groupings of high quality parts and low quality parts.
%However, there are several values of $y \in \mathbf{Y}$ that lie between two extremes and the cutoff for quality is not well defined.
%It would be useful to use similarity metrics to find the best possible partition of quality.
%To begin, the data set is partitioned randomly into two groups, $\mathbf{Y}_1$ and $\mathbf{Y}_2$.
%The centroids $m_1$, $m_2$ (or centers of mass, in engineering) of each grouping can be calculated as
%
%\eqn
%	\begin{split}
%		m_1 & = \frac{1}{|\mathbf{Y}_1|} \sum_{y_j \in \mathbf{Y}_1} y_j \\
%		m_2 & = \frac{1}{|\mathbf{Y}_2|} \sum_{y_j \in \mathbf{Y}_2} y_j. \\
%		\label{moment}
%	\end{split}
%\equ
%
%where $|\mathbf{Y}|$ is the average value of a grouping.
%The measurements were randomly partitioned at first; the goal is to re-partition each set so that similar measurements (similar levels of quality) are in the same set.
%To do this, we can re-assign each set by
%
%\eqn
%	\begin{split}
%		\mathbf{Y}_1 & = \{y_i : ||y_i - m_1||_2 \leq ||y_i - m_2||_2 \} \\
%		\mathbf{Y}_2 & = \{y_j : ||y_j - m_2||_2 \leq ||y_j - m_1||_2 \}. \\
%	\end{split}
%	\label{reassign}
%\equ
%
%We can interpret the re-assignment in Eqn. \ref{reassign} physically: if a measurement initially assigned to set $\mathbf{Y}_1$ is closer in distance to $\mathbf{Y}_2$ then it is \textit{more similar} to the other set.
%Thus, it is re-assigned.
%Since the original partition was random it is likely that there are low quality parts mixed in with high quality parts - in other words, outliers exist in each partition.
%Measuring the similarity of each data point to the mean of the groupings re-classifies these outliers into groupings that are more reflective of their quality.
%
%Once re-assignment is complete the centroids in Eqn. \ref{moment} can be re-calculated and updated.
%Then, data points are re-assigned once more based on how similar they are to the centroid of each partition.
%If we have partitioned the input settings $(x_1, x_2, x_3)$ along with their corresponding measurements, then we have lists of input settings which are likely to give good/bad quality parts.
%Further analysis can also be conducted, such as analyzing which regimes of inputs lead to good or bad quality - this is precisely what process maps represent.
%The difference in this case is that $n$ many manufacturing conditions can be related to a quality metric simultaneously, with little to no human inspection or intervention.
%Additionally, a researcher can dig further and analyze \textit{why} groups of input settings result in given quality for a material property.
\subsection{Supervised Machine Learning}
In a \textit{supervised machine learning algorithm} the goal is to determine a functional relationship $f(\mathbf{x}) = \mathbf{y}$ based on previous measurements of $\mathbf{y}$ at points $\mathbf{x}$ in the design space. That is, supervised machine learning algorithms relate manufacturing inputs to labeled output data. 

Functional mappings of input data $\mathbf{x}$ to process outcomes $\mathbf{y}$ can take the form of either regression or classification. In a regression problem, the goal is to find mappings between inputs $\mathbf{x}$ to continuous values of $\mathbf{y}$. An example includes predicting mechanical strength from processing conditions, where the process conditions can be continuous or discrete, like those in Table \ref{table:design_space}, and the output $\mathbf{y}$ can be any reasonable value of strength. A classification problem sorts inputs $\mathbf{x}$ into categories with associated labels. These classifications can be binary or one-of-many classes. An example would be training an algorithm to answer the question ``Will the build fail?" based on processing inputs, with the possible class labels being ``Yes" or ``No."


Functional relationships can take many forms, depending on the specific supervised ML algorithm being used. One method is to model the relationships as a vector product
\eqn
\mathbf{X}\beta = \mathbf{Y}.
\label{map}
\equ
where $\beta$ is a vector of coefficients that weight the machine inputs to approximate an entry in $\mathbf{Y}$. 

A researcher usually seeks this relationship through the measurements they have observed; in this case, the measurements are stored in the matrices of Eqns. \ref{matrix} \& \ref{outputs}.
A common method to find a vector representation of $\beta$, and a critical element in most machine learning algorithms, is through least squares regression. Least squares regression finds $\beta$ through a minimization problem, given by
\eqn
\min || \mathbf{X}\beta - \mathbf{Y} ||_{2}^{2}.
\label{leastsquares}
\equ
Equation \ref{leastsquares} can be interpreted analogously to similarity measurements for unsupervised algorithms: the closer that $\mathbf{X}\beta - \mathbf{Y}$ is to zero, the more similar $\mathbf{X}\beta$ is to $f(\mathbf{x})$.

The methods of solving equation \ref{leastsquares} are many and varied; indeed, much of this review will focus on finding solutions to Eqn. \ref{leastsquares} for various problems throughout additive manufacturing.
The result is an approximation to the functional relationship $f(\mathbf{x}) = \mathbf{y}$.
A new point of interest in the design space $\mathbf{x'}$ can be chosen and its associated material property $\mathbf{y'}$ can be predicted by computing

\eqn
\mathbf{x'}\beta = \mathbf{y'}.
\equ
This simple example demonstrates how functional relationships can elucidate more information about design spaces from previously generated data.



\subsection{Error Metrics}\label{errormetrics}
Models that are used to predict values, whether numerical regression or classification algorithms, must have metrics to assess success. There are a multitude of error metrics that are used in the machine learning community. Different error metrics provide different information about the model, such as its ability to predict mean values, its robustness against outliers, and uncertainty in predictions, amongst other information. Many different error metrics have been formulated by the statistics community and used by the ML community \cite{Navidi2006}. Here, we review many of the most commonly used error metrics. For readers interested in more in depth discussion and examples, the website DataQuest provides an open source article about common error metrics\cite{DataQuestError}, explanations of commonly used error metrics and their benefits/drawbacks can be found in Table V of Shan et al\cite{Shan2010}, and Botchkarev wrote a review article detailing different error metrics used by the machine learning community over time\cite{Botchkarev}.

The following parameter definitions are used in the ensuing basic introduction of common error metrics and the remainder of the article.
\begin{itemize}
	\item $\hat y$ -- the value predicted by a regression algorithm $f(\mathbf{x}) = \hat y$
	\item $y$ -- the actual value of a material process/structure/property at input location $\mathbf{x}$
	\item $n$ -- the sample size used to train a machine learning algorithm
\end{itemize}

The mean absolute error (MAE) assesses the absolute residual between the predicted value of a regression problem and the actual value. It is calculated as the absolute difference between predicted value $\hat y$ and the actual value $y$, normalized by the sample size. Stated mathematically, the mean absolute error is 
\begin{equation}
	\text{MAE} = \frac{1}{n} \sum_{i=1}^n \left|y_i - \hat y_i \right|.
	\label{MAE}
\end{equation}
MAE penalizes error linearly. The MAE penalizes outliers in the data with the same magnitude as data points lying close to the mean. The mean absolute error can also be changed into a percentage, the mean absolute percentage error (MAPE) by normalizing each individual error measurement against the actual value $y$. Stated mathematically, 
\begin{equation}
	\text{MAPE} = 100 \times \frac{1}{n} \sum_{i=1}^n \left|\frac{y_i - \hat y_i}{y_i}\right|.
	\label{MAPE}
\end{equation}

Other error metrics highlight the impact of outliers on the dataset. The mean squared error (MSE) squares the difference term in Eqn. \ref{MAE} to produce
\begin{equation}
	\text{MSE} = \frac{1}{n} \sum_{i=1}^n |y_i - \hat y_i|^2.
	\label{MSE}
\end{equation}
The MSE penalizes error quadratically. Outliers in the dataset will have a much larger impact on MSE than they will on the MAE. A downside of the MSE is that the errors are reported as the square of the units being predicted by the model. Some users wish to have an error with the same units as the value being predicted; thus the root mean squared error (RMSE) adds a square root such that
\begin{equation}
	\text{RMSE} = \sqrt{\frac{1}{n} \sum_{i=1}^n |y_i - \hat y_i|^2}.
	\label{RMSE}
\end{equation}

All of the above metrics produce a measure of the absolute value of the error in the model. In some cases it is useful to know if a model is \textit{over} predicting the value (negative error) or \textit{under} predicting the value (positive error). In these cases, the MAE can be modified to the mean percentage error (MPE), given as
\begin{equation}
	\text{MPE} = \frac{1}{n} \sum_{i=1}^n \left( \frac{y_i - \hat y_i}{y_i}\right).
	\label{MPE}
\end{equation}
The MPE can reveal if a machine learning prediction algorithm is skewed towards certain types of values.

All of the above error metrics are suitable for regression problems with continuous value of $\hat y$. In the case of a classification problem, where the outputs are non-numerical, a non-numerical method of measuring error must be defined. While several methods have been developed \cite{Metrics2018, Metrics2019} a common method is to use a \textit{confusion matrix}. A confusion matrix displays the percentage of classifications that were correctly identified, as well as the percentage of classifications made to the wrong class. An example confusion matrix can be seen in Figure \ref{confusionmatrix}. In this figure, the main diagonal of the figure displays the percentage of data points that were correctly identified by class. The off-diagonal components display when a certain class was mis-identified as another class and how often it occurred.

\begin{figure}
	\includegraphics[width=1\linewidth]{/Users/njohnson/git/thesis/document/chapters/review/Images/Fig2_confusionmatrix}
	\caption{A confusion matrix used in a study by DeCost and Holm \cite{DeCost2015}. The goal of the study was to classify materials based on images of their microstructures. The main diagonal of the matrix represents correct classifications. In the case of the upper-leftmost entry, 11 images of ductile cast iron were correctly identified as ductile cast iron. The upper-rightmost entry indicates that 1 image of ductile cast iron was incorrectly classified as a superalloy.}
	\label{confusionmatrix}
\end{figure}


\subsection{The Bias-Variance Tradeoff and Model Validation}\label{bvar}


\begin{figure*}
	\includegraphics[width=0.75\linewidth]{/Users/njohnson/git/thesis/document/chapters/review/Images/Fig3_bias_variance_tradeoff.png}
	\caption{Illustrations of high bias and high variance models. A toy dataset was generated from the polynomial $y=5+0.1x+0.1x^2+0.1x^3+0.002x^4+ \text{Random Noise}$. The fits in a) and b) are both parameterizations of a model. Each model (line) in both fits has approximately the same error but does not accurately capture the behavior of the data due to poor model assumptions; in this case, fitting a first order polynomial to a dataset generated from a fourth order polynomial. This is an example of high bias models. A twentieth order polynomial was fit to a subset of the full dataset in c), shown in blue. While the model has very good predictive error for the training dataset it will not extrapolate well to the data in the testing set; this is overfitting or high variance. A third order polynomial was fit to the data in d) demonstrating a good balance between bias and variance. The model accurately captures trends in the data while not overfitting the training dataset.}
	\label{biasvariance}
\end{figure*}

Now that basic methods of machine learning and associated error metrics have been defined we proceed to introduce how machine learning models are fit and validated. The following discussion focuses on finding parameters to fit a machine learning algorithm, how those parameters are validated, and common obstacles that arise in validating the model.

The cost function\footnote{Also sometimes called the loss function or reward function depending on if the objective is to minimize or maximize the value \cite{CostFunction}.} $C({\bf x}; \boldsymbol{\theta})$, is the metric that quantifies the cost of a particular model parameterization. That is, for every input dataset $\mathbf{x}$ there is an associated set of parameters $\boldsymbol{\theta}$ for the machine learning model that best fit $\mathbf{x}$ to their associated outputs $\mathbf{y}$. The training step is concerned with finding the model parameterization that minimizes or maximizes the cost, depending on the application. There are many different choices for cost function and each machine learning algorithm will use its own specific methodology. Perhaps the best known loss function is the squared loss, given in Equation \ref{leastsquares}. 

The loss function is used to minimize a \textit{parameterization} of a machine learning algorithm. For example, a least squares regression algorithm is parameterized by the weighting constants $\beta_i$,
\begin{equation}
	\hat y = \beta_0 + \beta_1 x + \beta_2 x^2 + \ldots + \beta_n x^n.
	\label{beta}
\end{equation}
The model parameters are the weights $\beta_i$ that are fit to the linear regression. The goal of training a machine learning algorithm is to find model parameters that minimize the loss function. If the values of $\beta_i$ are optimally chosen then the value of $\mathbf{X}\beta-\mathbf{Y}$ in Eqn. \ref{leastsquares} should be minimized. The actual method of performing this optimization can take on many forms and is discussed in-depth elsewhere. The scikit-learn package, part of the Python scipy library, provides many methods for optimization of cost functions\cite{ScipyOptimization}. Gradient descent is a common method for performing cost function optimization\cite{GradientDescent2017}. An article by Brochu et al. discusses optimization of cost functions using Bayesian optimization, an important topic in modern statistics \cite{Brochu2010}.

Certain machine learning methods -- such as neural networks, decision trees, and ridge regression -- also have model \textit{hyperparameters}. These parameters define aspects of the model itself, not aspects of a specific parameterization of the relationships between $\mathbf{x}$ and $\mathbf{y}$. For linear regression of a polynomial function to a dataset the weights $\beta_i$ are model parameters and the order of the polynomial is a model hyperparameter. Hyperparameters will be discussed more in-depth later as specific machine learning algorithms are introduced in Section III.

All machine learning models follow a basic training and validation process:
\begin{enumerate}
    \item Divide data into training, test, and validation data: $\{{\bf X}, {\bf y}\} \to \left\{ \{{\bf X}, {\bf y}\}_{\rm train}, \{{\bf X}, {\bf y}\}_{\rm test}, \{{\bf X}, {\bf y}\}_{\rm validate}\}  \right\}$.
    \item Estimate the model parameters, $\hat{\boldsymbol{\theta}}$, using $\{{\bf X}, {\bf y}\}_{\rm train}$ using an appropriate cost function. 
        \item Adjust the model hyperparameters using $\{{\bf X}, {\bf y}\}_{\rm test}$ based on the accuracy of the best fit parameterization of $\boldsymbol{\theta}$.
    \item Validate the best parameterization and check against over- or under-fitting by evaluating the model on the validation set $\{ {\bf X}, {\bf y}\}_{\rm validate}$.
\end{enumerate}
These steps are repeated until the model performance, as measured by the model error estimate, converges.

As the complexity of the model increases -- such as the complexity of a polynomial in a linear regression problem --  so does the tendency of that model to overfit to the training data and generalize poorly to unseen inputs, leading to an increase in the out-of-sample error. This balance between the ability of the model to represent the inherent complexity between the input and output spaces (i.e., reduce the \emph{model bias}) while minimizing the out-of-sample error (i.e., reduce the \emph{model variance}) is the basis for the \emph{bias-variance tradeoff} that is central to all machine learning models. Visual examples of overfitting, underfitting, and proper fitting can be seen in Figure \ref{biasvariance}. The goal in validating a machine learning model is to find a balance between overfitting the training dataset and underfitting the testing dataset, as shown in Figure \ref{tvtrmse}. While the RMSE shows a decrease in the training dataset as model complexity increases, the RMSE of the testing dataset increases significantly.

Overfitting and selection bias can be sussed out through use of \textit{cross validation}. Cross validation is the process of training machine learning models on subsets of the training set and evaluating with the remaining data to see how sensitive the model performance is to the choice of different inputs. Cross validation is often referred to as $k$-fold cross validation because the machine learning model is trained on $k$ different subsets. 

\begin{figure}
	\includegraphics[width=1\linewidth]{/Users/njohnson/git/thesis/document/chapters/review/Images/Fig4_train_vs_test_rmse.png}
	\caption{The calculated root mean squared error for six different models fit to the dataset shown in Figure \ref{biasvariance}. The training data continues to decrease as the models become more complex, demonstrating overfitting. However, when the model is evaluated against a test dataset the RMSE increases significantly with more complex models.}
	\label{tvtrmse}
\end{figure}

In $k$-fold cross validation the training data set is randomly split into $k$ different groups or \textit{folds}. The machine learning model is trained on $k-1$ of the folds and tested on the $k^{\text{th}}$ fold. For each training and testing set the fit model parameters and associated error should be kept in order to assess how each parametrization of the model performs. 

Randomizing, training, and validating on multiple subsets of the data elucidates the model's ability to perform on new datasets. If a model is suffering from overfitting, or high variance, then it will have very low predictive error on the training set but perform poorly on the testing set. If the model is suffering from high bias then it may demonstrate similar performance metrics between fitting of each $k^{\text{th}}$ set but has high prediction error in general. High bias often results from improper assumptions in the machine learning algorithm or a poor choice of model hyperparameters. Cross validation reveals these behaviors in machine learning models by providing error metrics for models trained on many different subsets. Necessary changes to the model hyperparamters, or even changes in machine learning modeling used, can be discovered from cross-validation.

One specific case of cross validation where $k=n$ is called \textit{leave one out} cross validation. In this method the models are trained on all data points except one, then tested on the remaining data point. Leave one out cross validation is especially useful for assessing the impact on outliers of the model performance. 

Another method of cross validation called leave-one-cluster-out (LOCO) cross validation was introduced by Meredig et al. \cite{Meredig2018} for materials science applications. LOCO CV was introduced to highlight problems in the distribution of data in materials datasets. Often, datasets from materials science are limited around specific clusters of material compositions or properties. An example for AM is that most datasets generated focus around weldable alloys like 300 series steels, superalloys, and titanium alloys. As a result the prediction performance of machine learning algorithms may be biased toward these clusters of materials. LOCO CV uses a nearest-neighbor clustering approach -- akin to the example given in Section \ref{unsupervised} -- to evaluate the impact of clustering of material types on prediction performance.

The above methods are for the validation of individual machine learning models. In many cases it is worthwhile to train several different machine learning models on the same problem and assess the best model. As is shown in Table \ref{ML}, several different machine learning algorithms can often be applied to the same task. Because each algorithm has different assumptions, one type of ML model may perform better on a dataset than others. Thus, it is worthwhile to use tools that can compare the performance of different ML models for the same application.

%Figure \ref{ROC} shows a receiver operating characteristic (ROC) curve for several different machine learning algorithms trained on the same dataset. An ROC is used to classify how often a machine learning algorithm makes the correct classification based on its \textit{true positive rate} and \textit{false positive rate}. For a binary classification problem (an output belongs to class A or class B), a classification algorithm can make four types of classifications 
%
%\begin{figure}
%	\includegraphics[width=1\linewidth]{Images/Liu_ROC}
%	\caption{}
%	\label{ROC}
%\end{figure}
\subsection{Comparison Across Machine Learning Approaches}\label{comparison}
The validation of a single machine learning model can be addressed by the methods presented in Sections \ref{errormetrics} \& \ref{bvar}. Finding the best possible parameterization of an individual model does not guarantee that a researcher has found the best possible solution to their specific problem. It is generally good practice to evaluate several machine learning approaches to a problem and choose the best approach across all algorithms that may be reasonably expected to perform.. Table \ref{ML} shows that many different algorithms can be used for the same types of problems. Different algorithms may have vastly different performance even for the same problem or dataset.

For example, Principal Component Analysis (PCA) and kernel ridge regression (KRR) can both be used as regression tools; PCA relies on the assumption of linearity between inputs and outputs while KRR does not. Often, a researcher might not know the if the relationship being studied is linear or not and therefore should try both options to see which produces a better result.

In general, researchers can follow a few steps to determine which model is best for their additive manufacturing problem:
\begin{itemize}
	\item Evaluate if there are statical correlations in the data of interest 
	\item Pre-process and featurize data for use with a machine learning algorithm
	\item Tune the model parameterization and hyperparameterization through error analysis and cross validation
	\item Compare error metrics across several algorithms and select one algorithm as the best performer
\end{itemize}

Regression models can be validated against each other using the error metrics in Section \ref{errormetrics}. It is important to use multiple error metrics for comparison because different machine learning algorithms handle outliers and statistical correlations differently. For classification problems, a graph called a receiver operating characteristic (ROC) curve has been developed to compare the classification success of different algorithms. An example ROC curve can be seen in Figure \ref{ROC}. The ROC curve compares the true positive and false positive classification rates for a binary classifier, a group of problems whose solution can take one of two outcomes. To ensure that Type I error (false positive) accurately reflects the performance of the model, the less common outcome should always be taken as the True condition, and the more common outcome as the False condition. (Footnote. Although restricted to binomial classification, the ROC curve may be extended to multinomial classification by recursion. That is, A or not A; and if not A, then B or not B; and if not B, then C or not C; etc. where A, B, C, etc. are all potential outcomes in order of increasing frequency.)More information on ROC curves can be found at Google's developers page \cite{GoogleROC}.

Tools to compare across machine learning algorithms are invaluable and should be considered as a mandatory part of any machine learning approach. It is often the case that evaluating many machine learning algorithms against each other will lead to better overall performance because the best approach can be chosen from many. The ML packages listed in the next section all contain tools for comparing machine learning algorithm performance.

\begin{figure}
	\includegraphics[width=1\linewidth]{/Users/njohnson/git/thesis/document/chapters/review/Images/Fig5_Liu_ROC}
	\caption{An example receiver operating characteristic curve from the work of Liu et al. \cite{Liu2020}. The goal of the study was to class material properties based on additive manufacturing machine inputs. The classes were regimes of material quality like ``high density" or ``low density." The dataset was built by mining data from literature on additively manufactured metals. The area under the curve (AUC) shows the integrated area under each algorithm's ROC curve; a perfect classifier has AUC$=1$. In the example shown, Na\"ive Bayes significantly outperforms the other two algorithms and thus is the best choice of machine learning approach for this problem.}
	\label{ROC}
\end{figure}
\subsection{Machine Learning Toolboxes}
Most of the machine learning algorithms and approaches discussed in this review are, in some form, free and openly accessible. Many machine learning packages exist across many different programming languages and platforms. Table \ref{data_tools} highlights a variety of computational tools and packages and their relevance to AM synthesis optimization. 

\begin{landscape}
           \renewcommand{\arraystretch}{0.8}
    \setlength{\tabcolsep}{5pt}
    \begin{center}

        \begin{longtable}{p{4cm}p{4cm}p{11cm}} 
                \endfirsthead
                \caption{Commonly-used machine learning, statistical analysis, and computer vision toolboxes. Some toolboxes listed are open source, while some are packaged with commercial software like MATLAB. \label{data_tools}} \\ \hline
           
            Language/Platform & Package & Applications  \\ 
           % \endhead


            \hline
            Python & scikit-learn \cite{sklearn} & General data mining toolbox; packages for classification, regression, clustering, dimensionality reduction, model selection, and data pre-processing. \newline \\
            		& tensorflow \cite{tensorflow} & Machine learning toolkit for data mining and data flows; specifically focuses on the use of neural networks and deep learning for model building and problem solving. \newline \\
			& keras \cite{keras} & Deep learning-specific machine learning toolbox; designed for intuitive building of neural network systems. \newline \\
			& OpenCV \cite{opencv} & Algorithm toolbox for machine learning and computer vision; contains wide range of tools for image processing including image pre-processing, template matching, object identification, and convolutional neural networks. \newline \\
			
	   MATLAB & Statistics and Machine Learning Toolbox \cite{matlabml} & Commercial data analysis and machine learning toolbox with a wide range of applications in data analysis including clustering, classification, regression, and dimensionality reduction. \newline \\
	   		&\raggedright Computer Vision Toolbox\cite{matlabcv} & Algorithm toolbox for machine learning and computer vision; contains tools for a wide range of image analysis including pre-processing, object identification, template matching, and convolutional neural networks. \newline \\ 
			
	  C $++$ & OpenCV \cite{opencv} & Algorithm toolbox for machine learning and computer vision; contains wide range of tools for image processing including image pre-processing, object identification, template matching, and convolutional neural networks. \newline \\
	  	& tensorflow \cite{tensorflow} & Machine learning toolkit for data mining and data flows; specifically focuses on the use of neural networks and deep learning for model building and problem solving. \newline \\ 
		
	R & Machine Learning in R (MLR) \cite{mlr} & Infrastructure for incorporating common machine learning functions in R in an easy way; provides robust packages for a wide range of machine learning-based tools including regression, classification, clustering, sampling methods, model optimization and more; has built in parallelization methods. \newline \\  \hline
	
            \bottomrule
        \end{longtable}

    \end{center}
    
\end{landscape}





