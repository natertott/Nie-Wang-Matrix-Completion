
    \renewcommand{\arraystretch}{0.8}

    \setlength{\tabcolsep}{5pt}
\begin{table}
\begin{center}
 \caption{A possible design space for laser powder bed fusion additive manufacturing. There are over $10^4$ possible combinations of machine inputs, based on the listed ranges and step sizes. Any possible combination of these parameters is a point in the design space.}
         \label{table:design_space}        
         \begin{tabular}{c|c|c|c}       
            \toprule
            \hline
             Parameter & range & step size & levels \\ \midrule
            \hline
            Power & 100-200 W & 10 W & 10 \\
            Scan speed & 500-1000 mm/s & 100 mm/s & 5 \\
            Spot size & 50-100 $\mu$m & 10 $\mu$m & 5 \\
            Energy density & 1-5 J/mm$^2$ & 1 J/mm$^2$ & 5 \\
            Sample Build Direction & 0-180$^\circ$ & 90$^\circ$  & 3 \\
            Amount of recycled powder & 0-100\% & 10\% & 10 \\
            Hatch spacing & 0.1-0.50 mm & 0.1 mm  & 5 \\
            \hline
            \bottomrule
        \end{tabular}
\end{center}
\end{table}
 
 
\subsection{The Design Space of Additive Manufacturing}
The \textbf{design space} of metals AM is the set of all PSPP relationships. More specifically, the term `design space' will be used throughout this article in reference to the set of AM data that is used and calculated by machine learning algorithms. An example design space for laser powder bed fusion (LPBF) of metals, the most industrially prolific of current metals AM technologies, is graphically depicted in Figure \ref{AMgene}. A complementary example of a process design space of LPBF is given in Table \ref{table:design_space}. Observable process phenomena may link the manufacturing parameters to the resulting materials properties, hence they may also be used to augment the manufacturing parameters and material properties within the design space. Examples include melt pool morphology, temperature history, and cooling rates.

A single combination of process parameters, observed process phenomenon, measured material properties, and a part's performance can be considered as a \textit{coordinate}, or point, in the design space. Single coordinates, defined this way, can sometimes lead to a multitude of material properties due to latent variables, unforseen complications, and the stochasticity of the process. Explicit consideration of process phenomenon in the design space coordinate can be used to more accurately establish unique points within the design space. In summary, any part that is processed under a single set of conditions and is observed to have a set thermal history and set of material properties can be considered to be manufactured \textit{at that point} in the design space.

While the design space of AM is vast, data cannot always be given to machine learning algorithms `as is.' It is important to consider the sources of data in the design space and how they need to be changed or curated for use with ML.
