\subsection{The Assumptions Behind Machine Learning}
Two fundamental assumptions underpin the use of machine learning:
\begin{enumerate}
\item \textit{The Relational Hypothesis}: A correlative relationship exists between the data input to the ML model and the response of the system.
\item \textit{The Similarity Hypothesis}: Similar points in the design space will have similar properties.
\end{enumerate}


The relational hypothesis is a foundation for predictive models: after all, no prediction is possible in the absence of a correlative relationship between input and response.
 
The similarity hypothesis supposes that data are comparable: that according to some measure of similarity, similar input will produce similar output. 

There are two types of machine learning covered in this review: unsupervised and supervised. Unsupervised learning will find trends in a dataset that are indicative of the underlying behavior. Supervised learning will learn a function $f(\mathbf{x}) = y$ that encodes part of the PSPP relationship. We proceed to walk through toy examples of each type; keep in mind that these are simplified examples meant to provide intuition behind the uses of machine learning. Scientists and engineers should research machine learning models, their uses, and their specific underlying assumptions before applying them.
