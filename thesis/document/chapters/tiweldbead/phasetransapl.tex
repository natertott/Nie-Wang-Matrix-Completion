\begin{figure*}[t]
    \centering
    \includegraphics[width=\linewidth]{/Users/njohnson/git/thesis/document/chapters/tiweldbead/tiweldbead_images/plane_spacing.pdf}
    \caption{The lattice parameter of the $\beta$ and $\alpha'$ phases from initial solidification of the weld. Initially, the $\beta$ phase lattice parameter decreases with cooling until the $\alpha'$ phase forms. At this point, the $\beta$ phase lattice parameter is constant due to constraint by the $\alpha'$ phase forming. When the $\alpha'$ phase forms its $(0001)$ plane shares a d-spacing with the $\beta (110)$. However, it rapidly relaxes to a more equilibrium d-spacing, as shown by the change in the $a$ and $c$ parameter.}
    \label{fig:latparams}
\end{figure*}

Rapid solidification of alloys has recently become a central topic in materials science and engineering due to the adoption of new advanced manufacturing techniques. Additive manufacturing and advanced robotic welding have enabled manufacturing of advanced engineering components with geometries and properties that are un-achievable with traditional casting and forging. Many of the boons of advanced manufacturing are in additive manufacturing of aerospace components where unique lightweight geometries can be created to reduce fuel consumption during flight. Ti-6Al-4V is one of the most commonly used aerospace alloys due to its high strength-to-density ratio and relatively high thermal stability. 

Ti-6Al-4V is also one of the most commonly scrutinized additive alloys because of the wide range of microstructures that can form during manufacturing. Rapid solidification of Ti-6Al-4V combined with high thermal gradients and repeated thermal cycling results in a wide range of microstructural characteristics related to grain morphology \cite{Semiatin1997, Ahmed1998, Plichta1977, Beladi2014, Katzarov2002}, alloy partitioning \cite{Szkliniarz1995, Fan2005, Boivineau2006, Sridharan2019}, phase fractions \cite{Tan2016}, and more. High variance in microstructure causes uncertainty in mechanical performance. Engineering the microstructure of additively manufactured Ti-6Al-4V is predicated on an understanding of how the microstructure forms, especially understanding transformation pathways to different phases and morphologies of grains. Ti-6Al-4V is an $\alpha/\beta$ alloy containing an HCP $\alpha$ and a BCC $\beta$ phase. The mechanisms of phase transformation and the final morphologies of grains is dependent on the thermal history a part experiences during manufacture. The rapid solidification process has been documented to produce a martensitic (diffusionless) transformation upon rapid cooling.

For the sake of further discussion, the low temperature HCP phase in Ti-6Al-4V will be referred to as the $\alpha$ phase. Any BCC phase of Ti-6Al-4V, whether at low temperature or elevated temperature, will be referred to as the $\beta$ phase. When the HCP phase is formed through a martensitic transformation it will be referred to as the $\alpha'$ phase.

Elmer and Palmer largely pioneered the study of Ti-6Al-4V phase transformations using in situ high energy X-ray diffraction. In a series of papers published in the mid 2000s they characterized the transformation from $\alpha$ to $\beta$ at relatively slow cooling rates ($1^{\circ}$ C/s - $10^{\circ}$ C/s) as well as in rapid solidification conditions ($1000^{ \circ}$ C/s or more). Several aspects of the transformation were characterized. Elmer and Palmer claim that a change in the $\beta$ lattice parameter is due to element partitioning during the formation of $\alpha$ phase. It has been well characterized that during diffusional formation of $\alpha$ from a parent $\beta$ grain, V partitions to the $\alpha$ grain boundary while Al remains trapped in the $\alpha$ phase \cite{Williams1967, Sridharan2019, Tan2015, Nandwana2019}. Elmer and Palmer claim that the movement of V to the $\beta$ phase can be observed by a shrinking of the $\beta$ lattice parameter at a temperature of around $600^\circ$ C/s \cite{Elmer2005}. This is consistent with other findings that an increase of V in BCC Ti alloys shrinks the lattice parameter \cite{Xu2015}.

Elmer and Palmer also characterized phase transformations of Ti-6Al-4V welds using both spatially and temporally resolved high energy X-ray diffraction techniques \cite{Elmer2004, Elmer2003}. Their characterization of phase transformations at different locations in welds was coupled with the Johnson-Mehl-Avrami-Kolmogorov (JMAK) equation to fit parameters such as temperature and transformation coefficient. In this series of papers they claim to observe a combination of diffusionally formed $\alpha$ phase and martensitically formed $\alpha'$ phase. The cooling rates reported in \cite{Elmer2003} indicate that they were in the martensitic phase transformation regime, however their calculation of transus temperature sometimes places the $\beta \to \alpha$ transformation below the martensite start temperature. Microscopy of the welds revealed a decrease in $\alpha'$ fraction away from the center of the weld indicating regions of slower cooling rate or different transformation temperature.

Accurate modeling of this phase transformation is important for engineering manufacturing conditions of Ti-6Al-4V that prevent $\alpha'$ formation, as well as understanding the impact of $\alpha'$ on mechanical performance. The titanium martensitic transformation that occurs during rapid solidification causes embrittlement of the material \cite{Xu2015}. Many studies report observing martensitically-formed HCP phases in rapidly cooled Ti-6Al-4V but there has been little characterization of the transformation in situ. This paper presents observations of the martensitic phase transformation as it occurs using high energy X-ray diffraction. The formation of the HCP martensitic phase can be observed as a transformation of the high temperature $\beta$ phase $(110)$ peak into the low temperature $\alpha$ phase $(0002)$ peak. 

This letter provides measurements of $\beta$ phase and $\alpha'$ phase lattice parameters during the transformation. Relaxation of the $\alpha'$ phase was observed during its initial formation, followed by further relaxation due to cooling of the material.

This study builds on the previous high energy X-ray diffraction characterization of Ti-6Al-4V work of Palmer and Elmer \cite{Elmer2003, Elmer2004, Elmer2005} as well as the work of Kenel et al. \cite{Kenel2017} and Malinov et al. \cite{Malinov2002}. In the previous investigations, phase changes from the $\beta$ phase field to $\alpha + \beta$ phase field were performed under slow cooling conditions or on weld beads large enough to have a variable cooling rate. This letter advances upon that work by comparing and contrasting phase change mechanisms under rapid cooling conditions in small volumes, i.e. a sample that only experienced a martensitic transformation.

%%%% Figure
\begin{figure*}
    \centering
    \includegraphics[width=\linewidth]{/Users/njohnson/git/thesis/document/chapters/tiweldbead/tiweldbead_images/peak_split_plot2.pdf}
    \caption{Formation of the $\alpha'$ $(0002)$ peak from the $\beta$ $(110)$ peak around $t = 2.945$s after the welder was turned off.. Upon initial formation, the $(0002)_{\alpha'}$ forms at a d-spacing near the $(110)_\beta$ peak. However, it quickly shifts to a more stable unit cell dimension. As the transformation proceeds, the intensity of the $\alpha'$ peak grows and the $\beta$ peak decreases. By the end of the experiment, $t=57.15$s when the sample had reached room temperature there were no observable $\beta$ peaks remaining.
    }
    \label{peaksplit}
\end{figure*}

Phase changes were characterized using high energy X-ray diffraction at the Advanced Photon Source, Argonne National Laboratory, beamline 1ID. Weld deposits were made using a Fronius Cold Metal Transfer welder depositing on a water cooled substrate of Ti-6Al-4V. The welder was controlled robotically and liquid depositions were made on the substrate in the path of the X-ray beam. The wire feed rate was 2.5 mm/s, the electrode current was 50 A on average, and the accelerating voltage was 12 V. Welds beads were deposited with an approximate height and diameter of 2.5 mm. Approximating the weld bead as a sphere results in a volume of 22.45 mm$^3$. 

Data collection began with the unobstructed X-ray beam shone in the direction of the detectors. Then, the CMT welder deposited liquid Ti-6Al-4V in the path of the beam. X-ray scattering occurred through the amorphous material at first, followed by X-ray diffraction as the material solidified. Variable cooling rates throughout the bulk of the sample were a concern, as it is possible that cooling rate decreases away from the water cooled substrate. Repeated measurements were made at different heights above the substrate to observe any changes in cooling rate. No cooling rate differences were observed, in contrast with similar studies performed on other alloys \cite{Brown2019}.

A Lynx ASI detector was used to characterize solidification at a rate of 200 Hz. The ASI detector had a $d$-spacing coverage of 1.58-2.6 $\textup{\AA}$, allowing observation of the $(1000)_\alpha$,$(0002)_\alpha$ $(10\bar{1}1)_\alpha$, $(10\bar{1}2)_\alpha$, $(110)_\beta$ and $(200)_\beta$ peaks. All possible peaks were rarely observed simultaneously due to the orientation of grains and the low number of grains in the sample. The detector had an azimuthal coverage of about 15$^\circ$. The diffraction images were integrated and analyzed using the General Structure Analysis Software II (GSASII) \cite{GSASII}. Single peak fits were performed on the $(1000)_\alpha$ and $(0002)_\alpha$ peaks to get $\alpha$ lattice parameters. 

The ASI detector had too small of an azimuthal/d-spacing coverage to ensure good statistics for phase fraction calculations. A 2-dimensional GE 41RT detector was used simultaneously for larger signal coverage, but with a lower refresh rate of 10 Hz. The GE detector was a 2048 pixel by 2048 pixel (200 $\times$ 200 $\mu$m$^2$ pixel size) area detector with a d-spacing range of 0.66 - 3.31 $\textup{\AA}$ and an azimuthal coverage of nearly $180^\circ$. The GE images were integrated over the full $d$-spacing and azimuthal range. Rietveld Refinement was performed to calculate phase fraction during solidification. In all cases, the final $\beta$ phase fraction was zero or close to zero within the error bars of the Rietveld refinement fit. The lattice parameter measurements performed using the ASI detector were compared to a Rietveld refinement fit on the GE detector. Both fits produced comparable values within each other's range of uncertainty.

Volume-averaging effects need to be taken into account because the experiment operated in transmission mode. Transmission X-ray diffraction produces data from all locations in the sample meeting the Bragg condition. This means that scattering/diffraction signals will be simultaneously measured for volumes near the surface as well as in the bulk of the sample. If different phenomenon are occurring at different depths from the sample surface -- such as differences in transformation temperature, differences in cooling rate, etc. -- then the signal observed will contain information from all these locations; however, the signals overlap and obscure the behavior of individual grains.

Furthermore, Rietveld refinement assumes a certain crystal model and cannot always account for anisotropies in the results. For example, if the $\alpha$ phase has different coefficients of thermal expansion for the $a$ axis and $c$ axis then the $\left(1000\right)$ and $\left(0001\right)$ peaks will move in $d$ spacing at different rates.

Along with thermal anisotropies, full integration of the images obscures the behavior of different grains. Two grains may have lattice planes in the diffraction condition but at different locations in the sample. This likely means that spots will appear near the same $d$ spacing range but at a different azimuthal angle. If the internal temperature of the two grains is different, they will be at slightly different $d$ spacing values. Integrating over the azimuth will produce a wide peak since it is taking the sum of two peaks that almost, but not exactly, overlap.

To account for both anisotropy in the thermal expansion and thermal gradients within the sample, the ASI images were integrated over regions corresponding to individual grains. Individual grains were identified by individual spots which appeared on the detector. Figure \ref{peaksplit} demonstrates one such region in the first row. Each image is actually a subregion of a larger detector image which showed multiple spots. The histograms in the second row were produced by only integrating over this part of the image. Single peak fits were made for these histograms to monitor the behavior of individual grains.

\begin{figure}
    \centering
    \includegraphics[width=1\linewidth]{/Users/njohnson/git/thesis/document/chapters/tiweldbead/tiweldbead_images/APL_figure_microscopy.pdf}
    \caption{Optical and scanning electron microscopy of the weld bead characterized. Optical microscopy in (a) and (b) show the fine, acicular (needlelike) structure of $\alpha'$ grains contained within the former $\beta$ grain boundary. }
    \label{microscopy}
\end{figure} 

Of particular interest to this study was observing the mechanism of phase change from high temperature to low temperature because it impacts grain morphology and, therefore, the mechanical performance of the material. 

Figure \ref{fig:latparams} shows the d-spacing of $\left(120\right)_\beta$, $\left(1000\right)_\alpha$, and $\left(0001\right)_\alpha$. After the $\beta$ phase formed from solidification of the liquid melt its lattice parameter decreased due to cooling. Upon formation of the $\alpha'$ phase, however, the $\left(110\right)_\beta$ plane exhibited a knee in its $d$-spacing.  The $\left(0001\right)_\alpha$ initially forms with a d-spacing near that of the $(110)_{\beta}$ peak. However, the geometry of the $\alpha'$ unit cell quickly shifts away from the $(110)_\beta$ equilibrium value to include a longer $c$ axis and shorter $a$ axis. This is likely a more stable configuration for the $\alpha'$ unit cell. 

During this shift, the $\beta$ lattice parameter is constant. As mentioned previously, $\beta$ Ti-6Al-4V has demonstrated a decrease in lattice parameter when its $V$ content increases. Thus, if diffusion of $V$ into the $\beta$ unit cell is occurring there must be a separate external force acting against unit cell contraction. It is possible that the $\beta$ unit cell is kept at a constant volume by changes in the $\alpha'$ unit cell which shares a boundary according to the $(110)_\beta||(0001)_\alpha$ orientation relationship. This orientation relationship is demonstrated in Figure \ref{peaksplit}, where the $(0002)_\alpha$ peak can be seen splitting off of the $(110)_\beta$ peak.

However, the authors believe that diffusion did not occur, as evidenced by the lack of $\beta$ phase at room temperature and the lack of a shrinkage in $\beta$ lattice parameter upon formation of the $\alpha$ phase. Rather, $\beta$ and $\alpha$ coexistence occurs because different parts of the sample are experiencing a martensitic transformation at different times. The $\beta$ phase can coexists with $\alpha$ inside a prior $\beta$ grain as the shear transformation progresses throughout the prior $\beta$ crystal.

To further confirm the hypothesis that the solidified material was entirely martensitic $\alpha'$, optical microscopy and electron beam backscatter detection (EBSD) were used to characterize the microstructure. These images can be seen in Figure \ref{microscopy}. Martensitic HCP Titanium is often characterized by its morphology: long, acicular (needlelike) grains. Optical microscopy in subpanels (a) and (b) of Figure \ref{microscopy} reveal the needlelike structure inside of prior $\beta$ grains. While there is a transition to equiaxed $\alpha$ grains near the substrate, diffraction images were not taken here.

Scanning electron microscopy reveals a grain size of roughly $1-2 \mu$m in width and $5-10 \mu$m in length. A low confidence index was achieved during microscopy, which can be due to a wide variety of reasons including a large amount of strain on the $\alpha'$ lattice which would cause a poor fit with a structure match of equilibrium HCP Ti-6Al-4V. EBSD was also used to index the crystal structures present in the weld beads. No $\beta$ phase was identified with any reasonable confidence index in the samples analyzed, further indicating that it was not present in the regions examined. The lack of $\beta$ at room temperature in repeated HEXRD measurements combined with its lack of presence in EBSD scans indicates that the structure may be entirely martensitic $\alpha'$, with no diffusional transformation to $\beta$ possible.

The $\beta \to \alpha$ transition measured is different than that typically published for martenstitic transformations of $\alpha/\beta$ alloys. The mechanism of transformation from BCC to HCP by a shear transformation was first characterized by Burgers in 1934 \cite{Burgers1934}. Normally, only the shear transformation along the $\left[111\right]$ direction is described without the secondary relaxation along the $a$ and $c$ axes. What we have observed constitutes a two part unit cell transformation; first, the BCC unit cell shears into the HCP unit cell in a martensitic transformation. Then, the HCP unit cell distorts. The reason for the secondary HCP unit cell distortion is unknown, but conjectures about its origin can be made.

Upon initial formation, the HCP unit cell has a c/a ratio of 1.566. This is far from the ideal close packed hexagonal c/a ratio of $\sqrt{8/3} \approx 1.63$. The HCP unit cell undergoes an extension along the $c$ direction and shrinkage along the $a$ direction, distorting the unit cell to a $c/a$ ratio of 1.59, closer to the ideal ratio. Discussing the equilibrium $c/a$ ratio for Ti-6Al-4V can be a somewhat confusing because the $c$ and $a$ lattice parameters depend on a multitude of factors including composition of the $\alpha$ phase, grain morphology, and residual stresses in the sample. Thus, it makes more sense to compare the $c/a$ ratio observed to $c/a$ ratios measured in samples of Ti-6Al-4V manufactured other ways. Stapleton et al. used HEXRD to measure $c/a = 1.597$ for forged Ti-6Al-4V \cite{Stapleton2008}. Tan characterized Ti-6Al-4V manufactured through electron beam melting and found $c/a = 1.595$ \cite{Tan2015}. Xu et al. published a wide variety of $c/a$ ratios depending on the $V$ concentration of the $\beta$ phase. They found values ranging from $c/a = 1.558$ for V-rich HCP phases and $c/a=1.6$ for V lean HCP phases. Since no $\beta$ phase was present at room temperature for these samples it is impossible to comment on the $V$ content of the two phases based on lattice parameter alone. 

High energy X-ray diffraction revealed lattice parameter changes in the martensitic formation of HCP $\alpha'$ Ti-6Al-4V. When the transformation occurs the $\alpha'$ phase initially forms with a $(0001)_{\alpha'}$ d-spacing equal to the $(110)_{\beta}$ d-spacing. However, the unit cell quickly shifts to a more equilibrium geometry, with an extended $c$ axis and shrunk $a$ axis. Following this initial shift both phases continue to shrink in volume due to cooling. This observation has implications for modeling phase transformations of Ti-6Al-4V during additive manufacturing. In particular, it reveals that the microstructure of rapidly solidified small volumes is entirely martensitic. It also indicates that constrains on the $\beta$ unit cell may be due to the $(0001)_{\alpha}||(110)_\beta$ orientation relationship. 

