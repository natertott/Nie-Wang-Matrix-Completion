\documentclass[nofootinbib,notitlepage]{revtex4-1}
\usepackage{amsmath}
\usepackage{amssymb}
\usepackage{graphicx}
\usepackage{subcaption}
\usepackage{xcolor}

\begin{document}
\title{Phase Transformations of Rapidly Solidified Ti-6Al-4V}
\author{Nathan SDEF Johnson}
\affiliation{Department of Mechanical Engineering, Colorado School of Mines, Golden, CO 80401}
\affiliation{Los Alamos National Laboratory, Los Alamos, NM 87544}

\author{Adam Pilchak}
\affiliation{Air Force Research Laboratory, Dayton, OH}

\author{Branden B. Kappes}
\affiliation{KMMD, Castle Rock, CO}

\author{Behnam Amin-Ahmadi}
\affiliation{Department of Mechanical Engineering, Colorado School of Mines, Golden, CO, 80401}

\author{Craig A. Brice}
\affiliation{Department of Mechanical Engineering, Colorado School of Mines, Golden, CO 80401}

\author{Aaron P. Stebner}
\affiliation{Departments of Mechanical Engineering and Materials Science, Georgia Institute of Technology, Atlanta, GA}

\begin{abstract}
There has been decades of literature published on the transformation from BCC to HCP titanium through a martensitic transformation. The martensitic HCP structure is often referred to as a the $\alpha'$ phase, as opposed to the diffusionally formed BCC $\beta$ phase and HCP $\alpha$ phase. In this article, a theory is presented that contends the formation of HCP Ti-6Al-4V through a martensitic phase transformation. First, literature is reviewed to show inconsistencies in theories of the martensitic transformation of the $\alpha'$ phase. Next, thermodynamic models are presented that show there is a driving force for V to segregate to the $\beta$ phase even at the fastest cooling rates. Finally, experimental evidence is presented that demonstrates the segregation of V to the $\beta$ phase even in rapid solidification conditions (10$^7$ C/s). This evidence challenges current literature stating that a martensitic phase transformation occurs in Ti-6Al-4V.
\end{abstract}
\maketitle

\textbf{Outline}
\begin{itemize}
	\item The citation history of the martensitic phase transformation in Ti-6Al-4V \begin{itemize}
		\item The original citation of JC Williams in CP Ti dating back to the Titanium Olympics in 1968
		\item Papers that cite JC Williams original paper
		\item Papers that have a daisy-chain of citations 
		\item Papers that have done characterization of the martensitic phase transformation in Ti-6Al-4V
		\item Models of the martensitic phase transformation
		\end{itemize}
	\item The thermodynamic driving force for V diffusion in Ti-6Al-4V \begin{itemize}
		\item Present Branden's model for V diffusion
		\end{itemize}
	\item The composition of phases in Ti-6Al-4V at different cooling rates \begin{itemize}
		\item Take Behnam's microscopy and show the partitioning of V in Ti-6Al-4V at different cooling rates
		\item Make the argument that the concentration of V in the $\beta$ phase is indicative of diffusional transformation
		\end{itemize}
	\item Discussion \begin{itemize}
		\item Current phase diagrams of Ti-6Al-4V from ThermoCalc vs. published phase diagrams with martensitic start temperatures
		\item Transformation routes in Ti-6Al-4V \begin{itemize}
			\item $\beta \to \alpha + \beta$?
			\item $\beta \to \alpha' \to \alpha + \beta$
			\item $\beta \to \alpha' + \beta \to \alpha + \beta$?
			\end{itemize}
		\item We make the argument that the first transformation is always the one that occurs in Ti-6Al-4V
		\item The fact that $\beta$ is always present in samples cooled at any rate, coupled with the theory that V should diffuse distances up to mm at any temperature
		\end{itemize}
\end{itemize}

\section{The History of the Martensitic Transformation in Ti-6Al-4V}
The origin of publications on the martensitic phase transformation in Titanium alloys begins with a seminal paper by J.C. Williams, published in 1967, which compares the phase transformation of Ti to that of Zr, both of which transform from a high-temperature BCC structure to a room-temperature HCP structure \cite{Williams1967}. Williams would go on to publish another article in 1972 that furthered the concept of a martensitic phase transformation at high cooling rates \cite{Williams1972}. 

\section{Thermodynamic Driving Force for the Diffusion of Vanadium at Temperatures and Cooling Rates}

\section{The Partitioning of Alloying Elements}

\section{Conclusion}

\bibliography{martensiteTi}

\end{document} 
