%~~~~~~~~~~~~~~
\section{Introduction}
Continuous metal micro lattice structures have been theorized and described since the 1990s. These structures are able to provide specifically engineered mechanical properties, like stiffness and strength, at a fraction of the density of bulk metals. Early adoptions of the lattice structure include the related open and closed cellular foam materials \cite{Bonatti2017} as well as the snap-fit and weld joined lattices \cite{Dong2015}. 

In 2001, Deshpande, Fleck, and Ashby (DFA) presented a continuum theory for one type of lattices structure: the octet truss \cite{Deshpande2001}. Octet truss lattice structures are analogous to face centered cubic crystal structures, where the nodes of the lattice correspond to atom sites. The DFA model provides predictions for macroscopic stress and strain in the trusses, as well as calculations of important mechanical properties like yield strength, buckling strength, and failure modes. In particular, the DFA model predicts that the mechanical deformation in octet trusses should be stretch-dominated first, with buckling occurring as a secondary mechanism. While the DFA model is robust, the technology at the time of its introduction was limited to characterize such complex structures. Advancements in imaging analysis and computing have made characterization of these structures possible. Furthermore, advancements in additive manufacturing (AM) have allowed for these complex structures to be manufactured with little machining and for arbitrarily large and complex geometries to be built.

The combination of a unique geometry with the octet truss and the additive manufacturing process requires multiple approaches for characterization. The geometry of the octet truss means that models like the DFA model can be used to assess mechanical behavior, specifically failure of the part. These continuum models can be compared against computational models using finite element analysis or against macroscopic stress-strain response observed. However, the additive manufacturing process adds another complication into the mix. The additive process is known to create unique microstructures. Additive manufacturing impacts grain morphologies \cite{Tan2015, Zhu2018}, dislocation density \cite{Zhang2015, Wang2017, Gallmeyer2020} crystallographic texture \cite{Wang2019}, phase fractions \cite{Gallmeyer2020}, and defect structures \cite{Yang2017,Matthews2016}, all of which impact the material structure-property relationship and therefore the mechanical performance. 

Several studies have performed mechanical testing or finite element modeling of octet trusses and, in some cases, both. Dong et al, examined the macroscopic response of snap-fit octet truss lattices and compared the results to the DFA model, finding that snap-fit octet truss lattices moduli are well predicted by the DFA model \cite{Dong2015}. Latture, Begley, and Zok evaluated the mechanical performance of ideal lattice structures with octet trusses being one amongst many studied \cite{Latture2017}. Bonatti and Mohr looked at macroscopic mechanical measurements of large (5x5x5 unit cell) octet truss structures and characterized the failure mechanisms \cite{Bonatti2017}. Tancogne-Dejean did a comprehensive study that compared macroscopic stress-strain results for an octet truss to finite element modeling \cite{Tancogne-Dejean2016}. Their approach examined different unit cell densities, strut thicknesses, and strut shapes and the impact on mechanical response. Tancogne-Dejean was able to achieve decent agreement in failure modes for different densities between a finite element model and additively manufactured samples. 

Other studies have examined how additive manufacturing impacts lattice materials. Yan et al. investigated additively manufactured gyroid structures, a different kind of period lattice material \cite{Yan2012}. Yan was primarily concerned with the manufacturability of the structure and its impact on surface roughness, shape, and deformation. They found that some cracking occurred in the samples during printing due to residual stress buildup. Tancogne-Dejean used selective laser melting to build and assess the macroscopic response of octet truss samples \cite{Tancogne-Dejean2016}. They found that changes in morphology and texture of the sample can result in mechanical property differences up to 20\%. Liu et al. studied the role of geometric imperfections of the struts due to the SLM process on octet truss lattices \cite{Liu2017}. They counted a statistical distribution of strut deformities based on how far the strut deviated from a perfect circle. These distributions of imperfections were then introduced into a finite element model which varied the beam width and thickness in order to capture imperfections. They found a decrease in Young's modulus in all directions along the struts as a result of the imperfections. This was linked back to the manufacturing process by the realization that struts in the orientation of the build plane were oversized.

These studies are all concerned with comparing continuum level models of octet trusses to macroscopic responses in the parts. What is often unclear is how the behavior of an individual unit cell impacts the macroscopic behavior of the samples. To this end, several authors have implemented homogenization theories which connect unit-cell-level mechanical properties to macroscopic mechanical properties. Park et al. developed a homogenization scheme which takes into account deformations at the unit cell scale due to build processes in fused deposition modeling and predicts mechanical performance at the macroscale \cite{Park2014}. Vigliotti et al. used the representative volume element approach to develop a homogenization scheme which linked discrete elements (unit cells) to the macroscopic behavior of arbitrarily shaped lattice structures \cite{Vigliotti2014}. Mohr used a similar approach which treated the macroscopic part as a homogenous medium and calculated mechanical properties based on the mechanics of unit cells \cite{Mohr2005}. In all cases, however, the unit cell is treated as a  a perfect material.

The success of homogenization schemes is built upon having accurate properties for the representative volume elements. Consideration of the additive manufacturing process including deformation, defects, microstructure, texture, and more must be made for models to accurately predict macroscopic deformation behavior of lattice materials. To this end, it is beneficial to examine how additive manufacturing impacts an individual unit cell.

In this case, samples were manufactured out of Ti-5V-5Mo-5Hf-3Cr (Ti-5553). Some studies have looked at the impact of additive manufacturing on Ti-5553 specifically, such as Schwab et al. \cite{Schwab2016}. Characterization of the phases resulting from the AM process will be especially important because Ti-5553 is known to have unexpected secondary phases form \cite{Dehghan-Manshadi2011, Zheng2016}. Post-processing heat treatments will be equally important to characterize, as the heat treatment schedule can significantly vary mechanical properties like Young's modulus and yield strength \cite{Kar2014}.

This study is unique because it combines all of the above characterizations into a single picture to evaluate how the additive manufacturing process impacts the expected mechanical performance of two unit cells of a Ti-5553 octet truss lattice structure. Based on the nomenclature of Latture, Begley, and Zok \cite{Latture2019} the sample geometry is $\{2\text{FCC}\}^1$ The present investigation combines high energy X-ray diffraction, digital image correlation, finite element modeling, and fractography to evaluate the mechanical performance, texture, strain distribution, failure mechanisms, and residual stresses of the parts. 

The study herein takes a processing-structure-property approach to characterize the mechanical response of individual unit cells of additively manufactured Ti-5553 octet truss lattices. While previous studies have evaluated macroscopic behavior of lattices and compared them with continuum level predictions of mechanical properties, no study has yet investigated individual unit cells. The measurement and characterization of individual unit cells is important for verifying the many homogenization schemes which exist for these parts.
