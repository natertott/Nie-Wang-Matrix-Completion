\documentclass[twocolumn,nofootinbib]{revtex4-1}
\usepackage{amsmath}
\usepackage{amssymb}
\usepackage{graphicx}
\usepackage{floatrow}
\usepackage{subcaption}
\usepackage{xcolor}


\begin{document}
\title{Mechanical Behavior of Additively Manufactured Ti-5553 Octet Truss Lattices}
\author{Nathan S. Johnson}
\affiliation{Colorado School of Mines}
\affiliation{Los Alamos National Laboratory}

\author{Maria Strantza, Ibo Matthews}
\affiliation{Lawrence Livermore National Laboratory}

\author{Dave MacKnelly}
\affiliation{Atomic Weapons Establishment}

\author{Jun-Sang Park, Peter Kenesei}
\affiliation{Argonne National Laboratory}

\author{Bj\o rn Clausen, Donald W. Brown, John S. Carpenter}
\affiliation{Los Alamos National Laboratory}

\author{Craig A. Brice, Aaron P. Stebner}
\affiliation{Colorado School of Mines}

\begin{abstract}
We used various characterization techniques to assess the response of additively manufactured Ti-5553 octet truss lattice structures under compressive load.
\end{abstract}
\date{\today}
\maketitle

%~~~~~~~~~~~~~~
\section{Introduction}
\textbf{Figure: The samples manufactured, their orientation on the build plate, and maybe their microstructure(?)}

%~~~~~~~~~~~~~~~~~~~~~~~~~~~
\section{Characterization Techniques}

%~
\subsection{High Energy X-ray Diffraction}
\subsubsection{Experimental Setup}
\textbf{Figure: The orientation of the sample in the room at APS}
\subsubsection{Stress and Strain Calculations}
\textbf{Equations: I.C. Noyan's theory on calculation of crystal stresses from measured strains}.

%~
\subsection{Finite Element Modeling}
\subsection{Ex Situ Characterization}
\subsubsection{X-ray Computed Tomography}
\subsubsection{Microscopy and Fractography}

%~~~~~~~~~~~
\section{Results}
\subsection{Stress and Strain Measurements with HEXRD}
\textbf{Figure: A plot showing the sample and the stress/strain response at a few different locations around the sample.}
\subsection{FEA Model Without and With Residual Stresses}
\textbf{Figure: The X and Y strains calculated at maximum load  and how they compare with/without residual stress applied.}
\subsection{Void Nucleation After Compression and Failure Mechanism}
\textbf{Figure: Show the CT images before and after compression, maybe show the distribution as well.}

%~~~~~~~~~~~~~
\section{Discussion}
\subsection{Anisotropy in Stress Distribution Due to Residual Stress}
\textbf{Figure: Show the principal y strains at 200um load side-by-side with the principal y strains from the FEA model with residual stress applied.}
\subsection{Failure Occurring in Pure Stretch}
\textbf{Figure: Show the fractography results that Maria obtained.}

\bibliography{bib.bib}
\end{document}


